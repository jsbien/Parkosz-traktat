% lualatex ze wzgledu na OCG
%
\documentclass[dvipsnames,12pt]{article}
\usepackage{polyglossia}
\setdefaultlanguage{polish}

\usepackage[a4paper,landscape]{geometry}
\usepackage[doublespacing]{setspace}
\usepackage{fontspec}

\usepackage{graphicx}
\usepackage{xcolor}

\usepackage{soul}

\usepackage{ocgx}
\usepackage{zref-savepos}
\usepackage{xifthen}
\usepackage{xspace}

% https://williewong.wordpress.com/2013/12/04/automated-annotation-in-latex-using-ocg/
% Automated annotation in LaTeX using OCG
% by Willie Wong

%%%%% Annotation utils %%%%%
% The goal is to provide a clickable
% tool-tip like interface to show cross reference information. 
% This duplicates the function of the 'fancy-preview' and
% 'fancytooltips' packages, and does not support mouse-over events,
% but has the advantage of working in evince also. 

% http://en.wikibooks.org/wiki/LaTeX/Colors#The_68_standard_colors_known_to_dvips

% We need some packages
%\usepackage[usenames,dvipsnames]{color}
\usepackage{zref-savepos}
\usepackage{xifthen}
\usepackage{xspace}
% the package ocgx also depends on ocg-p, I think.

% Some configuration stuff
% Default colours, see
% http://en.wikibooks.org/wiki/LaTeX/Colors
\newcommand*\annotatetextcolour{Red} % For the inline text
% \newcommand*\annotatetextcolour{OliveGreen} % For the inline text
\newcommand*\annotateboxbordercolour{Dandelion} % For the annotate box
\newcommand*\annotateboxbkgdcolour{Goldenrod} % Ditto
\newcommand*\annotateboxtextcolour{Black} % Ditto
\newcommand*\annotateboxtextfont{\small} % Other font configuration stuff
\newcommand*\annotateheremarktext{note} % Default text for \annotatehere 
\newcommand*\annotatewithmarkmark{$\Uparrow$} % Default mark for \annotatewithmark

\makeatletter
% Dummy variables
\newcounter{@anntposmark}
\newlength\@annt@oldfboxsep

% Usage: \annotatetext{text}{annotation}
% Note: there can be some bugs when the text to be annotated starts
% near the end of a line. The other two commands seems to behave
% better, but will still need extensive testing!
\newcommand\annotatetext[2]{%
        \stepcounter{@anntposmark}%
        \zsavepos{@annt@pos\the@anntposmark}%
        \hskip\dimexpr - \zposx{@annt@pos\the@anntposmark}sp + \zposx{@anntleftmargin}sp + 3em%
        \smash{\raisebox{3ex}{\makebox[0pt][l]{\begin{ocg}{Annotation Layer \the@anntposmark}{anntlayer\the@anntposmark}{0}\fcolorbox{\annotateboxbordercolour}{\annotateboxbkgdcolour}{\parbox[b]{\textwidth-6em}{{\annotateboxtextfont\color{\annotateboxtextcolour} #2}}}\end{ocg}}}}%
        \hskip\dimexpr + \zposx{@annt@pos\the@anntposmark}sp - \zposx{@anntleftmargin}sp - 3em%
        \switchocg{anntlayer\the@anntposmark}{{\color{\annotatetextcolour}#1}}%
        \xspace}

\newcommand\annotatetextBlue[2]{%
        \stepcounter{@anntposmark}%
        \zsavepos{@annt@pos\the@anntposmark}%
        \hskip\dimexpr - \zposx{@annt@pos\the@anntposmark}sp + \zposx{@anntleftmargin}sp + 3em%
        \smash{\raisebox{3ex}{\makebox[0pt][l]{\begin{ocg}{Annotation Layer \the@anntposmark}{anntlayer\the@anntposmark}{0}\fcolorbox{\annotateboxbordercolour}{\annotateboxbkgdcolour}{\parbox[b]{\textwidth-6em}{{\annotateboxtextfont\color{\annotateboxtextcolour} #2}}}\end{ocg}}}}%
        \hskip\dimexpr + \zposx{@annt@pos\the@anntposmark}sp - \zposx{@anntleftmargin}sp - 3em%
%        \switchocg{anntlayer\the@anntposmark}{{\color{Blue}#1}}%
        \switchocg{anntlayer\the@anntposmark}{{\color{NavyBlue}#1}}%
        \xspace}

\newcommand\annotatetextOlive[2]{%
        \stepcounter{@anntposmark}%
        \zsavepos{@annt@pos\the@anntposmark}%
        \hskip\dimexpr - \zposx{@annt@pos\the@anntposmark}sp + \zposx{@anntleftmargin}sp + 3em%
        \smash{\raisebox{3ex}{\makebox[0pt][l]{\begin{ocg}{Annotation Layer \the@anntposmark}{anntlayer\the@anntposmark}{0}\fcolorbox{\annotateboxbordercolour}{\annotateboxbkgdcolour}{\parbox[b]{\textwidth-6em}{{\annotateboxtextfont\color{\annotateboxtextcolour} #2}}}\end{ocg}}}}%
        \hskip\dimexpr + \zposx{@annt@pos\the@anntposmark}sp - \zposx{@anntleftmargin}sp - 3em%
        \switchocg{anntlayer\the@anntposmark}{{\color{OliveGreen}#1}}%
        \xspace}

\newcommand\annotatetextMagenta[2]{%
        \stepcounter{@anntposmark}%
        \zsavepos{@annt@pos\the@anntposmark}%
        \hskip\dimexpr - \zposx{@annt@pos\the@anntposmark}sp + \zposx{@anntleftmargin}sp + 3em%
        \smash{\raisebox{3ex}{\makebox[0pt][l]{\begin{ocg}{Annotation Layer \the@anntposmark}{anntlayer\the@anntposmark}{0}\fcolorbox{\annotateboxbordercolour}{\annotateboxbkgdcolour}{\parbox[b]{\textwidth-6em}{{\annotateboxtextfont\color{\annotateboxtextcolour} #2}}}\end{ocg}}}}%
        \hskip\dimexpr + \zposx{@annt@pos\the@anntposmark}sp - \zposx{@anntleftmargin}sp - 3em%
        \switchocg{anntlayer\the@anntposmark}{{\color{Magenta}#1}}%
        \xspace}



% Usage: \annotatehere[note mark text]{annotations}
% The default mark text is set above, the mark text is type-set in a superscript box
\newcommand\annotatehere[2][\annotateheremarktext]{%
        \stepcounter{@anntposmark}%
        \zsavepos{@annt@pos\the@anntposmark}%
        \hskip\dimexpr - \zposx{@annt@pos\the@anntposmark}sp + \zposx{@anntleftmargin}sp + 3em%
        \smash{\raisebox{3.3ex}{\makebox[0pt][l]{\begin{ocg}{Annotation Layer \the@anntposmark}{anntlayer\the@anntposmark}{0}\fcolorbox{\annotateboxbordercolour}{\annotateboxbkgdcolour}{\parbox[b]{\textwidth-6em}{{\annotateboxtextfont\color{\annotateboxtextcolour}#2}}}\end{ocg}}}}%
        \hskip\dimexpr + \zposx{@annt@pos\the@anntposmark}sp - \zposx{@anntleftmargin}sp - 3em%
        \setlength\@annt@oldfboxsep\fboxsep%
        \setlength\fboxsep{1pt}%
        \switchocg{anntlayer\the@anntposmark}{\raisebox{1ex}{\fcolorbox{\annotatetextcolour}{White}{\color{\annotatetextcolour}\tiny \scshape #1}}}%
        \setlength\fboxsep\@annt@oldfboxsep%
        \xspace}

% Usage: \annotatewithmark[mark]{annotations}
% Very similar to \annotatehere, but the mark is unframed, and no conversion made to small caps
\newcommand\annotatewithmark[2][\annotatewithmarkmark]{%
        \stepcounter{@anntposmark}%
        \zsavepos{@annt@pos\the@anntposmark}%
        \hskip\dimexpr - \zposx{@annt@pos\the@anntposmark}sp + \zposx{@anntleftmargin}sp + 3em%
        \smash{\raisebox{3.3ex}{\makebox[0pt][l]{\begin{ocg}{Annotation Layer \the@anntposmark}{anntlayer\the@anntposmark}{0}\fcolorbox{\annotateboxbordercolour}{\annotateboxbkgdcolour}{\parbox[b]{\textwidth-6em}{{\annotateboxtextfont\color{\annotateboxtextcolour}#2}}}\end{ocg}}}}%
        \hskip\dimexpr + \zposx{@annt@pos\the@anntposmark}sp - \zposx{@anntleftmargin}sp - 3em%
        \setlength\@annt@oldfboxsep\fboxsep%
        \setlength\fboxsep{1pt}%
        \switchocg{anntlayer\the@anntposmark}{\raisebox{1ex}{{\color{\annotatetextcolour}\tiny #1}}}%
        \setlength\fboxsep\@annt@oldfboxsep%
        \xspace}

% Usage: \annotatelabel{label}{annotations}
% Replacement for \label, where attached to each label there is an annotation 
% text which can be recalled using \annotateref{label}. See below. 
\newcommand\annotatelabel[2]{%
        \label{#1}%
        \global\@namedef{@annt@label@#1}{#2}}
% Usage: \annotateref{label}  and  \annotateeqref{label}
% Replacement for \ref and \eqref, where we insert an \annotatewithmark with 
% the text set with \annotatelabel
\newcommand\annotateref[1]{\ref{#1}\annotatewithmark{\@nameuse{@annt@label@#1}}}
\newcommand\annotateeqref[1]{\eqref{#1}\annotatewithmark{\@nameuse{@annt@label@#1}}}

\AtBeginDocument{\zsavepos{@anntleftmargin}} %This figures out where the left margin is
\makeatother


%%% Local Variables: 
%%% mode: latex
%%% TeX-master: shared
%%% End: 



\usepackage{longtable}
\usepackage{hyperref}

\usepackage{fancyvrb}

\usepackage{relsize}

\usepackage{draftwatermark}


\setmainfont[Mapping=tex-text]{DejaVu Serif}
\setsansfont[Mapping=tex-text]{DejaVu Sans}
\setmonofont{DejaVu Sans Mono}


\newfontfamily\Cardo{"Cardo"}

\font\Quivira="Quivira"

% http://scholarsfonts.net/cardofnt.html

\font\Parkosz="Parkosz"
\def\largeParkosz{\fontspec[Scale=1.5]{Parkosz}}

\newcommand{\symbolParkosz}[1]{{\largeParkosz\symbol{#1}}}

\newcommand{\sParkosz}[1]{{\fontspec[Scale=3.0]{Parkosz}\symbol{#1}}}

\newcommand{\parkosz}[1]{\colorbox{lime}{\textit{#1}}}
\newcommand{\PARKOSZ}[1]{{\textit{#1}}}

% za ciemny:
% \newcommand{\almanica}[1]{\colorbox{lightgray}{\textit{#1}}}
\newcommand{\almanica}[1]{\colorbox{pink}{\textit{#1}}}



%hh hyphenation head
%ht hyphenation tail

% dc dropped capital

\newcommand{\dc}[1]{{\relsize{3}\colorbox{black!15}{\textbf{#1}}}}

\newcommand{\hyphh}[2]{#1\textcolor{blue}{#2}}
\newcommand{\hypht}[2]{\fbox{#2}}

% http://wbl.klf.uw.edu.pl/13/2/iParkosz.djvu?djvuopts=&page=37&zoom=width&showposition=0.5,0.55

% marginalia
% zapisy na marginesie i nad rządkami
%\newcommand{\margin}[1]{\colorbox{blue!10}{\{#1\}}}
\newcommand{\margin}[1]{\annotatetextBlue{\{#1\}}{zapisy na marginesie}}


% \renewcommand{\over}[1]{\colorbox{blue!10}{\{#1\}}}

\renewcommand{\over}[1]{\annotatetextBlue{\{#1\}}{zapisy nad rządkami}}

% litery i wyrazy dodane, (których w tekście brak)
%\newcommand{\add}[1]{\colorbox{olive!10}{<#1>}}
\newcommand{\add}[1]{\annotatetextOlive{<#1>}{litery i wyrazy dodane, (których w tekście brak)}}

% litery i wyrazy zbędne
% \newcommand{\extra}[1]{\colorbox{magenta!10}{[#1]}}
\newcommand{\extra}[1]{\colorbox{magenta!10}{[#1]}}

% przekreślenia
% MATHEMATICAL LEFT WHITE SQUARE BRACKET' (U+27E6)
% 'MATHEMATICAL RIGHT WHITE SQUARE BRACKET' (U+27E7)
\newcommand{\overstr}[1]{\annotatetextMagenta{⟦#1⟧}{przekreślenia}}

% mało fontów:
% RIGHT WHITE SQUARE BRACKET (U+301B)


% wątpliwe odczytania
% confidence
% nie działa parametr specyfikacji koloru - może jakiś głupi błąd
\newcommand{\conf}[2]{\textcolor{orange}{#1}}


% wcięcia Kucały
% akapit:
\newcommand{\indentK}{\hskip2em}
% cytat
\newcommand{\indentKcyt}{\hskip5em}

% wcięcia Parkosza:
\newcommand{\indentP}{\colorbox{yellow}{\hskip2em}}

\parindent0pt

\DefineVerbatimEnvironment%
{VerbatimLatin}{Verbatim}
{commentchar=\%,numbers=left,numberblanklines=false,
commandchars=\\\{\}}

% \fboxrule2pt

% Parkosz...

\newcount\ppageno
\newcount\ppreviouspageno


\newcount\plineno
\newcount\psublineno

\newcount\pnoteno
\newcount\psecnoteno

% odstęp na oko - lepiej zmierzyć i wstawić
\def\fulllines{\everypar{\advance\plineno by 1\llap{\the\ppageno-\ifnum\plineno<10 0\fi\the \plineno \hskip 1.5em}}}

\def\fullpreviouslines{\everypar{\advance\plineno by 1\llap{\the\ppreviouspageno-\ifnum\plineno<10 0\fi\the \plineno \hskip 1.5em}}}



% liczenie pustych wierszy:
% \let\engraf\par
% \def\par{\leavevmode\endgraf}

\def\splitlines{\advance\plineno by 1\psublineno=0\everypar{\advance\psublineno by 1\llap{\textcolor{green}{\the\ppageno-\ifnum\plineno<10 0\fi\the \plineno-\the\psublineno \ }}}}

\def\newsplitline{\advance\plineno by 1}

\def\newParkoszpage{\everypar{}}



% ¹ 'SUPERSCRIPT ONE' (U+00B9)
% ² U+00B2 	SUPERSCRIPT TWO
% U+00B3 	SUPERSCRIPT THREE	
\catcode`\¹=13
\catcode`\²=13
\catcode`\³=13
\catcode`\⁴=13
\catcode`\⁵=13

% \advance\pnoteno by 1
% \def\tip{Jeden z tych przykładów winien być napisany przez b!, drugi przez ɓ.}
% \footnotetext[\the\pnoteno]{\tip}
% \def¹{\annotatehere[\the\pnoteno]{\tip}}

\def\newtip#1#2{
% \pnoteno=\psecnoteno
% \advance\pnoteno by 1
% \footnotetext[\the\pnoteno]{#1}
% \def¹{\annotatehere[\the\pnoteno]{#1}}}
\footnotetext[#1]{#2}
\def¹{\annotatehere[#1]{#2}}}



% \def\secondtip#1{
% \psecnoteno=\pnoteno
% \advance\psecnoteno by 1
% \footnotetext[\the\psecnoteno]{#1}
% \def²{\annotatehere[\the\psecnoteno]{#1}}}

\def\secondtip#1#2{\footnotetext[#1]{#2}\def²{\annotatehere[#1]{#2}}}
\def\thirdtip#1#2{\footnotetext[#1]{#2}\def³{\annotatehere[#1]{#2}}}
\def\fourthtip#1#2{\footnotetext[#1]{#2}\def⁴{\annotatehere[#1]{#2}}}
\def\fifthtip#1#2{\footnotetext[#1]{#2}\def⁵{\annotatehere[#1]{#2}}}

\parindent5em

\author{\textbf{Marian Kucała}\\(Korekta OCR, skład i transliteracja alfabetu Parkosza: Janusz S. Bień)} 

\title{PRZEKŁAD POLSKI\\Traktat o ortografii polskiej [Jakuba Parkosza]}

\date{\textbf{Licencje}:\\ Creative Commons [Wspólnota twórcza]\\
Uznanie autorstwa --- Na tych samych warunkach --- Polska\\
wersja 2.5 lub późniejsza\\\textit{także}\\
GNU Free Documentation License\\
\textbf{\today}}

\begin{document}


\maketitle
\thispagestyle{empty}
\newpage

\marginpar{s. 83}

Jezus Chrystus.

\bigskip
Walcz za ojczyznę,ponieważ jej obrona przynosi
zasłużoną sławę.

\bigskip
\newtip{1}{W nawiasy ujęto niektóre wyrazy i zwroty nie mające odpowiedników w tekście łacińskim.}
Ojczyzna oznacza tutaj społeczność, społeczność (zaś)¹ długotrwałość,
% 1	W nawiasy ujęto niektóre wyrazy i zwroty nie mające odpowiedników w tekście łacińskim.
(a) długotrwałość to, co (bardziej) pożądane, jak to pokazano w
\newtip{2}{Arystotelesa.}
trzeciej księdze \textit{Topików}¹. Stąd wynika pogląd uświęcony
% 2	Arystotelesa.
niewzruszoną powagą ojców (Kościoła) i filozofów, że lepiej w pocie
czoła pracować dla rzeczypospolitej niż służyć czyjemukolwiek
pożytkowi prywatnemu. Bo jeśli się dobrze ma społeczność, dobrze się
\newtip{3}{Arystoteles.}
ma (również) jej część, a nie odwrotnie. Nie przeczy temu Filozof¹ w
% 3	Arystoteles.
\newtip{4}{Tj. \textit{Kategoriach}.}
\textit{Predikamentach}¹, mówiąc: Gdyby nie istniały elementy pierwsze,
%  4	Tj. Kategoriach.
tj. proste substancje, nie mogłoby też istnieć to, co się z nich
składa, tj. ogół — ponieważ to trzeba rozumieć w znaczeniu naturalnym
lub, jeśli ktoś woli, moralnym.
Dlatego też i Rzymianie długo zachowywali zwyczaj wyróżniania
zwycięzcy w walce za ojczyznę różnymi zaszczytami i licznymi darami, o
czym się może przekonać każdy, kto zechce czytać ich dzieje. 
To samo
ma na myśli Teodolus w swoich 
\newtip{5}{Theodul żyjący w X w. napisał dzieło pt. \textit{Ecloga, qua comparatur miracula Novi Testamenti cum veterum poetarum commentis}.}
\textit{Eklogach}¹, mówiąc (ekloga IV):
% 5	Theodul żyjący w X w. napisał dzieło pt. Ecloga, qua comparałur miracula Novi Testamenti cum veterum poetarum commentis.
\begin{quote}

Ponad pochwały ludzkie wzniósł się ten, co pierwszy\\
U stóp góry Olimpu igrzyska stanowił.\\
\marginpar{s. 84}
Odtąd w laur uwieńczony zwycięzca powracał\\ 
Tryumfalnie do domu, zwyciężonemu zaś wstyd towarzyszył.
  
\end{quote}

\bigskip

Ze względu też na tę najbardziej zdrową i niejako z konieczności
wynikającą zasadę był zwyczaj nie tylko u chrześcijan, ale również u
pogan — jak poświadcza Arystoteles — że ilekroć zginął ktoś z
walczących za społeczność, to go społeczność owa zawsze grzebała z
wielkimi zaszczytami, a jego potomków, to jest synów i córki,
obdarzała nieustannie dowodami życzliwości i wdzięczności. Jest bowiem
jasne, że żołnierzowi broniącemu ojczyzny należy się sława.

\bigskip

\newtip{6}{6 Bardziej odpowiednie byłoby używanie w przekładzie tego
  akapitu pierwszej osoby: ja\ldots uznałem itd. — wprowadzenie tej formy
  sugerował p. prof. J. Domański (recenzent opracowania). Byłoby to
  jednak niemal uznaniem za autora tych słów — samego Parkosza (czego
  zresztą wykluczyć nie można).}
Tak więc i my¹, kierując się wspomnianymi napomnieniami — nie dla
% 6 Bardziej odpowiednie byłoby używanie w przekładzie tego akapitu
% pierwszej osoby: ja... uznałem itd. — wprowadzenie tej formy sugerował
% p. prof. J. Domański (recenzent opracowania). Byłoby to jednak niemal
% uznaniem za autora tych słów — samego Parkosza (czego zresztą
% wykluczyć nie można).
zyskania chwały, gdyż ta się należy samemu Bogu, ale przez wzgląd na
zależne od jego woli dobro społeczne — uznaliśmy za słuszne z
przyczyn, o których będzie mowa niżej, narzecze ojczyste, które rzecz
jasna jest językiem Polaków, wyrażać w piśmie literami łacińskimi z
dodaniem niewielu znaków odmiennych. Chcielibyśmy bowiem, pracując nad
udoskonaleniem teraźniejszego złego i nieodpowiedniego sposobu
pisania, zostać w tym zakresie uznanymi za broniących dobra
powszechnego i wprowadzających odpowiadający potrzebom sposób
pisania.

Lecz zanim się zacznie przedstawiać rzecz samą, należy dodać
wstępne wyjaśnienia, ażeby ci, którzy poznają jej potrzebę, tym
chętniej ją przyjęli i mogli się śmiało przeciwstawić takim, którzy są
jej niechętni. Znajdzie się tu również uzasadnienie, dlaczego z tej
rzeczy winno się owocnie korzystać.

Niech hasłem naszego założenia będą słowa Platona, boskiego filozofa,
\newtip{7}{Tj. w \textit{Tymeuszu}.}
(napisane) w \textit{Timajosie}¹1: Na to jest nam dana mowa, by sądy
% 7	Tj. w Tymeuszu.-84-
naszego umysłu mogły się uzewnętrzniać. A trzeba
\marginpar{s. 85}
zauważyć, że w słowie „sądy” chodzi o społeczność ludzką, o której
mówi Arystoteles w pierwszej księdze \textit{Polityki}: Człowiek jest
stworzeniem politycznym, to jest społecznym, oswojonym i domowym,
zdolnym do przekazywania swych pojęć innym, a to przez znaki
głosowe. To wyraża również w rozdziale pierwszym
\newtip{8}{Traktatu \textit{O zdaniu}.}
\textit{Hermeneutyki}¹, mówiąc: Istnieją więc znaki głosowe wrażeń
% 8	Traktatu O zdaniu.
powstających w duszy. I dalej w tymże rozdziale potwierdza, że dźwięki
mowy są znakami pojęć, a pismo jest znakami dźwięków; z tych dźwięków
składa się mowa ludzka. Stąd mowa nie jest niczym innym, jak tylko
narzędziem głosowym, przy pomocy którego pojęcia naszego umysłu dajemy
poznać innym.

Mowa ludzka jest zróżnicowana, a wielość i zróżnicowanie
języków wywodzi się z rozdzielenia języka synów Noego budujących
wieżę, jak to czytamy w Piśmie św.: Zejdźmy i pomieszajmy język synów
Noego, aby żaden nie rozumiał głosu swego najbliższego. I tak
rozproszył ich (Bóg) z owego miejsca po całej ziemi. I przestali
budować miasto. A owemu miejscu nadano nazwę Babel, dlatego że tam
\newtip{9}{\textit{Księga Rodzaju} XI, 7-9.}
został pomieszany język całej ziemi¹. Z powodu tego pomieszania
% 9	Księga Rodzaju XI, 7-9.
języka, tj. mowy, różne narody mają różne postacie i litery, czyli
znaki przedstawiające w piśmie ich języki zależnie od ich wynalazców,
\newtip{10}{Tj. Aleksander de Villa Dei, ur. ok. 1150 r. w Villedieu w
  Normandii, w napisanym heksametrem podręczniku gramatyki
  pt. \textit{Doctrinale puerorum in unum digestum}.}
jak to mówi \textit{Metrysta}¹:
% 10 Tj. Aleksander de Villa Dei, ur. ok. 1150 r. w Villedieu w
% Normandii, w napisanym heksametrem podręczniku gramatyki
% pt. Doctrinale pue- rorum in unum digestum.

\begin{quote}
  Abraham patriarcha wynalazł hebrajskie litery,
\newtip{11}{Kadmus (w tekście: Catius) — w mitologii greckiej syn Agenora i
Telefany, który miał przenieść z Egiptu czy Fenicji do Grecji alfabet.}

Kadmus¹ zaś greckie, a łacińskie wynalazła 
% 11 Kadmus (w tekście: Catius) — w mitologii greckiej syn Agenora i
% Telefany, który miał przenieść z Egiptu czy Fenicji do Grecji alfabet.
\newtip{12}{Karmentis lub Karmenta — w mitologii rzymskiej nimfa i
  wieszczka. Wynalazek, o którym tu mowa, przypisywano nie jej, ale
  jej synowi Ewandrowi.} 
Karmentis¹.
% 12	Karmentis lub Karmenta — w mitologii rzymskiej nimfa i wieszczka.
% Wynalazek, o którym tu mowa, przypisywano nie
% jej, ale jej synowi Ewandrowi.
\end{quote}
Łącząc te litery w wyrazy, przy ich pomocy swoje myśli i zamiary
pokazują współczesnym, a potomnym pozostawiają
\marginpar{s. 86}
do czytania na piśmie. A chociaż różne narody mają różne języki i
różne znaki wyrażające je w piśmie, jak to widać u Jana de
\newtip{13}{Właściwie John de Maundeville, 1300-1372, którego opis podróży do
Palestyny znany był w Polsce jako dzieło Jana de Montalma lub
Montalmo. W rękopisie Biblioteki Jagiellońskiej nr 2392, s. 1-37,
znajduje się \textit{Johannis de Montalma Peregrinatio ad Terram Sanctam anno
1322}.}
Montalino¹, 
% 13 Właściwie John de Maundeville, 1300-1372, którego opis podróży do
% Palestyny znany był w Polsce jako dzieło Jana de Montalma lub
% Montalmo. W rękopisie Biblioteki Jagiellońskiej nr 2392, s. 1-37,
% znajduje się JoJiannis de Montalma Peregrinatio ad Terram Sanctam anno
% 1322.
\newtip{14}{Tj. Trojan; chodzi zapewne o pismo arabskie używane
  m.in. przez Turków, których identyfikowano z Teukrami.}
który abecadła różnych narodów: Teukrów¹,
% 14	Tj. Trojan; chodzi zapewne o pismo arabskie używane m.in. przez Turków, których identyfikowano z Teukrami.—86 —
Chaldejczyków i innych, spisał w ich własnych kształtach, to jednak
wszystkie abecadła, w tym również trzy bardziej rozpowszechnione,
mianowicie hebrajskie, greckie i łacińskie, często się wprawdzie
różnią swoimi brzmieniami — inaczej je (tj. litery) bowiem piszą i
inaczej nazywają, gdyż niektóre narody nazywają i piszą swoje (litery)
przy pomocy wyrazów, inne przy pomocy sylab, np. Grecy: \textit{alfa},
\textit{beta} itd., Rusini: \textit{az}, \textit{buki}, \textit{we},
\textit{de}, \textit{hlahol} itd., natomiast Łacinnicy wyrażają litery
najprościej, bo przez same głoski, np. \textit{a}, \textit{b},
\textit{c}, \textit{d}, itd. — wszystkie jednak narody zgadzają się w
wyrażaniu głównego elementu (dźwiękowego): \textit{alfa} u Greków i
\textit{az} u Rusinów jest to przecież to samo, co \textit{a} u
Łacinników.

Chociaż więc liczne narody mają różne kształty liter wyrażających ich
mowę, to jednak niektóre zgadzają się w używaniu znaków i kształtów
jednego i tego samego wzoru, np. Włosi, Francuzi, Anglicy, Czesi,
Niemcy i inni: wszyscy oni przejęli sposoby (pisma) i litery z języka
łacińskiego, dodawszy do niektórych z nich dla odróżnienia kropki, aby
ich mogli należycie używać do pisania we własnym języku.

Dzięki temu narody te, a zwłaszcza sąsiadujące z nami,
Polakami, mianowicie Czesi i Niemcy, piszą wszystkie akty państwowe,
przywileje i inne ważne rzeczy we własnym języku przy pomocy liter
łacińskich, dodawszy niektóre rozróżnienia. W ten sposób wszystkie
akty prawne i umowy, które między sobą zawarli, a które trzeba
zachować na piśmie, każdy, kogo to dotyczy, może czytać w brzmieniu
autentycznym i 
\marginpar{s. 87}
dosłownym w tym języku, w którym je zawierano, i prawdy dokonanej rzeczy nie
zniekształca przekład z różniących się od siebie języków, np. z łaciny
na taki lub inny język — a takie zniekształcenia nieraz się zdarzały —
lecz można ją prosto i zrozumiale odczytać w tym samym języku, w
którym akt został dokonany.

Trzeba jeszcze zwrócić uwagę, że wspomniane zniekształcenia w
przekładach mogą pochodzić nie tylko z nieudolności tłumaczy. Niekiedy
wynikają one z konieczności narzuconej przez rozmaitość zdań czy
zwrotów w różnych językach, wyrażających jedną i tę samą myśl za
pomocą różnych form i różnych ich układów, zgodnie ze zwykłymi
właściwościami każdego języka. Wiadomo bowiem, że jakieś zdanie czy
wyrażenie w jednym języku może brzmieć prawdziwie, stosownie do
zwyczaju tego właśnie języka, a stanie się od razu fałszywe, jeśli się
je przełoży słowo po słowie.

Bo choć Łacinnik powie poprawnie np.: \textit{Cervisia defecatur} albo
\textit{purgatur}, i podobnie tłumacz polski poprawnie powie:
\textit{Piwo się ustawa} albo: \textit{Piwo sie czyści}, to jednak w
% sie ????
podobnym (innym) zwrocie taki przekład nie może mieć zastosowania. Bo
wprawdzie Łacinnik, wyrażając swoją myśl właściwie, mówi 
\newtip{15}{'Chleb jest jedzony'.}
\textit{Panis comeditur}¹, i tłumacz po polsku mówi: \textit{Chleb się je}, jest to jednak,
% 15	'Chleb jest jedzony*.
\newtip{16}{Chodzi prawdopodobnie o to, że przekład \textit{panis
  comeditur} jako \textit{chleb się je} jest zły z punktu widzenia przede
  wszystkim logicznego: w zwrocie łacińskim \textit{panis} jest podmiotem, w
  zwrocie polskim \textit{chleb} — dopełnieniem (nie je sam siebie, jak piwo
  czyści się samo).}
jak widać, przekład fałszywy¹. Tak samo jest i z innymi
% 16 Cliodzi prawdopodobnie o to, że przekład panis eomeditur jako
% clileb się, je jest zły z punktu widzenia przede wszystkim logicznego:
% w zwrocie łacińskim panis jest podmiotem, w zwrocie polskim chleb —
% dopełnieniem (nie je sam siebie, jak piwo czyści się samo).
zwrotami.

Podobnie w języku niemieckim (łacińskie) \newtip{17}{'Przerzuć przez dom'.}
\textit{Proicias per
  domum}¹ można przełożyć jako: \textit{Werf heber Haus}, przy czym
% 17	'Przerzuć przez dom'.—87 —
przyimek \textit{heber} oznacza 'na zewnątrz’, czyli 'ponad'
domem. Lecz gdy (tłumacz łacińskie \textit{per}) pojmie jako znaczące
'przez wnętrze’ domu, wówczas użyje innego przyimka i powie:
\textit{Wer durch Haus}, rozumiejąc w ten sposób, że jakiś dom ma
dwoje
\marginpar{s. 88}
drzwi naprzeciw siebie otwartych na przestrzał, i myśląc o nich chce
powiedzieć, że trzeba coś przerzucić. W języku łacińskim natomiast te
dwa różne zwroty (niemieckie) wyraża się tylko jednym sposobem,
mówiąc: \textit{Proicias per domum}, i nie czyniąc różnicy między
\newtip{18}{Tj. między jednym a drugim znaczeniem.}
jednym a drugim sposobem wyrażania¹
% 18	Tj. między jednym a drugim znaczeniem.
A choć dzięki sprawności rozumu możemy jeden i ten sam zwrot tłumaczyć
sobie w takim lub innym znaczeniu i w ten sposób od wprowadzającej w
błąd ekwiwokacji dochodzić do jego autentycznego i zasadniczego sensu
\newtip{19}{Ojciec i doktor Kościoła, ok. 347-420, znany głównie
  jako tłumacz Biblii na j. łaciński (tzw. Wulgata).}
— pisze bowiem św. Hieronim¹
% 19	Ojciec i doktor Kościoła, ok. 347-420, znany głównie jako tłumacz Biblii na j. łaciński (tzw. Wulgata).
\newtip{20}{W tekście błędnie: ad Pachomium — do Pachomiusza. Pachomiusz,
organizator życia klasztornego w Kościele, zmarł rok przed urodzeniem
się Hieronima. Hieronim tłumaczył jego regułę zakonną, stąd
prawdopodobnie wynikła pomyłka autora czy raczej kopisty tekstu.}
w liście do Pammachiusza¹ o najlepszym
% 20 W tekście błędnie: ad Pachomium — do Pachomiusza. Pacho- miusz,
% organizator życia klasztornego w Kościele, zmarł rok przed urodzeniem
% się Hieronima. Hieronim tłumaczył jego regułę zakonną, stąd
% prawdopodobnie wynikła pomyłka autora czy raczej kopisty tekstu.
sposobie tłumaczenia, powołując się na różne świadectwa i przykłady
\newtip{21}{Tj. Cycerona. Marcus Tullius Cicero, mąż stanu, mówca i filozof rzymski z I w. p.n.e.}
różnych tłumaczy i autorów, jak Tulliusza¹,
% 21	Tj. Cycerona. Marcus Tullius Cicero, mąż stanu, mówca i filozof rzymski z I w. p.n.e.
\newtip{22}{Publius Terentius Afer, komediopisarz rzymski z II w. p.n.e.}
Terencjusza¹,
% 22	Publius Terentius Afer, komediopisarz rzymski z II w. p.n.e.
\newtip{23}{Św. Hilary z Poitiers, IV w.}
Hilarego¹,
% 23	Św. Hilary z Poitiers, IV w.—88 —
które pokazują, że każdy znający się na rzeczy tłumacz powinien się
pilnie przykładać do tego, by tłumaczyć nie słowo w słowo, lecz żeby
w tłumaczeniu znaczenie właściwie oddawać; a i Horacy mówi: Chcąc być
wiernym tłumaczem, nie będziesz się starał tłumaczyć słowo w słowo,
lecz sens oddawać — to jednak wiadomo, że nie dla wszystkich jest to
równie łatwe. Albowiem, jak tenże św. Hieronim mówi w tym samym
miejscu, trudna to rzecz i mozolna, idąc cudzym szlakiem, tj. śladem
cudzego pisania, nie wypaść gdzieś z koleiny, aby to, co dobrze
wyrażono w obcym języku, zachowało tę samą ozdobność w
przekładzie. Przecież każdy język ma różne właściwości swoich wyrazów,
a zatem itd.
 
Bardzo więc przezornie niektóre narody spisują swoje
czyny, postępki i dokonania we własnym języku, ażeby się (tu) nie
\marginpar{s. 89}
wkradł jakiś błąd z obcego języka. Taka przezorność jest nieznana
jednemu tylko przesławnemu językowi polskiemu.

\newtip{24}{Tj. prawa kanonicznego.}
Widząc to znakomity Jakub syn Parkosza z Żórawic, doktor dekretów¹,
% 24	Tj. prawa kanonicznego.
\newtip{25}{W Krakowie.}
kanonik krakowski i rektor kościoła parafialnego na Skałce¹, w tej
% 25	W Krakowie.{
myśli, by naród polski w tych rzeczach nie pozostawał nadal — z
wielkim dla siebie niebezpieczeństwem — w tyle za innymi, wynalazł i
przedstawił w niniejszej rozprawce jeden całkowicie wystarczający
sposób, dzięki któremu można będzie zadowalająco pisać po polsku,
ażeby naród pocieszony mógł spisywać swoje dzieje we własnym
języku.

Wprawdzie wielu błędnie myślącym, a właściwiej trzeba
powiedzieć: fantastycznie i chimerycznie bredzącym, wydaje się, że
używanie środka zaleconego w tej rozprawce, zachęcającej, by panowie
Polacy, tak jak inne narody, myśli swoje wyrażali we własnym języku,
może się stać okazją do błędów. Twierdzą oni mianowicie, że okazja
taka przydarzyła się pewnym ludziom, a przyczynę błędów upatrują nie w
czym innym, lecz w tym właśnie, że pośród wielkiej ilości ksiąg
teologicznych i innych istnieją też księgi napisane w ich własnym
języku. Stąd to — według tak twierdzących — zdarza się, że nie tylko
mężczyźni, ale i kobiety, czytając takie księgi, wpadają w
błędy.

Trzymajmy się jednak z daleka od takiego paralogistycznego mniemania,
aby odrobina kwasu nie zepsuła (zdrowej) intencji, z jaką sporządzono
niniejsze ciasto. Bo nie rzecz sama ani nie narzędzie zdolne jest
otrzymać pochwałę lub naganę, ale tylko jego właściwe używanie lub
nadużywanie. Jeśliś upadł po nadużyciu wina, nie jest to upadek wina,
ale twój. Gdyby tak nie było, toby się trzeba wyrzec i Pisma św.,
ponieważ jego powagą także i heretycy popierają swoje mniemania,
dowodząc każdego artykułu za pomocą Pisma św., jak się okazuje z
rozprawy czcigodnego Benedykta, doktora dekretów, opata z
\newtip{26}{Autora m. in. rozprawy \textit{Tractatus contra diversos
    errores hereticorum} (znajdującego się m.in. w rękopisie
  Biblioteki Jagiellońskiej nr 423, k. 88-111).}
Marsylii¹. Ale nie dlatego zasługują na naganę, że się powołują na
% 26	Autora m. in. rozprawy Tractatus contra diversos errores hereticorum—89 —
% (znajdującego się m.in. w rękopisie Biblioteki
% Jagiellońskiej nr 423, k. 88-111).
Pismo św., lecz dlatego, że je opacznie rozumieją. Dobre lub złe 
\marginpar{s. 90}
używanie narzędzia zależy od zdolności używającego, gdyż — według
\newtip{27}{Prawdopodobnie Inocentego III albo Inocentego V (por. następną notkę).}
Inocentego¹ — narzędzie nie działa samo przez się, ale w połączeniu
% 27	Prawdopodobnie Inocentego III albo Inocentego Y (por. następną notkę).
z tym, kto je wprawia w ruch. Znaczy to, że nie działa ono swobodnie,
lecz tylko z woli osoby narzędzie w ruch wprowadzającej. Bo — jak mówi
\newtip{28}{Późniejszy papież Inocenty V, 1225-1276. Główne dzieła:
  \textit{Commentaria in 4 libros sententiarum, Commentarii in epistolas
  s. Pauli} (m.in. w kilku rękopisach w Bibliotece Jagiellońskiej).}
Piotr z Tarantasio¹ — to wola właśnie zamyka oczy i otwiera je, jak
% 28 Późniejszy papież Inocenty V, 1225-1276. Główne dzieła: Com-
% mentaria in 4 libros senłentiarum, Commentarii in epistolas s. Pauli
% (m.in. w kilku' rękopisach w Bibliotece Jagiellońskiej).
chce.

Niech więc ustąpi bezsensowny zarzut i niech się zacznie
pożyteczna służba dla języka polskiego. Itd.


\newtip{29}{Jedne z ksiąg Pisma św.}
Święty Hieronim pisze w przedmowie do \textit{Ksiąg Królewskich}¹, że
% 29	Jedne z ksiąg Pisma św.
Hebrajczycy, Syryjczycy i Chaldejczycy mają dwadzieścia dwie
litery. Łacinnicy zaś mają o jedną literę więcej oraz przytłumione,
chrapliwe \textit{k} i \textit{q}. Łacińskie litery różnią się
wprawdzie kształtami, ale nie wszystkie różnią się brzmieniem. Czasem
wprawdzie znak literowy ma różne brzmienia, zarazem jednak niektóre
litery są zbędne. Nie potrzebuje bowiem język łaciński litery
\textit{k}, a \textit{h} jest także tylko znakiem przydechu.

Nasz zaś słowiański, a zwłaszcza polski język potrzebuje o wiele
więcej liter. Przede wszystkim bowiem gdy Łacinnicy zadowalają się
\newtip{30}{Transkrybowaną tutaj przez ą.}
pięcioma samogłoskami, to Polacy dorzucają szóstą: \textit{ø}¹, bez
% 30	Transkrybowaną tutaj przez ą.
której w języku polskim nie można pisać. Co prawda, w pewnych
wypadkach można zamiast \textit{ø} pisać \textit{an},
np. \textit{ranka}, \textit{manka} 'ręka, męka', w innych wypadkach
jednak w żaden sposób nie można tak pisać, np. w \textit{mą̄ka},
\textit{drąga}. W przeciwnym razie między \textit{męką} a
\textit{mąką} i wielu innymi nie będzie \newtip{31}{31 Nie można pisać
  \textit{manka}, \textit{dranga}, bo można by to czytać zarówno \textit{mąka},
  \textit{drąga}, jak i \textit{męka}, \textit{dręga}.}  różnicy¹.
% 31	Nie można pisaó manka, dranga, bo można by to czytać zarówno mąka, drąga, jak i męka, dręga.-90-


\marginpar{s. 91}

U Polaków wszystkie samogłoski to się wzdłużają, to się wyraźnie
skracają. Ze wzdłużania ich lub skracania wynika różne znaczenie
wyrazów. Przykład na \textit{a}: \textit{wiercimāk}. Jeśli się tu
\textit{a} przedłuży, to będzie jeden wyraz, a jeśli się skróci — dwa
\newtip{32}{Tj. \textit{wiercimāk} — \textit{wierci} \textit{mak}. Objaśnienia znaczeń tych i
  innych wyrazów podano w indeksie. Krótkość samogłosek oznaczamy
  (łuczkami) tylko w niektórych przykładach.}
wyrazy¹. 
% 32	Tj. wiercimak — wierci małe. Objaśnienia znaczeń tych i innych wyrazów podano w indeksie. Krótkość samogłosek oznaczamy (łuczkami) tylko w niektórych przykładach.
\newtip{33}{Że rzeczownik \textit{biel} miał \textit{e} długie, a
  rozkaźnik \textit{biel}! \textit{e} krótkie, to potwierdza
  \textit{Słownik polszczyzny XVI wieku}, w którym rz. \textit{biel}
  ma w mianowniku i bierniku \textit{e} pochylone (\textit{é}),
  czasownik \textit{bielić} \textit{e} jasne (formy imperatiwu brak),
  imperativus \textit{dziel}! \textit{e} jasne.}  
Przykład na \textit{e}: \textit{biēl} — \textit{biĕl}!¹
% 33 Że rzeczownik biel miał e długi3, a rozkaźnik bieli e krótkie, to
% potwierdza Słownik polszczyzny XVI wieku, w którym rz. biel ma w
% mianowniku i bierniku e pochylone (e), czasownik bielić e jasne (formy
% im- peratiwu brak), imperatiyus dzieli e jasne.
\newtip{34}{Może należy czytać \textit{bił}.}
Przykład na \textit{y} (\textit{i}): \textit{bȳł}¹ —
% 34	Może należy czytać bił.
\newtip{35}{Oba te przykłady niejasne. Drugi ma mieć samogłoskę krótką,
tymczasem w rzeczownikach przed dźwięcznym \textit{ł} występuje z reguły
samogłoska pochylona (dawna długa).}
\textit{by̆ł}¹. Przykład na \textit{o}, gdzie się wzdłuża, a gdzie
% 35 Oba te przykłady niejasne. Drugi ma mieć samogłoskę krótką,
% tymczasem w rzeczownikach przed dźwięcznym ł występuje z reguły
% samogłoska pochylona (dawna długa).
skraca: \textit{kōt} (\textit{kót}) — \textit{kŏt}. Na \textit{ą}:
\textit{drą̆ga} — \textit{drą̄ga} (\textit{dręga} —
\textit{drąga}). Przykład na \textit{u}: \textit{drŭga} —
\textit{drūg}.

Łacinnicy w zakresie długości i krótkości samogłosek w piśmie nie
stosują żadnej różnicy albo tylko niewielką, ponieważ co do iloczasu,
tj. wzdłużania i skracania samogłosek, mają wystarczające reguły w
\newtip{36}{Priscianus z Cezarei (Mauretania), V/VI w., autor gramatyki
łacińskiej pt. \textit{Institutiones grammaticae}.}
pismach gramatyków, mianowicie Prisciana¹, 
% 86 Priscianus z Cezarei (Mauretania), Y/YI w., autor gramatyki
% łacińskiej pt. Institutiones grammaticae.
\newtip{37}{Eberhardus z Bethume, XIII w., autor wierszowanej gramatyki
pt. \textit{Graecismus}.}
Eberharda¹,
\newtip{38}{38 Aleksander de Yilla Dei, por. notkę nr 10.}
Aleksandra¹ i innych. Przypuszczają więc, że czytelnicy łacińscy
% 38 Aleksander de Yilla Dei, por. notkę nr 10.
znają te reguły, i dlatego samogłoski krótkie i przedłużone piszą tym
samym sposobem, czyli tym samym kształtem, mając na względzie krótkość
pisma. A ponieważ Polakom nie wykształconym (w zakresie gramatyki)
trudno by było wyłożyć reguły prozodii, należy więc (w języku polskim)
zaznaczać iloczas samogłosek w piśmie. To zaś nie jest łatwo uczynić w
inny sposób, jak tylko podwajając samogłoskę długą, a pojedynczo
pisząc krótką. Tak jak w \textit{Adaam}, gdzie pierwsze \textit{a}
jest krótkie, drugie długie i podwojone.  
\marginpar{s. 92} Gdybyśmy chcieli różnice w wymawianiu długich i krótkich samogłosek
wyrażać za pomocą kształtu liter, nie byłoby to może zbyt
trudne. Grecy bowiem zachowują taki sposób pisania, inaczej pisząc o
\newtip{39}{39	Tj. \textit{Ω} (omega) — \textit{ο} (omikron).}
długie, inaczej o krótkie¹. My jednak, którzy wszystko (w tym
% 39	Tj. Q (omega) — o (omikron).
względzie) zapożyczyliśmy od Łacinników, zdecydowaliśmy się poniechać
tej niezwykłości, bo mogłaby być niektórym niemiła.

Uważamy jednak za konieczne iloczas samogłosek wyrażać w piśmie przez
pisanie ich podwójnie lub pojedynczo, ponieważ, jak już wspomniano,
pominięcie tego nastręczałoby znaczne trudności w odróżnianiu
znaczeń wyrazów. Dlatego, choć nie każda długa samogłoska będzie (w
piśmie) podwajana, będzie się podwajania przestrzegać przynajmniej
tam, gdzie z jej skrócenia i wzdłużenia wynika wyraźna różnica
znaczenia tego samego wyrazu. Chociaż zresztą i w zakresie spółgłosek
tego rodzaju różnicę można również zaznaczyć, o czym będzie mowa
niżej.

A więc niemal wszystkie litery — z wyjątkiem \textit{h}, \textit{k},
\textit{q}, \textit{r}, \textit{t} — różnią się w brzmieniu. Bowiem
\textit{c} zmienia brzmienie pięciokrotnie, a \textit{s}
sześciokrotnie. Zaś \textit{v} (\textit{u}) oznaczające samogłoskę
jest — podobnie jak inne samogłoski — bądź długie, bądź krótkie, a
oznaczające spółgłoskę także parokrotnie zmienia brzmienie. Bo kiedy
się łączy spółgłoska i samogłoska, wtedy właściwą pisownię ma taką jak
\textit{wſta}, \textit{wmaar} (\textit{w usta}, \textit{w
  umiār}). Używa się także (litery \textit{v}) na oznaczenie samej
spółgłoski, a wówczas (ta spółgłoska) czasem twardnieje, czasem jest
wymawiana miękko. Przykład pierwszego (typu wymawiania):
\textit{wykład}, \textit{wyje}, \textit{wyja}. Przykład drugiego:
\textit{wiła} \textit{wiłuje}. A kiedy oznacza samą samogłoskę
\textit{u}, pisze się tak samo jak \textit{u}, w górnej części
otwarte, w dolnej połączone. Kiedy zaś się wzdłuża, wówczas jest (w
piśmie) podwajane. Przykład pierwszego (jest) w \textit{uÿe}
\newtip{40}{Oba przykłady niejasne. Pierwszy to prawdopodobnie forma 3 osoby od
czasownika \textit{ujeść}, co do drugiego to Łoś dopuszcza dwie możliwości:
zniekształcone \textit{ujazd} lub \textit{ujas} `wuj (zgrubiałe)'.}
\textit{uÿass} (\textit{uje}, \textit{ujas})¹. Przykład drugiego: \textit{ſzvwam}
\textit{dmuchaa} (\textit{suwam}, \textit{dmuchā}).
\marginpar{s. 93}
Kiedy jest spółgłoską i twardnieje, wówczas (w pisowni) z wyższego
pierwszego rożka wyprowadza się przeciągnięcie (linii),
np. \textit{ʋaal}, \textit{ʋyl}, \textit{ʋikład} (\textit{wāł},
\textit{wył}, \textit{wykład}). Kiedy się zaś zmiękcza, wtedy się
pisze z równymi rożkami, tak jak: \textit{viła}, \textit{vidzaall},
\textit{vino} (\textit{wiła}, \textit{widziāł}, \textit{wino}).

Gdy ta różnica między miękką i twardą spółgłoską \textit{v} będzie
zachowywana, nie będzie konieczne pisanie podwójnego \textit{y} (dla
oznaczania miękkości), jak się niegdyś pisało, np. \textit{vyatr}
\textit{wyege}, \textit{wyoſna}, \textit{vyøonczek} (\textit{wiatr
  wieje}, \textit{wiosna}, \textit{Wiącek}). (Nie należy tak pisać),
ponieważ nie byłoby różnicy między \textit{wijał} i \textit{wiał}
'dął', ale miękkie \textit{w} pisze się tak: \textit{vaal},
\textit{vathr}, \textit{vege}, \textit{voſna}, \textit{vøczek},
\textit{vvaaƚ} (\textit{wiāł}, \textit{wiatr}, \textit{wieje},
\textit{wiosna}, \textit{Wiącek}, \textit{wwiāł} czy
\textit{uwiāł}). Błędem jest bowiem czynić przy pomocy większej ilości
rzeczy to, co równie dobrze może być uczynione przy pomocy mniejszej
ilości.

A te oto sześć liter:
\textit{b},\textit{f},\textit{l},\textit{m},\textit{n},\textit{p},
\newtip{41}{To „iloczas” tu zbędne.}
zachowując to samo (w zasadzie) brzmienie i iloczas¹, są wymawiane
bądź twardo, bądź miękko.

\newtip{42}{A raczej drugiego, tj. miękkiego wymawiania.}
O \textit{b}. Przykład pierwszego (typu wymawiania)¹ \textit{b}:
\newtip{43}{Tj. twardego wymawiania.}
\textit{bik}, \textit{bit} 'bity'. Przykład drugiego¹: \textit{był}
'był’, \textit{byk} 'byk'. Na podstawie samogłoski, podwojonej i
wzdłużonej, nie może tu być (widocznej) różnicy. Okazuje się bowiem,
że zarówno \textit{b} wymawiane miękko jak i twardo łączy się tak z
długim \textit{y} jak z krótkim \textit{i}. Bo chociaż tam, gdzie się
\textit{i} (\textit{y}) wzdłuża, \textit{b} powinno twardnieć, a gdzie
się \textit{i} (\textit{y}) skraca, \textit{b} winno by się zmiękczać,
jak w przykładach wyżej podanych \textit{byk} i \textit{bik}, to
jednak spotyka się wyrazy, w których \textit{i} się skraca, a
\textit{b} raz się zmiękcza, raz twardnieje, jak w \textit{by̆t}
'mieszkanie’, \textit{bĭt} 'bity’; w tych wyrazach \textit{i}
(\textit{y}) się skraca, a jednak \textit{b} się i zmiękcza, i
twardnieje. Czasem na odwrót: \textit{i} (\textit{y}) się wzdłuża, a
jednak \textit{b} się wymawia i twardo, i miękko, jak w \textit{bīł}
'bił’, \textit{bȳł} 'był’.

F. Również \textit{f} jest wymawiane bądź twardo, bądź miękko,
\marginpar{s. 94}
\newtip{44}{Tj. \textit{Chwał} (i dalej: \textit{chwiał}, \textit{chwist}, \textit{chwyta}).}
np. \textit{Fāł}¹ 'imię własne’ — \textit{fiāł} 'chwiał,
poruszał'. Tych wyrazów też nie można różnicować przez dodane
samogłoski, bo oba: \textit{Fāł} i \textit{fiāł} mają \textit{ā}
długie, czyli (w pisowni) podwojone, a f się w brzmieniu różni. I
gdyby się twierdziło, że samogłoska wzdłużona, czyli przedłużona
powoduje twardość, a skrócona miękkość spółgłoski, jak to powiedziano
o \textit{b}, toby się to okazało nieprawdziwe w przykładach dopiero
co przytoczonych: \textit{Fāł} — \textit{fiāł}. Niekiedy nawet
samogłoska się wzdłuża, a \textit{f} się zmiękcza, np.\textit{fīst}; i
odwrotnie, samogłoska się skraca, a \textit{f} twardnieje,
np. \textit{fy̆tā}; i znowu \textit{i} się skraca, np. \textit{fīst}
\textit{fy̆ta} — \textit{fĭgi}.

\textit{L}, \textit{ł}. Tak samo i \textit{l} bądź twardnieje, bądź
się zmiękcza, np. \textit{list} 'list’ lub 'liść’ — \textit{łyst}
'część nogi’; \textit{lis} 'lis’ — \textit{łys} 'łysy’. Mimo że
wszystko inne jest tu jednakowe, \textit{l} raz twardnieje, raz się
zmiękcza. Tak również na końcu (wyrazów), np. \textit{stāl} 'stal’ —
\textit{stāł} 'stał’.


\textit{M}. Podobnie twardnieje i zmiękcza się \textit{m} znajdujące
się koło tych samych samogłosek i spółgłosek. Przykład: \textit{Mika}
'Mikołaj’ — \textit{mykā} 'pociąga’; również położone na końcu wyrazu,
np. \textit{dym}, \textit{grom}, \textit{jim} 'im’ — \textit{jiḿ}
'chwytaj’, \textit{przyḿ} 'przyjmij’.


\textit{N}. Podobnie twardnieje i zmiękcza się \textit{n}. Przykład na
twarde: \textit{nyski} 'z miasta Nysy’. Przykład na miękkie:
\textit{niski} 'niewysoko położony’. Tak również na końcu:
\textit{syn} — \textit{kōń}.

\textit{P}. Podobnie twardo i miękko wymawia się \textit{p}:
\textit{Pan} \textit{Paweł} \textit{pyszno} \textit{pije}
\textit{piwo}, \textit{Piotr} \textit{powiē} \textit{popu}.


Jak więc zaznaczymy różnice brzmień tych liter?  Jeśli zechcemy jakoś
te różnice zaznaczyć, napotkamy niemałe trudności. Jeśli nie będziemy
zaznaczać, musi to pociągnąć za sobą wiele błędów. Pisanie bowiem tym
\newtip{45}{Tj. głoski.}
samym znakiem litery¹ miękko i twardo wymawianej jest niemałym
błędem. A jak już powiedziano, różnica nie może być zaznaczona przez
\newtip{46}{Tzn. przez zróżnicowanie pisowni samogłosek.}
samogłoski długie, czyli dodane¹.

\marginpar{s. 95}
\textit{B}. Więc, jeśli łaska, (piszmy) \textit{b} twarde bez łuczka
(zagięcia u góry) i kwadratowe, tak jak \textit{b}, które muzycy także
nazywają \textit{b} twardym, a \textit{b} miękkie z łuczkiem u góry i
niżej okrągłe, tak jak \textit{b}, które muzycy także nazywają
\newtip{47}{Chodzi o znakowanie muzyczne dla tonacji moll i dur. Por. Riemann
Musiklexikon herausgegeben von H. H. Eggebrecht, Bd. III Sachteil,
Mainz 1967, s. 71.}
\textit{b} miękkim¹. Przykład (na \textit{b} twarde): \textit{ƀaƀø}
\textit{bik} \textit{ƀodze} (\textit{babą} 'babę’ \textit{byk}
\textit{bodzie}). Przykład drugiego \textit{b}, miękkiego:
\textit{Beneck}, \textit{ɓyka}, \textit{ɓigę} (\textit{Bieniek},
\textit{bika}, \textit{bije}).

\textit{F}. F twarde piszmy jako podwójne, czyli podwojone, \textit{f}
miękkie jako pojedyncze. Przykład pierwszego: \textit{ffaaɬ}
\textit{ffaſth} \textit{ffita} (\textit{Fāł} \textit{fast}
\textit{fyta}). Przykład drugiego: \textit{fisth} \textit{figi}
(\textit{fist} \textit{figi}).

\textit{L}. Tak samo (jak \textit{b}) \textit{l} twarde niech będzie
bez łuczka. Przykład: \textit{ƚapka} \textit{ƚekce} \textit{ƚiſzego}
(\textit{łapka} \textit{łekce} \textit{łysego}), \textit{ƚoſze}
\textit{ƚudzy} \textit{ƚøthka} (\textit{łosie} \textit{łudzi}
\textit{łątkaq}). Miękkie (\textit{l}) niech będzie z łuczkiem u góry,
Przykład: \textit{ɬaaſz}, \textit{ɬis}, \textit{ɬoſth}
a. \textit{ɬoſch}, \textit{ɬeſćh}, \textit{ɬudze}, \textit{ɬąnkawką}
(las, lis, lost, lesz, ludzie, łąkawka).

\textit{M}, \textit{ḿ}. Twarde \textit{m} (niech będzie) z ogonkiem
\newtip{48}{W samym traktacie końcowe m i n jest z reguły pisane z przedłużaniem
ostatniej laski ku dołowi}
przy trzeciej nóżce, jak się zwykło pisać na końcu wyrazów¹, jak
np. \textit{ɱara}, \textit{ɱego}, \textit{ɱige}, \textit{ɱoge},
\textit{ɱødro}, \textit{ɱuſchø} (\textit{mara}, \textit{mego},
\textit{myje}, \textit{moje}, \textit{mądro}, \textit{muszą}
'muszę’). \textit{M} zaś miękkie — bez ogonka, np. \textit{maal}
\textit{mecz} \textit{mikołai} \textit{mood}, (\textit{miāł}
\textit{mieć} \textit{Mikołaj} \textit{miōd}).

\textit{N}, \textit{ń}. Tak również \textit{n} twarde (należy pisać) z
ogonkiem, jak się zwykło pisać na końcu wyrazów, np. \textit{ɲapaſcz},
\textit{gɲath}, \textit{ɲaƚøcz}, \textit{ɲoc}, \textit{ɲos},
\textit{ɲødza} (\textit{napaść}, \textit{gnat}, \textit{nałącz}
'nałęcz', \textit{noc}, \textit{nos}, \textit{nądza} 'nędza’). Miękkie
(\textit{n}) — bez ogonka, np. \textit{nevaſta} \textit{ne}
\textit{ve} \textit{niczs} (\textit{niewiasta nie wie nics}). Między ɲ
twardym a y podwójnym będzie różnica, ponieważ ogonek ɲ pociąga się
na prawo, ogonek \textit{y} zgina się w lewo. Ponadto nad \textit{y} pisze się podwójną
kropkę dla odróżnienia od \textit{ɲ} twardego.

\textit{P}, \textit{p}. \textit{P} twarde niech będzie kwadratowe, tak
jak \textit{b} twarde. \textit{P} miękkie niech będzie
zaokrąglone. Przykład pierwszego: \textit{Ᵽan}
\marginpar{s. 96}
\newtip{49}{Forma nie całkiem jasna, może być od \textit{piać} lub \textit{pojeść}. Chyba błędnie
oznaczono \textit{o} jako długie.}
\textit{Ᵽaveƚl Ᵽooge Ᵽyſchno} (\textit{pan Paweł poje¹
  pyszno}). Przykład drugiego: \textit{Ƥotr Ƥivo Ƥige
  Ƥilno}. (\textit{Piotr piwo pije pilno}), \textit{Ƥech Ƥecze Ƥeceną}
(\textit{Piech piecze pieczenią}).

Nie trzeba też już będzie po tych literach wymawianych miękko pisać
podwójnego \textit{y}, jak to dotychczas pisaliśmy przy wszystkich
wymienionych spółgłoskach. Piszemy bowiem (dotąd) \textit{byƚ} 'bił'
przez podwójne \textit{y}, chcąc zaznaczyć miękkie \textit{b}; ale
widać wyraźnie, że podwójne \textit{y} pisze się także koło \textit{b}
twardego, np. \textit{byƚ} 'był'.

Jakaż więc będzie między nimi różnica?  Otóż — aby było jasne, że przy
\newtip{50}{Tj. oznaczaniu miękkości przez \textit{y}.}
dotychczasowym sposobie pisania¹ nie może występować różnica między
wyrazami — jeśli napiszemy wspomnianym wyżej sposobem \textit{byaal}
'biały' i \textit{byaal} 'bijał', to w piśmie nie będzie różnicy,
podczas gdy w wymowie i w znaczeniu różnica występuje. Niech się więc
odtąd pisze \textit{ɓaal} 'biały', a \textit{ɓyaal} 'bijał' — w ten
sposób różnica będzie wyraźnie zaznaczona. Tak też między
\textit{bika} oznaczającym narzędzie murarzy, gdzie jest \textit{b}
\newtip{51}{Biernik lp.}
miękkie i \textit{i} się skraca, a \textit{bika} (byka) 'byka'¹ albo
\textit{bika} (byka) 'ryczy', gdzie jest \textit{b} twarde i
\textit{i} długie, znowu by nie było (w pisowni) żadnej
różnicy. Lepiej jest więc różnicę \textit{b} zaznaczyć kwadratem dla
\textit{b} twardego, a zaokrągleniem dla miękkiego.

Tak samo jest przy \textit{l}. Jeśli zechcemy miękkie \textit{l}
oznaczać za pomocą podwójnego \textit{i} (tj. \textit{y}), to w
pewnych wyrazach będzie to odpowiednie, w innych zaś wcale nie,
np. \textit{lyſchka} (\textit{liszka}) 'lis', \textit{lyſſka}
\newtip{52}{Oba te nieco różnie napisane wyrazy brzmiały niewątpliwie jednakowo.}
(\textit{liszka})¹ `gąsienica', \textit{lyſth} (list)
'liść'. Podczas gdy — zgodnie z tym, co zostało powiedziane wyżej —
każda z samogłosek napisana podwójnie winna być wymawiana jako długa,
to we wszystkich dopiero co podanych przykładach (tak napisana
samogłoska) jest w wymowie krótka. Oto pierwszy
\marginpar{s. 97}
przykład niedorzeczności (w dotychczasowej pisowni). A wreszcie, jeśli
się do \textit{l} występującego na końcu wyrazów dodaje podwójne
\textit{y}, powoduje to (również) zamieszanie w znaczeniu. Przykład:
\textit{ſtaal} `stāl'. Gdyby po \textit{l} było napisane podwójne
\textit{y}, powstałoby \textit{ſtaaly} `stāli'. Gdzie (więc) będzie
nasza stal?  Jeśliby zaś napisać bez \textit{y}, to będzie
\textit{ſtal} 'stał'. Taki jest drugi przykład niedorzeczności. Lepiej
więc będzie wyrażona różnica, gdy napiszemy \textit{l} twarde bez
łuczka, jako laskę np. \textit{ſtaaƚ} \textit{ƚyſſy} 'stał łysy', a
\textit{l} miękkie z łuczkiem, a bez dodawania \textit{y},
np. \textit{ſtaɬ} 'stal', \textit{ɬiſth} (\textit{list}) 'liść',
\textit{ɬuud} 'lud'.

To samo się odnosi do \textit{m}. (Wyrazy) \textit{Mikołaj},
\newtip{53}{Czy też \textit{miła}.}
\textit{misa}, \textit{mila}¹, \textit{miāł} 'coś miałkiego' lub
czasownik '(on) miał' pisano niegdyś z podwójnym \textit{y} chcąc
oznaczyć \textit{m} miękkie. Ale oto pokazało się, że za pomocą
\textit{y} podwójnego nie można odróżniać znajdującego się koło
samogłosek i spółgłosek \textit{m} miękkiego od \textit{m} twardego,
np. \textit{mika} 'Mikołaj' — \textit{ɱikaa}, (\textit{mykā})
'pociąga'. Lepiej jest więc wprowadzić zróżnicowanie, pisząc m miękkie
\newtip{54}{Tu zbędne.}
bez ogonka, np. \textit{Mika}, \textit{Mikolay}, \textit{mykaa}¹,
\textit{myasto}, \textit{maal} (\textit{Mika}, \textit{Mikołaj},
\textit{miasto}, \textit{miał}). Wtedy nie będzie już konieczne
pisanie podwójnego \textit{y} między miękkim \textit{m} a następującą
samogłoską, np. \textit{myedz}, \textit{myood} (\textit{miedź},
\textit{miōd}), ale \textit{m} miękkie zajmie miejsce podwójnego
\textit{y}, a (następujące po nim) \textit{a} lub \textit{o}, lub inna
samogłoska pozostanie długa, np. \textit{maal} (\textit{miāł}) 'coś
miałkiego', \textit{meedz} (\textit{miēdź}), \textit{mood}
(\textit{miōd}) 'miód', \textit{mąſſo} (miąso) 'mięso',
\textit{maaſga} (\textit{miazga}).

Podobnie jest ze spółgłoskami \textit{p} i \textit{n}, przy których
popełniano (jeszcze) gorsze błędy. Bo ilekroć się trafiało \textit{n}
i \textit{p} miękkie, zawsze pisano je przy pomocy podwójnego
\textit{y} przed odpowiednią samogłoską, niezależnie od tego, czy się
\newtip{55}{W tekście: sive \textit{i} breviabatur. sive producebatur, więc
dosłownie: czy się \textit{i} skracało, czy też przedłużało. To nie
bardzo ma sens. Dlatego uważamy, że „i” choć bez skrótu oznacza
„illa”, tj. ona (samogłoska).}
ona skracała, czy wzdłużała¹. Przykład pierwszego z \textit{a}:
\textit{gnyaaſdo} (\textit{gniāzdo}), z \textit{e}: \textit{nyewyaſta}
(\textit{niewiasta}), z \textit{i}: \textit{nycz} (\textit{nic}). Taka
pisownia była
\marginpar{s. 98}
niewystarczająca do zróżnicowania, bo między \textit{Nija}, co było
nazwą bożka, a \textit{nia}, sylabą znajdującą się w wyrazie
\textit{gniazdo}, nie było (w pisowni) różnicy. Tak samo jest z
\textit{n} znajdującym się na końcu wyrazu, np. \textit{zgon} —
\textit{koń}. Na końcu owego wyrazu \textit{koń} musiało się dodawać
podwójne \textit{y} (i dla oznaczenia miękkości \textit{n}, i dla
oznaczenia \textit{i} (końcówki)), a przy takiej pisowni nie może być
różnicy między mianownikiem l. pojedynczej \textit{koonÿ}
(\textit{koń}), a dopełniaczem l. mnogiej \textit{koonÿ}
(\textit{koni}). Tak samo (brak różnicy) między \textit{gonÿ}
(\textit{goń}!) w rozkaźniku 'odpędzaj', a \textit{gonÿ}
(\textit{goni}) 'odpędza'.

Tak jest również z \textit{p}. Ilekroć trzeba było napisać \textit{p}
miękkie, zawsze dodawano (po nim) podwójne \textit{y} przed
następującą samogłoską, np. \textit{pyaſek} 'piasek', \textit{pyechna}
(\textit{Piechna}), \textit{pyotr} (\textit{Piotr}), \textit{pyąthno}
(\textit{piątno}) 'piętno'. Lecz ta różnica, czyli jej zaznaczanie,
\newtip{56}{Tzn. ta pisownia nie wystarcza do wyrażenia różnicy wymowy.}
nie wystarcza¹. Bo przy takiej pisowni nie będzie różnicy między
\textit{pija} a \textit{pia}, między \textit{pije} a
\newtip{57}{\textit{Pia} i \textit{pie} są najprawdopodobniej sylabami, nie wyrazami.}
\textit{pie}¹. Ale podobnie jak o \textit{n} powiedziano, że się
zmiękcza, kiedy jest pisane bez ogonka, (niezależnie od tego) czy po
nim następuje jakaś samogłoska, czy nie, np. \textit{gnaſdo},
\textit{newaaſta}, \textit{niſczotha} (\textit{gniazdo},
\textit{niewiasta}, \textit{niszczota}), tak (można powiedzieć) o
\textit{p} miękkim: kiedy się zaokrągla, winno się stawać (w wymowie)
miękkie przed następującymi jakimikolwiek samogłoskami albo kiedy
występuje na końcu. Przykład pierwszego z a\textit{}: \textit{paaſek}
(\textit{piāsek}), z \textit{eq}: \textit{pechna} (\textit{Piechna}),
z \textit{i}: \textit{piſſarz} (\textit{pisarz}), \textit{piſſczek}
(\textit{piszczek}), z \textit{o}: \textit{Potr} (\textit{Piotr}), z
\textit{ą}: \textit{pøthno} (\textit{piątno}) 'piętno'.

Pozostają teraz do rozpatrzenia litery, które zmieniają brzmienie
inaczej niż przez twardnienie i zmiękczanie, a są to: \textit{c},
\textit{d}, \textit{g}, \textit{s}.


I tak, po pierwsze, niemała trudność występuje przy
\textit{c}. Albowiem \textit{c}, jak już powiedziałem, zmienia się w
brzmieniu pięciorako, zależnie od tego, z jakimi samogłoskami się
łączy. Bo połączone z \textit{a} zbiega się niekiedy w brzmieniu z
\textit{k} i niejako traci barwę głosową, np. \textit{cath} 'kat',
\textit{captuur} 'kaptur'. Tak go
\marginpar{s. 99}
też w tym samym brzmieniu używają Łacinnicy. I tak też łączy się u
\newtip{58}{'Głowa'.}
nich z \textit{a}, \textit{o} i \textit{u}, np. \textit{caput}¹,
\newtip{59}{'Szyja'.}
\textit{collum}¹, 
\newtip{60}{'Kukułka'.}
\textit{cuculus}¹. Niekiedy łączy się z innymi
samogłoskami i nie zbiega się w brzmieniu z \textit{k}, ale jest jakby
syczące, tj. zachowuje własne brzmienie, które ma w alfabecie. Tak go
też używają Łacinnicy, kiedy się łączy z \textit{e} lub z \textit{i},
\newtip{61}{'Cebula'.}
np. \textit{cepe}¹, 
\newtip{62}{'Pokarm'.}
\textit{cibus}¹, z \textit{a} również, ale
\newtip{63}{'Czyżyk'.}
bardzo rzadko, np. \textit{caix}¹.

Otóż wymawiane w ten (ostatni) sposób jest w języku polskim trojako
\newtip{64}{Dosłownie: grubo, grubiej i miękko. Określenie wymowy samogłoski \textit{cz}
jako „grosse” — grubo, a \textit{ć} - „grossius” — grubiej (i podobnie
samogłosek \textit{sz} i \textit{ś}) może dziwić. Może tu chodzić o określenia z pewnym
wartościowaniem czy nacechowaniem ekspresywnym tych głosek, których
nie było w języku łacińskim. Przymiotnik \textit{grossus} był często używany z
ujemnym nacechowaniem.}
szumiące: grubo, silniej i \textit{cienko}¹, i to niezależnie od
tego, z jakimi samogłoskami się łączy. Przykład pierwszego (rodzaju
brzmienia w połączeniu) z \textit{a}: \textit{czas}, \textit{czapka},
\textit{czasza}, \textit{czapla}(?). Przykład drugiego (rodzaju
brzmienia): \textit{ciało}, \textit{ciask}, \textit{ciemię}. Przykład
trzeciego: \textit{cap} 'kozioł' u Wołochów lub wykrzyknik oznaczający
\newtip{65}{Łoś objaśnił te wyrazy jako derywaty od niemieckich imion \textit{Hinze} (od
\textit{Heinrich}), \textit{Kunze} (od \textit{Konrad}), pełniące funkcję zleksykalizowanych
połączeń typu \textit{Paweł i Gaweł}.}
uderzenie, \textit{hynca} \textit{kunca}¹65. Również połączone z
\textit{e} jest szumiące trojako. Przykład pierwszego: \textit{czego},
\textit{czeka}. Przykład drugiego: \textit{ciemię},
\textit{cielę}. Przykład trzeciego: \textit{cepy},
\textit{cebula}. Tak samo przy połączeniu z pozostałymi samogłoskami,
np. \textit{czym}, \textit{czyn} 'oręż', \textit{czyń} 'czyń',
\textit{czop}, \textit{czuje}, \textit{cząbr}(?),
\textit{czosnek}. Wprowadzamy takie rozróżnienie (w pisowni), ażeby
należycie i w zróżnicowany sposób wyrażać w pismach te (słowa), tak
jak się różnią między sobą w brzmieniu. Bóżnicę zaznaczyć nie jest
łatwo, a jednak, jeśli łaska,
\marginpar{s. 100}
niech będzie wprowadzone takie zróżnicowanie. Jeśli łaska — ponieważ
\newtip{66}{Tzn. wyrażanie brzmień.}
wszystkie brzmienia¹ i wszystkie kształty (liter) są zależne od
upodobania twórcy i jego zwolenników.

A więc w wypadku \textit{c} szumiącego wymawianego grubo, jak w
przykładach pierwszej grupy: \textit{czas}, \textit{czego},
\textit{czym}, \textit{czółka}, \textit{czubacz}, pisze się \textit{c}
razem z \textit{z}, jak się przywykło pisać. A to (pisanie) uznamy tym
skwapliwiej, im je szybciej wypróbujemy w codziennym używaniu.

\newtip{67}{Takie tłumaczenie łac. „grossius” może nie być odpowiednie.}
A kiedy się wymawia silniej szumiąco (cisząco)¹, jak w przykładach
drugiego typu z \textit{a} i z \textit{o}, będziemy mogli łatwo
oznaczyć różnicę. I tak, jeśli napiszemy \textit{cz} i \textit{y},
np. \textit{czyalo}, \textit{czÿaſɲo}, \textit{czÿemyø},
\textit{czÿelø}, \textit{czÿolek}, \textit{czÿvlaa}, \textit{czÿągne}
(\textit{ciało}, \textit{ciasno}, \textit{ciemią} 'ciemię',
\textit{cielą} 'cielę', \textit{ciołek}, \textit{ciułā},
\textit{cichnie}), to w tych wyrazach różnica będzie wprawdzie
dostatecznie wyrazista, ale w innych taka różnica (pisowni) żadnym
sposobem nie wystarczy, mianowicie, kiedy się spotyka przykłady z
\textit{i}: \textit{czym}, \textit{czyn}, co tu jest przykładem
szumiącej wymowy pierwszego typu. A jakbyśmy pisali \textit{cin} z
drugim typem szumu, czyli sylabę z wyrazów takich jak
\textit{Bozącin}, \textit{Prądocin}, \textit{Kocin} lub podobnych?
Nie wiem bowiem, jakby się różniły w piśmie wyrazy: \textit{czym} 'z
czym' z pierwszym typem szumu i \textit{czyn} (\textit{cin}) z drugim
typem będące sylabą w wymienionych wyrazach \textit{Bozącin} i
podobnych. A tak samo \textit{czÿrpaaɬ} 'czerpał' i \textit{czÿrpaɬ}
'cierpiał', przynajmniej w pierwszej sylabie, o której jest
mowa. Różnią się przecież te sylaby \textit{p} twardym i miękkim. Tak
jest również ze \textit{ſczÿrpaa} (\textit{ścirpā}) 'cierpnie,
drętwieje' i \textit{ſczyrpaa} (\textit{sczyrpā}) 'sczerpie' i wielu
podobnymi.

Z tej przyczyny — jeśli łaska — ilekroć się trafi owa bardziej
szumiąca (cisząca) wymowa, to w miejscu \textit{cz} i podwójnego
\textit{y} będziemy pisać pojedyncze \textit{c} z jednym pociągnięciem
w dolnej części, jak to starożytni pisali zamiast \textit{z}, co widać
% A763 LATIN SMALL LETTER VISIGOTHIC Z (Michael Everson * http://www.evertype.com/ Tue, 12 May 2015)
w dawnych księgach, mianowicie tak: \textit{ç} i {\Quivira ꝣ}.
% Italics only in Everson fonts:
%\textit{ꝣ}. 
I tej litery będziemy
używać zawsze, ilekroć się nam trafi \textit{c}, \textit{z} i
\newtip{68}{Tj. nie wymawiane.}
\textit{i} zduszone¹, tj. (\textit{c}) silniej
\marginpar{s. 101}
szumiące (ciszące), jak w wymienionych przykładach: \textit{çalo},
\textit{çaſɲo}, \textit{çemø}, \textit{çolek}, \textit{çula},
\textit{çøgwe} (\textit{ciało}, \textit{ciasno}, \textit{ciemią}
'ciemię', \textit{ciołek}, \textit{ciuła}, \textit{ciągnie}). I tym
sposobem będzie (zaznaczona) wyraźna różnica między (wyrazami)
pierwszego typu: \textit{czym} i \textit{czyn} i (wyrazami drugiego
typu): \textit{chciēj} 'chciej', \textit{cirpiāł}, \textit{ścirpiāł} i
\textit{cipāł}.

Zaś w (wymówieniach) szumiących cienko (syczących) niech będzie pisane
pojedyncze \textit{c} bez (oznaczenia) jakiegokolwiek szumu,
\newtip{69}{\textit{Cyż} zamiast \textit{czyż} może być wynikiem mazurzenia lub
dysymilacji szumiących.}
np. \textit{cap}, \textit{cebuɬa}, \textit{ciſz} (cyż)¹ 'czyżyk', z
\textit{o}: \textit{co} 'co', \textit{cuudni} (\textit{cūdny})
'piękny'.

Wszystkie zaś inne wyrazy zaczynające się na \textit{c} lub po
\textit{c} mające \textit{a}, \textit{o}, \textit{u} i \textit{ą}
niech będą pisane przez \textit{k}. Przykłady pierwszego (typu, przed
samogłoskami): \textit{kath} \textit{kameen} \textit{koth}
\textit{kooth} \textit{kuurcz} \textit{kuur} (\textit{kat},
\textit{kamiēń}, \textit{kot}, \textit{kōt}, \textit{kūrcz},
\textit{kūr}). Przykłady drugiego (typu, przed spółgłoskami):
\textit{kɲaaƥ} \textit{kmotr} \textit{kroɬ} \textit{ktho}
(\textit{knāp}, \textit{kmotr}, \textit{krol}, \textit{kto}).

I tak samo można by postąpić z \textit{q}, pisząc w jego miejsce
\textit{k}, np. \textit{kvap} \textit{kvath} \textit{kveli}
\textit{kvikaa} (\textit{kwap}, \textit{kwiat}, \textit{kwieli},
\textit{kwika}). \textit{Q} bowiem także u Łacinników jest w pewnych
\newtip{70}{'Ktokolwiek'.}
wyrazach zbyteczne, z wyjątkiem \textit{quicumque}¹ i
\newtip{71}{‘Komukolwiek’.}
\textit{cuicumque}¹, które nie mogą być pisane wspomnianym sposobem,
czyli przez \textit{k}. Istotna różnica owych dwu wyrazów:
\newtip{72}{`Kto'.}
\textit{quicumque} i \textit{cuicumque} lub \textit{qui}¹ i
\newtip{73}{`Komu’.}
\textit{cui}¹ 
tkwi jednak nie w \textit{q} ani w \textit{c}, ale w
spółgłosce i samogłosce \textit{v} i \textit{u}. Dlatego gdyby
Łacinnicy zechcieli używać takiego jak nasze zróżnicowania między
samogłoską \textit{u} i spółgłoską \textit{v}, nie potrzebowaliby
\textit{q}, ponieważ w jego miejscu wystarczyłoby (pisać) \textit{c},
którego oni (i tak) używają. My na jego miejscu będziemy pisać \textit{k}.

Przy \textit{ch} szczelinowym natomiast nie ma w pisowni żadnej trudności,
ponieważ zatrzymuje ono dawną postać, tzn. 
\marginpar{s. 102}
pisownię przez \textit{c} i \textit{h}, np. \textit{chɬeƀ},
\textit{chmeeɬ}, \textit{chaarth}, \textit{chroſt} (\textup{chleb},
\textit{chmiēl}, \textit{chārt}, \textit{chrost}).

\textit{D} ma także dwa brzmienia, ale odróżniające się łatwo,
mianowicie \textit{d} pojedyncze, czyli twarde i \textit{dź}
miękkie. Przykład pierwszego: \textit{dal}, \textit{dēnko},
\textit{dym}, \textit{dōm}, \textit{dumā}, \textit{dąb}. Przykład
drugiego: \textit{dział}, \textit{dzień}, \textit{dziw},
\textit{dziolda}, \textit{dziula}, \textit{dziągil} 'dzięgiel',
\newtip{74}{W dotychczasowej pisowni.}
\textit{dziąsła}. Pierwsze przykłady pisze się¹ przez \textit{d}
pojedyncze, pozostałe przez \textit{d} i \textit{z}, co jest w dość
powszechnym użyciu.

Poza tym jednak \textit{dz}, w sytuacjach skądinąd identycznych,
czasem się wymawia twardziej, czasem bardziej miękko. Przykład
pierwszego: \textit{gwiżdż} 'gwiżdż!’, \textit{gwiżdż} 'orzech
przedziurawiony przez robaka’. Przykład drugiego: \textit{gwiźdź}
'tylna część siodła’. W każdym z tych (dwu typów) przykładów
\textit{dz} brzmi inaczej. I chociaż się wydaje, że to twardnienie i
miękczenie wynika częściowo z \textit{d} i \textit{z}, jednak naprawdę
pierwsze \textit{dz} jest wymawiane twardziej niż drugie. Odpowiedniej
różnicy między nimi nie możemy oddać (w pisowni) i rozróżnianie
brzmienia pozostawiamy bystrości czytelnika. Jeśli bowiem wynalazca
alfabetu łacińskiego, który miał (w tym względzie) najwyższe
umiejętności, pozostawił zdolności czytelnika (rozróżnianie w wymowie)
znacznej części liter, czemuż by nam nie było dozwolone to, co wolno
było bardziej doświadczonym.

(G). Chociaż \textit{g} u Łacinników ma także dwa
\newtip{75}{W wymowie \textit{gaudium}.}
brzmienia, np. \textit{gaudium}¹ — 
\newtip{76}{W wymowie \textit{jenus}.}
\textit{genus}¹, to jednak oni sami nie dbają,
ażeby tę różnicę w piśmie oznaczać, ale rozróżnianie pozostawiają
czytającym. My zaś podjąwszy się wyznaczyć różnice między brzmieniami
różnych liter, i tę (różnicę) wyznaczymy.

Są w polskim tego rodzaju wyrazy, że na początku, w środku lub na
końcu mają, literę \textit{g} w obu wspomnianych brzmieniach. Przykład
na wymowę pierwszego typu to (wyrazy): \textit{gād}, \textit{Gedka},
\textit{gid}, \textit{Godek}, \textit{guz}, \textit{gąś},
\textit{Magda}, \textit{migdał}, \textit{rōg}, \textit{smug}. Przykład
na wymowę drugiego typu to \textit{jē}, \textit{jēmy} i inne formy
odmiany (tego wyrazu). Niech więc wyrazy pierwszego typu będą 
\marginpar{s. 103}
pisane przez \textit{ɠ} z ogonkiem zakrzywionym w prawą stronę, tak jak
je piszą Italczycy, np. \textit{ɠraad}, \textit{ɠruuda},
\textit{ɠroſch} (\textit{grād}, \textit{grūda}, \textit{grosz}), a
wyrazy drugiego typu przez \textit{g} proste, jak jest używane, z
ogonkiem zwyczajnym, zagiętym w lewą stronę, tj. zwrócone
(wybrzuszeniem) na prawo, np. \textit{gee}, \textit{geeɱi},
\textit{gyɱ} (\textit{jē}, \textit{jēmy}, \textit{jim}).

Nie można pominąć milczeniem i tego, że Łacinnicy wyrażają niemal tym
samym dźwiękiem spółgłoskę \textit{i} i literę \textit{g}, pisaną
zgodnie z tym, cośmy tu mówili, bez ogonka zakrzywionego,
\newtip{77}{Tj. Genua.}
np. \textit{Ianua}¹, \textit{Iaronimus}, \textit{Iohannes},
\textit{Iunius}, \textit{genus}, \textit{gymnasium}. I jest konieczne,
ażeby Polacy tak samo czynili. Jeśli się więc ma chęć ukryć tę różnicę
przed czytającymi i nie zaznaczać jej żadnymi znakami, to, owszem,
dobrze, lecz łatwiej byłoby czytać, gdyby różnica została
zaznaczona. A moglibyśmy tę różnicę w wyrazach polskich zaznaczyć tym
sposobem, że ilekroć się pisze \textit{g} połączone z jakąś
samogłoską, a nie pisze się go przez \textit{g} okrągłe, wówczas się
nie powinno pisać zwykłego \textit{g}, lecz spółgłoskę \textit{I},
np. \textit{Iacuuſz}, \textit{Iaan}, \textit{Iost} (\textit{Jakūsz},
\textit{Jān}, \textit{Jost}), przy czym to ostatnie jest imieniem
świętego Jodoka. Podobnie ilekroć się (\textit{g}) łączy z
następującymi spółgłoskami: \textit{b}, \textit{n}, \textit{w},
\textit{d}, \textit{l}, \textit{r}, wymawia się je (jak \textit{g}) i
pisze się z ogonkiem zakrzywionym, np. \textit{maɠda}, \textit{ɠlaſz},
\textit{ɠnew}, \textit{ɠrod} (\textit{Magda}, \textit{głaz},
\textit{gniew}, \textit{grod}). Ale jeśli się łączy z samogłoskami
\textit{e} i \textit{i}, to chociaż się w wymowie różni od \textit{g}
zakrzywionego, nie należy w języku polskim pisać (tej głoski) przez
\newtip{78}{Tj. literę.}
spółgłoskę¹ \textit{i}, lecz owszem przez \textit{g},
np. \textit{geʋa}, \textit{geeɱi}, \textit{geɱu}, \textit{gÿɱ}
(\textit{Jewa}, \textit{jēmy}, \textit{jemu}, \textit{jim}), co
zresztą i w łacinie znajduje potwierdzenie.

Trzeba mieć ponadto na uwadze, że u Łacinników spółgłoska \textit{j}
łączy się ze wszystkimi samogłoskami, jak w przykładach wyżej
przytoczonych. I wszystkie wyrazy, gdzie \textit{g} łączy się z
\textit{a}, \textit{o} i \textit{u} i gdzie spółgłoska \textit{j}
zgadza się w brzmieniu z \textit{g}, zawsze są pisane przez
\textit{i}, a nigdy nie są pisane przez \textit{g}. Przykład: \textit{Iacobus},
\newtip{79}{'Brama'.}
\textit{Ianua}¹, na \textit{o}: 
\newtip{80}{‘Żart’.}
\textit{Iocus}¹, 
\newtip{81}{`Żartowniś'.}
\textit{ioculator}¹, na \textit{u}: 
\newtip{82}{'Prawo'.}
\textit{ius}¹,
\break 
\marginpar{s. 104}
\newtip{83}{'Sąd'.}
\textit{iudicium}¹. Gdzie zaś spółgłoska \textit{j} łączy się z
samogłoską \textit{e}, tam czasem piszą spółgłoskę \textit{i}, czasem
literę \textit{g}. Przykład: \textit{Ieronimus}, I\textit{eremias}, a
to prawie zawsze w imionach własnych. W pospolitych zaś najczęściej
\newtip{84}{'Rodzaj'.}
piszą przez \textit{g}, np. \textit{genus}¹, 
\newtip{85}{'Kolano’.}
\textit{genu}¹. Mówię
wyraźnie: najczęściej, ponieważ niekiedy także wyrazy pospolite są
\newtip{86}{'Post'.}
pisane przez \textit{i}, np. \textit{ieiunium}¹,
\newtip{87}{'Wątroba’.}
\textit{iecur}¹. Gdziekolwiek zaś \textit{g} łączy się z \textit{i},
jest z konieczności pisane przez \textit{g}, np. \textit{gymnaſium},
\newtip{88}{'Dwadzieścia'.}
\textit{viginti}¹, ponieważ nie może być pisane przez spółgłoskę
\textit{i}.

Pozostaje jeszcze przedstawić literę \textit{S} w jej
zróżnicowaniu. Więc najpierw trzeba zauważyć, że \textit{s} u
Łacinników ma zwyczajowo dwie postacie. Mianowicie, po pierwsze, pisze
się je z przeciągnięciem na długość, zwykle na początku i w środku
\newtip{89}{'Słońce'.}
wyrazów, np. \textit{ſol}¹, 
\newtip{90}{'Sól’.}
\textit{ſal}¹, 
\newtip{91}{'Wysłał’.}
\textit{miſit}¹,\newtip{92}{`Msza'.}\textit{miſſa}¹. Innym sposobem się pisze \textit{s} skręcone,
\newtip{93}{'Opat'.}
zwykle na końcu wyrazów, np. \textit{abbas}¹,
\newtip{94}{'Mężczyzna’.}
\textit{mas}¹. Oprócz tego zauważyć trzeba, że \textit{s} użyte bez
żadnego specjalnego oznaczenia przydechu, czyli szumu, czasem się
pisze zgodnie z jego właściwym brzmieniem, tj. tak jak zwykle brzmi
położone na końcu wyrazów, np. \textit{abbas}, \textit{mas},
\newtip{95}{'Kocioł’.}
\textit{lebes}¹. Tak samo, kiedy się znajduje na początku wyrazów,
poprzedzając samogłoskę lub spółgłoskę. Przykład pierwszego (typu,
tj. przed samogłoską) : \textit{ſaam} \textit{seen} \textit{Syɲ}
\textit{Sowa} \textit{Suuɱ} \textit{Søød} (\textit{sām}, \textit{sēn},
\textit{syn}, \textit{sowa}, \textit{sūm}, \textit{są̄d}). Przykład
drugiego (typu, tj. przed spółgłoską): \textit{Struſ} \textit{Sbil}
\newtip{96}{Wymowa \textit{s}-: \textit{sbit} czy \textit{sbył}, (por. notkę nr 70
do tekstu łacińskiego), \textit{sgaga} nie jest całkiem pewna, ale
bardzo prawdopodobna, skoro to sąprzykłady na \textit{s} w
nagłosie. Za taką wymową wypowiada się Łoś i nie uznaje twierdzenia
H. Ułaszyna (Materiały i Prace Komisji Językowej AU V, 1912,
s. 272-4), że wymawiano tu \textit{z}-.}
\textit{Sɠaɠa} \textit{Smøød} \textit{Spøød} (\textit{strus}, \textit{sbil}, \textit{sgaga}¹,
\textit{smą̄d}, \textit{spą̄d}).

Jeśli zaś, jak już wspomniano, (\textit{s}) położone koło
samogłoski albo spółgłoski staje się bardziej cienkie albo subtelne,
wtedy traci swój charakter i przechodzi w inną literę, mianowicie
\marginpar{s. 105}
w \textit{z}. Przykład pierwszego (typu, tj. przed samogłoską):
\textit{zawada}, \textit{Zygmunt}, \textit{Zofia}, \textit{Zuzanna},
\textit{zą̄b}.  Przykład drugiego (typu, tj. przed spółgłoską):
\textit{zmuda}, \textit{zbił} (lub \textit{zbył}),
\textit{zdrow}. Dlatego też w naszym polskim alfabecie \textit{z}
umieszczamy po \textit{s} z powodu bliskości brzmienia.

Czasem jednak \textit{s} staje się cienkie, a nie traci swojej
właściwości, mianowicie kiedy się znajduje między dwiema samogłoskami
posiadającymi odrębny charakter. Przykład na \textit{a}:
\textit{ɱaſal}, \textit{koſa}, \textit{kaaſaal} (\textit{mazał},
\textit{koza}, \textit{kāzāł}). Stąd, chociaż jest tu wymawiane
cienko, pisze się jednak swoim własnym kształtem, jak w przykładach
podanych wyżej. A jeśli ma wtedy zachować własne brzmienie, należy je
\newtip{97}{Może zamiast \textit{coſſa}.}
podwajać. Przykład: \textit{caſſa}¹, \textit{miſſa}, \textit{Koſſa}
(czy \textit{Roſſa}) (\textit{kasa} lub \textit{kosa}, \textit{misa},
\textit{kosa} lub \textit{rosa}).

Jeśli więc chcemy mieć właściwe rozróżnienie między \textit{s}
występującym we własnym brzmieniu, a \textit{s} cienkim,
np. znajdującym się między dwiema samogłoskami, piszmy pierwsze
zakręcone, tak jak jest w zwyczaju pisać na końcu wyrazów. Przykład:
\textit{Saam}, \textit{Seen}, \textit{Syɲ} (\textit{sām},
\textit{sēn}, \textit{syn}), \textit{Søød} (\textit{są̄d}) 'sąd',
\textit{SSøød} (\textit{ssą̄d}) 'naczynie'. Tak samo w połączeniu ze
spółgłoskami: \textit{Stado}, \textit{Sluɠa}, \textit{Sboſzɲi},
\textit{Struf}, \textit{Smøød} (\textit{stado}, \textit{sługa},
\textit{sbożny}, \textit{strus}, \textit{smą̄d}) itd. Gdzie się zaś
(\textit{s}) staje cienkie, niech będzie pisane pojedyncze i długie,
np. \textit{maſal}, \textit{cofa} (\textit{mazał}, \textit{koza}). I
tym sposobem ilekroć spotkamy ſ długie, będziemy je czytać bardziej
cienko, niż gdyby było podwojone jak w \textit{miſſa} 'misa', \textit{møſſo}
(miąso).

Niekiedy zaś \textit{s} nie jest cienkie ani też nie występuje w swoim
właściwym brzmieniu, lecz czasem staje się szumiące, a to gdy jest
połączone (w pisowni) z następującym \textit{ch}, np. \textit{ſchadi},
\textit{ſchipee}, \textit{ſchiɱuɲ}, \textit{ſchopa}, \textit{ſchum},
\textit{ſchamɓurzaa} (\textit{szady}, \textit{szypie},
\textit{Szymun}, \textit{szopa}, \textit{szum},
\textit{szamburzā}).

\newtip{98}{Por. notkę nr 64.}
Czasem też szumi (inaczej)¹, i to
wielorako.

Czasem mianowicie szumi grubo, np. \textit{ſzaak}, \textit{ſzegotha},
\textit{ſzywofh}, \textit{ſzoraʋ}, \textit{ſzuur}, \textit{ſzøødɬo}
(\textit{żak}, \textit{Żegota}, \textit{żywot}, \textit{żoraw},
\textit{żūr}, \textit{żą̄dło}). I wtedy się pisze przez \textit{ſ}
długie i \textit{z}, i to widać w przykładach.
\marginpar{s. 106}
Niekiedy jest silniej szumiące (ciszące), kiedy \textit{s} zatrzymuje
\newtip{99}{W rkp. chyba błędnie: \textit{s} (\textit{ſ}).}
własne brzmienie i przybiera dodatkowo jakby \textit{i}¹. I wówczas
winno się pisać przez ſſ rozciągnięte, czyli długie podwojone, i
\textit{z}. Przykład: \textit{ſſzano}, \textit{ſſzemø},
\textit{ſſzadlo}, \textit{ſſzivi}, \textit{ſſzirotha},
\textit{ſſzoſtra}, \textit{ſſzuthka} (\textit{siano}, \textit{siemią},
\textit{siadło}, \textit{siwy}, \textit{sirota}, \textit{siostra},
\newtip{100}{Czytać można \textit{siatka} lub \textit{siutka}. Za -\textit{u}-
przemawia kolejność alfabetyczna samogłosek}
\textit{siutka}¹).I nie wystarczy w tych i podobnych przykładach
pisać pojedynczego ſ długiego i \textit{z}, i podwójnego \textit{ÿ},
jak niegdyś bywało pisane. Przedtem bowiem tak bywało pisane:
\textit{fzÿano}, \textit{ſzÿemÿa}, \textit{fzÿoſtra} (\textit{siano},
\textit{siemią} 'siemię', \textit{siostra}). Chociaż rzeczywiście w
wyrazach, gdzie po \textit{sz} następuje \textit{a}, \textit{e},
\textit{o}, \textit{u} i \textit{ą}, można także tak pisać, to jednak
tam, gdzie po tej szumiącej spółgłosce następuje \textit{i},
np. \textit{ſziwi}, \textit{ſzirotha} (\textit{siwy},
\textit{sirota}), nie byłoby różnicy między \textit{ſzÿwÿ} 'siwy' a
\textit{ſziwi} 'żywy' i wielu podobnymi. Będzie więc lepsza i
znaczniejsza różnica, kiedy w tym wypadku silniejszego szumu (wymowy
ciszącej) będzie pisane podwójne długie zakręcone ſſ, niezależnie od
tego, czy (po nim) następuje \textit{i}, czy nie.


Czasem zaś (\textit{s}) jest wymawiane jako szumiące bardziej cienko,
jak gdyby się zbliżało do \textit{z}, a to \textit{z} niejako się
wzmacnia. I dlatego najwłaściwiej by je było pisać przez podwójne
\textit{zz}, np. \textit{zzaia}, \textit{zzeɬe}, \textit{zziɱa},
\textit{zzolo}, \textit{zzøba} (\textit{ziaja} lub \textit{ziara},
\textit{ziele}, \textit{zima}, \textit{zioło}, \textit{ziąba}
'zięba'). Nie wystarczy tu bowiem pisać pojedynczego \textit{ſ} i z z
podwójnym \textit{ÿ}, jak to dawniej bywało pisane,
np. \textit{ſzÿarno}, \textit{ſzÿemÿa} (\textit{ziarno},
\textit{ziemia}). Bo chociaż w tych wyrazach, gdzie po \textit{ź}
następuje \textit{a}, \textit{e}, \textit{o}, \textit{u} i \textit{ą},
taką pisownią czasem byśmy wyrazili różnicę między \textit{z}
pojedynczym a tym (innym), to jednak tam, gdzie po \textit{z}
następuje \textit{i}, między \textit{z} pojedynczym a owym
wzmocnionym, czyli podwojonym, nie będzie żadnej różnicy. Przykład:
\textit{zzÿɱa}, \textit{zzÿrɱɲo}, \textit{zziɱozzeɬon} (\textit{zima},
\textit{zimno}, \textit{zimozielon}). Więc rozróżnienie, o którym
mówiłem, jest lepsze, bo się sprawdza we wszystkich wypadkach.

Wreszcie przy pozostałych literach, mianowicie \textit{h}, \textit{k},
\textit{q}, \textit{r}, \textit{t}, \textit{x}, \textit{ÿ} i
\textit{z} — poza tym, że ta ostatnia w połączeniu z \textit{ſ} staje
się 
\marginpar{s. 107}
szumiącą — nie ma już żadnej trudności. Tak w pisowni, jak w wymowie
nie różnią się one od liter łacińskich, z wyjątkiem tego, że jak
powiedziano, będziemy używać \textit{k} zamiast \textit{c}
przytłumionego. Bo chociaż przedtem wyrazy: \textit{kat},
\textit{kłoda}, \textit{knap}, \textit{kmotr}, \textit{krōl},
\textit{kopa} itd. były pisane przez \textit{c}, my będziemy odtąd
wszystkie te wyrazy pisać przez \textit{k}, a więc \textit{kath},
\textit{kłoda}, \textit{kɲap}, \textit{kɱotr}, \textit{kroɬ},
\textit{kunrath}, \textit{køøth} (\ldots \textit{Kunrat},
\textit{ką̄t}). A \textit{q}, jak u Łacinników, tak i u Polaków jest
naprawdę zbyteczne. Bo zgodnie z tym, co już powiedziano, jak zamiast
przytłumionego \textit{c} używamy \textit{k}, tak i zamiast \textit{q}
możemy używać \textit{k}. Przykład w pisowni używanej (dotąd):
\textit{Quas}, \textit{quath}, \textit{quap}, \textit{qveli},
\textit{kvikaaq} (\textit{kwas}, \textit{kwiatq}, \textit{kwap},
\textit{kwieli}, \textit{kwika}). I nikt nas nie będzie mógł słusznie
zganić.

Na koniec: Są jeszcze (w piśmie) u Łacinników pewne znaki nie będące
wprawdzie literami, ale zastępnikami liter. Są to skróty albo
wierzchołki, zwane summitates, ponieważ częściej są umieszczane w
części górnej, a mianowicie:{\Cardo ⁊̵} (\textit{eciam} 'także'),{\Cardo ꝯ̃} (contra
'przeciw'), {\Cardo ſꝫ} (\textit{scilicet} `a mianowicie, rozumie się'), {\Cardo ꝯ}
(\textit{con}, \textit{com}, \textit{cum} — sylaba początkowa), {\Cardo t  }
(\textit{tur} — końcówka). My ich nie potrzebujemy dopisywać do
naszych liter — zarówno dlatego, że Polacy rzadko tych wierzchołków
czy skrótów, czyli zastępników używają, jak i dlatego, że gdyby ich
używać chcieli, to będą je pisać podobnie jak w łacinie.

Nie można również pominąć milczeniem tego, że wszystkie litery
abecadła, gdy są umieszczane na początku wersów albo rozdziałów, albo
akapitów, różnią się często właściwościami i kształtami od swoich
zwykłych właściwości i kształtów. My je pospolicie nazywamy
wersalikami albo kapitalikami, a nazwa ta jest trafna dla nagłówków. A
że różni różnymi sposobami rzeczone wersaliki i kapitaliki pisząc
dowolnie formują, przeto nie staramy się ustalać w tym względzie
reguł, ponieważ się (te litery) różnią zależnie od praktyki i nawyku
piszących. Na końcu zaś naszego abecadła dodamy ich
\newtip{101}{Tych przykładów jednak nie ma.}
przykłady.¹

Następuje teraz abecadło z rozróżnieniami i przykładami
omówionych właściwości i kształtów (liter). Uprzedzam jednak, że
\marginpar{s. 108}
litery w naszym abecadle mają z pewnych przyczyn nie ten sam porządek,
co w łacińskim. Albowiem \textit{k} nie jest umieszczone między
\textit{i} i \textit{l}, ale przed \textit{d}, żeby była bardziej
wyraźna różnica między \textit{c} we właściwym brzmieniu, a \textit{c}
przytłumionym, zamiast którego, jak powiedziano przedtem, będziemy
pisać \textit{k}. Również \textit{i} jest umieszczone przed \textit{h}
dla powiązania przykładów, a \textit{y} będzie następować zaraz po
\textit{i}, ażeby widać było między nimi różnicę. I \textit{z} (następuje)
zaraz po \textit{s} ze względu na podobieństwo brzmienia. Dalej już możemy
wszystkie omówione litery, czyli ich przykłady, umieścić na właściwych
miejscach, tak je zresztą umieścimy. Nie jest bowiem uchybieniem to
samo wiele razy zasadnie powtarzać. Itd.

\begin{quote}
Kto chce pisać doskonale 

język polski i też prawie, 

umiej obiecado moje, 

ktorem tak napisał tobie, 

aby pisał tak krótkie \textit{a}, \newtip{102}{Dla ułatwienia czytania dajemy tutaj
    \textit{ę} i \textit{ą} zamiast jednej nosówki (\textit{ą}).}  

\textit{aa} sowito, gdzie się¹ wzdłużā.  

Podług tego będzie pisān 

ludzi wszystkich ociec Adām.  

A też gdzie \textit{ƀ} będzie grube, 

tako pismem położysz jē.  

Nie pisz wirzchu okrągłego, 

pisząc \textit{Bartka zbawionego}.  

Z wirzchem okrągłem piszy \textit{ɓ}, \

textit{biodry} tako napiszesz swē.  

Gdzie \textit{c} głosu mieć nie będzie, 

w miasto jego tam \textit{k} siędzie.  

Jako \textit{kamiēń}, tako \textit{kaptūr}, 

pisan będzie przez \textit{k} i \textit{kūr}.

Ale gdzie \textit{c} swoj głos miewa, 

zwykłem pismem \textit{cało} tak da.  

Ale \textit{cielęciu} i \textit{ciału} \marginpar{s. 109}

pod ç przypiszy tak jemu.  

Jestli \textit{c} barzo grubieje, 

tako pismem \textit{czas} wyznaje.  

Gdy \textit{h} przypisano będzie,

\textit{ch} \textit{chwalebne} tako siędzie.  

Ale bych ci nic przedłużył 

ani teskności uczynił, 

patrzy obiecada mēgo 

tobie tu napisanego, 

boć w niem każde słowko tobie 

pismem rozny głos da w sobie.  

Pisz jē w jimię Boże tako, 

jeżem ci napisał jako.
\end{quote}

{
\obeylines
krótkie długie twarde miękkie
a ā b b b́
c ma w języku polskim pięć różnych odmian:
c k ć cz ch
właściwe miękkie krótkie długie twarde miękkie
d dź e ē f f́ 
właściwe niewłaściwe krótkie długie twarde miękkie
j g i(y) ī (ȳ) ł l
twarde miękkie twarde miękkie krótkie długiem

m ḿ n ń o ó
twarde miękkie właściwe właściwe
p ṕ q R r
S ma w języku polskim sześć swoich różnych odmian: 
z s ś sz ż ź 
krótkie długie
t u ū
kiedy (u) traci właściwości samogłoski, ma trzy
różne odmiany, które są widoczne w polskim abecadle, mianowicie w \textit{Adām}
\textit{byl} itd.:
ẃ w wu x y

}


% Rafał Wójcik (ed.)

% Culture of Memory
% in East Central Europe
% in the Late Middle Ages
% and the Early Modern Period
% Conference proceedings
% Ciążeń, March 12-14, 2008

% Biblioteka Uniwersytecka
% Poznań 2008

% Rafał Wójcik, Wiesław Wydra
% (Poznań)

% Jakub Parkoszowic’s Polish Mnemonic Verse
% about Polish Orthography from the 15th Century

\begin{quote}
  Adam̄ byl, bił, cał, kāł,

  czas, ciało, chōd, dāł, dziāł,

  eż fytā figi

  i jē je, hān, krol

  łys, lis mykā,

  Mika nyski niski,

  otōż \marginpar{s. 110} pije pyszno kwas,

  rosa, rzą̄sa, rozūm,

  sām, szād, siādł,

  żak ziarā za mną,

  to umiē uj(?), wiła

  wylāł w usta, ksią̄dz

  jęcząc jęczy, ją̄kā.
\end{quote}

Amen. A
trzeciego (alfabetu brak ?).

Ktokolwiek więc chce pisać w języku
polskim w sposób należyty i oddający zróżnicowanie wymowy, winien brać
pod uwagę przedstawione tu właściwości abecadła i tę ortografię
wprowadzać w zwyczaj powszechny, pisząc zgodnie z podanymi tu
przepisami \textit{a} krótkie i następujące po nim inne litery z odróżniającymi
je cechami.

Na chwałę Boga wszechmogącego i jego rodzicielki chwalebnej
dziewicy Marii. Z pomocą tego, który stworzył wszystko z niczego i
któremu wszystko jest posłuszne, którego władza trwa, który króluj o
na wieki wieków. 

Amen.

pisał sługa jest kałżdej godziny Warzykowski.


\end{document}







%  \input Parkosz01-03
%  \newpage
%  \input Parkosz02-04
%  \newpage
%  \input Parkosz03-05
% % \newpage
%  \input Parkosz04-06
% \newpage
% \input Parkosz05-07
% \newpage
% \input Parkosz06-08
%\newpage
%  \input Parkosz07-09
% \newpage
%  \input Parkosz08-10
% \newpage
%  \input Parkosz09-11
% \newpage
%  \input Parkosz10-12
% \newpage
%  \input Parkosz11-13
% \newpage
 \input Parkosz12-14
\newpage
 \input Parkosz13-15
\newpage
 \input Parkosz14-16

% pomaga na automatyczne wyswietlanie adnotacji?:
\newpage
\end{document}

% FR: interpretationis itp.

47	Chodzi o znakowanie muzyczne dla tonacji moll i dur. Por. Riemann Musiklexikon herausgegeben von H. II. Eggebrecht, Bd. III Sachteil, Mainz 1967, s. 71.

Tytuł 	
Riemann Musik Lexikon : Sachteil / begonnen von Wilibald Gurlitt ; fortgeführt und hrsg. von Hans Heinrich Eggebrecht.
Wydanie 	
12., völlig neubearb. Aufl.
Adres wyd. 	
Mainz : B. Schott's Söhne, 1967.
Opis fiz. 	
XV, 1087 s. : nuty ; 27 cm.
Warianty tytułu 	
Musiklexikon
Uwagi 	
Bibliogr. przy hasłach.
Temat polski 	
Muzyka --słowniki.
Muzyka --encyklopedie.
Temat angielski 	
Music --Dictionaries --German.
Hasło dod: 	
Gurlitt, Wilibald (1889-1963). Wyd.
Eggebrecht, Hans Heinrich (1919-1999). Wyd.


	
BUW Wolny Dostęp
	
232609 [3]
	
	
1
	
Nie wypożycza się
	
ML100 .R594 1959 t.3; 

http://tex.stackexchange.com/questions/25249/how-do-i-use-a-particular-font-for-a-small-section-of-text-in-my-document/37251#37251


%%% Local Variables: 
%%% mode: LaTeX
%%% TeX-PDF-mode: t
%%% TeX-engine: luatex 
%%% default-input-method: "Parkosz-slash"
%%% End:
