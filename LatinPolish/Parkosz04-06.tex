\url{http://wbl.klf.uw.edu.pl/13/2/iParkosz.djvu?djvuopts=&page=46&zoom=width&showposition=0.5,0.18}

\plineno=48
\ppreviouspageno=3
\psublineno=2

\splitpreviouslines

{
\color{blue}
Recedat ergo demens 

}

\ppageno=4


\plineno=0

\fulllines


arguicio et accedat comodosa Polonorum lingue in scripto servicio. Etc.


\Polish{Niech więc ustąpi bezsensowny zarzut i niech się zacznie
pożyteczna służba dla języka polskiego. Itd.}


\dc{V}iginti duas litteras apud Hebreos, Siros et Caldeos in prologo

\splitlines

super libros Regum beatus scribit Hieronimus. 

\newtip{29}{Jedne z ksiąg Pisma św.}
\Polish{Święty Hieronim pisze w przedmowie do \textit{Ksiąg Królewskich}¹, że
% 29	Jedne z ksiąg Pisma św.
Hebrajczycy, Syryjczycy i Chaldejczycy mają dwadzieścia dwie
litery.}

Latini autem una

\fulllines

magis habent litteras, caracteribus quidem et figuris sed non in omnibus

\splitlines

vocibus differentes et obtusum seu raucum \ment{k} et \ment{q}. 

\Polish{ Łacinnicy zaś mają o jedną literę więcej oraz przytłumione,
chrapliwe \ment{k} i \ment{q}.}

Quamvis quandoque

\fulllines

eidem caracteri diversos sonos attribuant, alique tamen eorum apud eos

\splitlines

superfluunt. 

\Polish{ Łacińskie litery różnią się
wprawdzie kształtami, ale nie wszystkie różnią się brzmieniem. Czasem
wprawdzie znak literowy ma różne brzmienia, zarazem jednak niektóre
litery są zbędne.}

\fulllines

Nam Latinum ideoma \ment{k} non indiget, \ment{h} eciam aspiracionis nota

\splitlines
est.

\Polish{ Nie potrzebuje bowiem język łaciński litery
\textit{k}, a \textit{h} jest także tylko znakiem przydechu.}

\indentK Nostrum autem Sclavonicum ideoma, et presertim Polonicum, multo

\splitlines

pluribus indiget litteris. 

\Polish{Nasz zaś słowiański, a zwłaszcza polski język potrzebuje o wiele
więcej liter.}

Primo namque cum Latini quinque vocalibus contenti sunt,

\fulllines

Poloni \extra{autem} sextam \ment{ø} adiciunt, nec sine ea illud ideoma scribi potest.

\newtip{30}{Transkrybowaną tutaj przez ą. [tutaj tylko w tłumaczeniu --- JSB]}
\Polish{ Przede wszystkim bowiem gdy Łacinnicy zadowalają się
pięcioma samogłoskami, to Polacy dorzucają szóstą: \ment{ø}¹, bez
której w języku polskim nie można pisać.}

Nam licet alicubi loco \ment{ø} \ment{an} scribi possit, ut \polski{ranka} \polski{manka}, \polski{røka}

\splitlines

\polski{møka}, alibi tamen nullo modo, ut \parkosz{møøka} \parkosz{drøga}. 

\Polish{ Co prawda, w pewnych
wypadkach można zamiast \ment{ø} pisać \ment{an},
np. \polski{ranka}, \ment{manka} \glosa{'ręka, męka'}, w innych wypadkach
jednak w żaden sposób nie można tak pisać, np. w \polski{mą̄ka},
\polski{drąga}.}

Alioquin inter penam et farinam

\splitlines

et multa alia non erit differencia.

\newtip{31}{Nie można pisać \polski{manka}, \polski{dranga}, bo
  można by to czytać zarówno \polski{mąka}, \polski{drąga}, jak i
  \polski{męka}, \polski{dręga}.}
\Polish{ W przeciwnym razie między \polski{męką} a
\polski{mąką} i wielu innymi nie będzie różnicy¹.}

\indentK Omnes eciam vocales apud Polonos

\splitlines

modo longantur, modo patulo breviantur. 

\Polish{%\marginpar{s. 91}
U Polaków wszystkie samogłoski to się wzdłużają, to się wyraźnie
skracają.}

Ex quarum longacione vel \hyphh{bre}{viacione}

splitlines

\hypht{bre}{viacione} diversus consurgit sensus diccionum. 

\Polish{Ze wzdłużania ich lub skracania wynika różne znaczenie
wyrazów.}

Exemplum de \ment{a} sicut \polski{\hyphh{verci}{maak}}

\fulllines

\polski{\hypht{verci}{maak}}, ubi si \ment{a} producitur, una diccio est, si corripitur, due sunt

\splitlines

dicciones. 

\newtip{32}{Tj. \polski{wiercimāk} — \polski{wierci} \polski{mak}. Objaśnienia znaczeń tych i
  innych wyrazów podano w indeksie. Krótkość samogłosek oznaczamy
  (łuczkami) tylko w niektórych przykładach.}
\Polish{Przykład na \ment{a}: \polski{wiercimāk}. Jeśli się tu
\ment{a} przedłuży, to będzie jeden wyraz, a jeśli się skróci — dwa
wyrazy¹.} 

\newpage

Exemplum de \ment{e} sicut \polski{beel} \polski{bel}. 


  \newtip{33}{Że rzeczownik \textit{biel} miał \textit{e} długie, a
  rozkaźnik \textit{biel}! \textit{e} krótkie, to potwierdza
  \textit{Słownik polszczyzny XVI wieku}, w którym rz. \textit{biel}
  ma w mianowniku i bierniku \textit{e} pochylone (\textit{é}),
  czasownik \textit{bielić} \textit{e} jasne (formy imperatiwu brak),
  imperativus \textit{dziel}! \textit{e} jasne.}  
\Polish{Przykład na \textit{e}: \textit{biēl} — \textit{biĕl}!¹}

Exemplum de \polski{i}: \polski{byl} \polski{bil}. 

\newtip{34}{Może należy czytać \polski{bił}.}
\Polish{Przykład na \ment{y} (\ment{i}): \polski{bȳł}¹ —}
\newtip{35}{Oba te przykłady niejasne. Drugi ma mieć samogłoskę krótką,
tymczasem w rzeczownikach przed dźwięcznym \ment{ł} występuje z reguły
samogłoska pochylona (dawna długa).}
\Polish{\polski{by̆ł}¹.}

Exemplum de \ment{o}, 

\splitlines

ubi longatur, ubi corripitur: \polski{kooth} \polski{koth}. 

\Polish{ Przykład na \ment{o}, gdzie się wzdłuża, a gdzie
skraca: \polski{kōt} (\polski{kót}) — \polski{kŏt}.}

\splitlines

De \ment{ø}: \polski{drøga} \polski{drøøga}.

\Polish{Na \ment{ą}: \polski{drą̆ga} — \polski{drą̄ga}
  (\polski{dręga} — \polski{drąga}).}

Exemplum de \ment{u}: \polski{druga} \polski{druug}.

\Polish{ Przykład na \textit{u}: \textit{drŭga} —
\textit{drūg}.}

\indentK Et quanquam Latini in produccione et \hyphh{correp}{cione}

\fulllines


\hypht{correp}{cione} vocalium in scribendo nullam aut paucam faciant differenciam,

pro eo, quia de quantitate — produccione et breviacione vocalium — \hyphh{suffi}{cientes}

\splitlines

\hypht{suffi}{cientes} habent regulas in libris grammaticorum, 

\Polish{Łacinnicy w zakresie długości i krótkości samogłosek w piśmie nie
stosują żadnej różnicy albo tylko niewielką, ponieważ co do iloczasu,
tj. wzdłużania i skracania samogłosek, mają wystarczające reguły w
pismach gramatyków,}

scilicet Prisciani,

\splitlines

\conf{Hebreardi}{}{\color{red}\footnotemark[27]}, Allexandri et aliorum positas, 
\footnotetext[27]{Prawdopodobnie błędnie zam. Heberhardi, tj. Eberliardi.}

\newtip{36}{Priscianus z Cezarei (Mauretania), V/VI w., autor gramatyki
łacińskiej pt. \textit{Institutiones grammaticae}.}
\Polish{mianowicie Prisciana¹, 
\newtip{37}{Eberhardus z Bethume, XIII w., autor wierszowanej gramatyki
pt. \textit{Graecismus}.}
Eberharda¹,
\newtip{38}{38 Aleksander de Yilla Dei, por. notkę nr 10. [TODO]}
Aleksandra¹ i innych.}

has igitur sciendas

\fulllines

per Latinos presupponunt, et ideo uno et eodem caractere seu figura

vocales breves et productas scribunt, ut brevitati scripturarum

\splitlines

consulant. 

\Polish{ Przypuszczają więc, że czytelnicy łacińscy
znają te reguły, i dlatego samogłoski krótkie i przedłużone piszą tym
samym sposobem, czyli tym samym kształtem, mając na względzie krótkość
pisma.}

\fulllines

Et quia has regulas prosodie Polonis illiteratis difficile

foret tradere, opportuit, ut inscribendo quantitas vocalium \hyphh{expri}{matur}

\splitlines

\hypht{expri}{matur}. 

\Polish{A ponieważ Polakom nie wykształconym (w zakresie gramatyki)
trudno by było wyłożyć reguły prozodii, należy więc (w języku polskim)
zaznaczać iloczas samogłosek w piśmie.}


Quod alio modo facere facile non est, nisi ut vocalis longa

\splitlines

geminetur et brevis simpla ponatur. 

\Polish{To zaś nie jest łatwo uczynić w
inny sposób, jak tylko podwajając samogłoskę długą, a pojedynczo
pisząc krótką.}

Ut \polski{adaam}, ubi primum \ment{a}

\splitlines

breve, secundum longum et geminatum. 

\Polish{Tak jak w \polski{Adaam}, gdzie pierwsze \ment{a}
jest krótkie, drugie długie i podwojone.}

Si enim differencias longe et brevis

\fulllines

vocalis tam in voce quam in caractere facere vellemus, forsan non

\splitlines

multum difficile foret. 

\Polish{%\marginpar{s. 92} 
Gdybyśmy chcieli różnice w wymawianiu długich i krótkich samogłosek
wyrażać za pomocą kształtu liter, nie byłoby to może zbyt
trudne.}

Nam Greci hunc modum scribendi servant, ut

\splitlines

aliter \ment{o} longum, aliter \ment{o} breve scribant. 

\newtip{39}{39	Tj. \ment{Ω} (omega) — \ment{ο} (omikron).}
\Polish{ Grecy bowiem zachowują taki sposób pisania, inaczej pisząc \ment{o}
długie, inaczej \ment{o} krótkie¹.}

Nos autem, qui omnia a

\fulllines

Latinis mutuavimus, hanc novitatem pretermittere voluimus, forsan

\splitlines

enim aliquibus foret odiosa.

\Polish{My jednak, którzy wszystko (w tym
względzie) zapożyczyliśmy od Łacinników, zdecydowaliśmy się poniechać
tej niezwykłości, bo mogłaby być niektórym niemiła.}

\indentK Necesse autem habemus quantitatem \hyphh{voca}{lium}

\fulllines

\hypht{voca}{lium} in scribendo geminando et simplando exprimere, quia, ut

\fulllines

premissum est, hoc pretermisso magna occurreret in distinguendis

\splitlines

significatis dicciorum difficultas. 

\Polish{Uważamy jednak za konieczne iloczas samogłosek wyrażać w piśmie przez
pisanie ich podwójnie lub pojedynczo, ponieważ, jak już wspomniano,
pominięcie tego nastręczałoby znaczne trudności w odróżnianiu
znaczeń wyrazów.}

Ideo etsi non omnis vocalis producta

\fulllines

geminabitur, saltem hoc observabitur, ubi ex eius breviatione et

produccione surgit notabilis diversitas significati eiusdem \hyphh{dic}{cionis}

\splitlines

\hypht{dic}{cionis}. 

\Polish{ Dlatego, choć nie każda długa samogłoska będzie (w
piśmie) podwajana, będzie się podwajania przestrzegać przynajmniej
tam, gdzie z jej skrócenia i wzdłużenia wynika wyraźna różnica
znaczenia tego samego wyrazu.}


Quamvis eciam et ex parte consonantium hec differencia notari possit, ut

\splitlines

infra dicetur.

\Polish{Chociaż zresztą i w zakresie spółgłosek
tego rodzaju różnicę można również zaznaczyć, o czym będzie mowa
niżej.}


\indentK \dc{P}orro omnes fere littere, exceptis scilicet \over{\ment{s}} \ment{h} \over{\ment{v}} \ment{k} \ment{q} \ment{r} \ment{t} \extra{\ment{c}}{\color{red}\footnotemark[28]}, in sono \hyphh{va}{riantur}
\footnotetext[28]{Litery nadpisane Ulanowski czytał \ment{b} i
  \ment{v}. Pierwszą należy raczej czytać \ment{s}. Obie mogą być
  jakimiś skrótami (choć skróceń nie zaznaczono), np. sine variacione
  — bez zmian (występują litery \ment{h}, \ment{k}, \ment{q},
  \ment{r}, \ment{t}). Ostatnie \ment{c} (Ulanowski je czyta
  jako \ment{r}, a \ment{r} poprzednie jako \ment{v}) znalazło
  się tu przez jakąś pomyłkę.}

\splitlines

\hyphh{va}{riantur}. 

\Polish{A więc niemal wszystkie litery — z wyjątkiem \ment{h}, \ment{k},
\ment{q}, \ment{r}, \ment{t} — różnią się w brzmieniu.}

Nam \ment{c} quinqies, \ment{s} sexies sonum variat. 

\Polish{Bowiem
\ment{c} zmienia brzmienie pięciokrotnie, a \ment{s}
sześciokrotnie.}

\ment{V} autem

\fulllines

in quantum vocalis ad modum aliarum vocalium et longum, et

breve est, in quantum autem consonaeciam aliquociens variat sonum.

\Polish{Zaś \textit{v} (\textit{u}) oznaczające samogłoskę
jest — podobnie jak inne samogłoski — bądź długie, bądź krótkie, a
oznaczające spółgłoskę także parokrotnie zmienia brzmienie.}

Nam cum componitur ex consona et vocali, tunc proprie scribitur, sic

\splitlines

\polski{wſta} \polski{wmaar}. 

\Polish{Bo kiedy
się łączy spółgłoska i samogłoska, wtedy właściwą pisownię ma taką jak
\ment{wſta}, \ment{wmaar} (\ment{w usta}, \ment{w
  umiār}).}


Ponitur eciam ut simplex consonans et tunc
\end{document}

%%% Local Variables: 
%%% mode: latex
%%% TeX-PDF-mode: t
%%% TeX-engine: luatex 
%%% TeX-master: "ParkoszLatinPolish"
%%% End: 
