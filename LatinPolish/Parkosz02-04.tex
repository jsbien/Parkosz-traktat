\url{http://wbl.klf.uw.edu.pl/13/2/iParkosz.djvu?djvuopts=&page=44&zoom=width&showposition=0.5,0.18}

\ppageno=2

\fulllines

ipsos sermones in scripto representantes iuxta suos inventores secundum illud

\splitlines
Metriste:

\newtip{10}{Tj. Aleksander de Villa Dei, ur. ok. 1150 r. w Villedieu w
  Normandii, w napisanym heksametrem podręczniku gramatyki
  pt. \textit{Doctrinale puerorum in unum digestum}.}
\Polish{przedstawiające w piśmie ich języki zależnie od ich
  wynalazców, jak to mówi \textit{Metrysta}¹:}

\indentKcyt Invenit hebraicas Habraham patriarcha figuras,

\Polish{Abraham patriarcha wynalazł hebrajskie litery,}

\indentKcyt Sed Catius

\splitlines

\indentKcyt \phantom{Sed Catius}grecas, 

\newtip{11}{Kadmus (w tekście: Catius) — w mitologii greckiej syn Agenora i
Telefany, który miał przenieść z Egiptu czy Fenicji do Grecji alfabet.}
\Polish{Kadmus¹ zaś greckie,}

Carmentis datque latinas.

\newtip{12}{Karmentis lub Karmenta — w mitologii rzymskiej nimfa i
  wieszczka. Wynalazek, o którym tu mowa, przypisywano nie jej, ale
  jej synowi Ewandrowi.} 
\Polish{ a łacińskie wynalazła Karmentis¹.}

\indentK Ex quibus litteris composicionem diccionum

\fulllines

facientes per easdem conceptus ac suas intenciones presentibus manifestant

\splitlines

et posteris ad legendum in scripto relinquunt. 

\Polish{Łącząc te litery w wyrazy, przy ich pomocy swoje myśli i zamiary
pokazują współczesnym, a potomnym pozostawiają
%\marginpar{s. 86}
do czytania na piśmie.}

Et quamvis diversarum nacionum

\fulllines

diversi sunt sermones \add{et diversi caracteres}, in scribendo illos representantes, ut patet in Iohanne de

\splitlines
\footnotetext[14]{W oryginale było prawdopodobnie Montalmo lub Montalma, w kopii
jednak 3 laski dla m nie są równe, słusznie więc Ulanowski czytał: Montalino.}
\conf{Montalino}{}{\color{red}\footnotemark[14]} 

\newtip{13}{Właściwie John de Maundeville, 1300-1372, którego opis podróży do
Palestyny znany był w Polsce jako dzieło Jana de Montalma lub
Montalmo. W rękopisie Biblioteki Jagiellońskiej nr 2392, s. 1-37,
znajduje się \textit{Johannis de Montalma Peregrinatio ad Terram Sanctam anno
1322}.}
\Polish{A chociaż różne narody mają różne języki i
różne znaki wyrażające je w piśmie, jak to widać u Jana de
Montalino¹,}

\fulllines

qui diversarum nacionum forinaliter abecedaria secundum suas

\fulllines

figuras conscripsit, ut Teucrorum, Caldeorum, et sic de aliis, omnia tamen


\splitlines

abecedaria, 

\newtip{14}{Tj. Trojan; chodzi zapewne o pismo arabskie używane
  m.in. przez Turków, których identyfikowano z Teukrami.}
\Polish{który abecadła różnych narodów: Teukrów¹,
Chaldejczyków i innych, spisał w ich własnych kształtach, to jednak
wszystkie abecadła,}

\fulllines

et tria horum comuniora, scilicet Hebraicum, Grecum et Latinum,

licet in suis vocibus differant, ut aliter scribantur et aliter proferantur

\Polish{ w tym również trzy bardziej rozpowszechnione,
mianowicie hebrajskie, greckie i łacińskie, często się wprawdzie
różnią swoimi brzmieniami — inaczej je (tj. litery) bowiem piszą i
inaczej nazywają,}

— quia quedam naciones proferunt et scribunt sua per dicciones, aliquando

per sillabas, ut Greci: \textit{alfa}, \textit{beta} etc., Ruteni: \textit{as} \textit{buky} \textit{ve} \textit{de} \textit{la} \textit{hol} etc.,

\Polish{ gdyż niektóre narody nazywają i piszą swoje (litery)
przy pomocy wyrazów, inne przy pomocy sylab, np. Grecy: \textit{alfa},
\textit{beta} itd., Rusini: \textit{az}, \textit{buki}, \textit{we},
\textit{de}, \textit{hlahol} itd.,}

Latini autem simplicissimi: per voces simplas exprimunt litteras, ut: \textit{a} \textit{b} \textit{c} \textit{d} etc. —

\Polish{ natomiast Łacinnicy wyrażają litery
najprościej, bo przez same głoski, np. \textit{a}, \textit{b},
\textit{c}, \textit{d}, itd. —}

\footnotetext[15]{Łoś poprawia na: Omnes tamen gentes in principalis
  conveniunt expressione elementi.}
omnes tamen gentes in \conf{principali}{} conveniunt \conf{expressionis}{} \conf{elemento}{}{\color{red}\footnotemark[15]}.

\textit{Alpha} enim apud Grecos et \textit{as} apud Rutenos idem est, quod \textit{a} apud

\splitlines

Latinos.

\Polish{ wszystkie jednak narody zgadzają się w
wyrażaniu głównego elementu (dźwiękowego): \textit{alfa} u Greków i
\textit{az} u Rusinów jest to przecież to samo, co \textit{a} u
Łacinników.}

\indentK Quamquam igitur multe naciones diversas sic habent figuras,

\fulllines

suos sermones representantes, nonnulle tamen sunt, que in unius

et in eiusdem idiomatis concordant caracteribus et figuris,


\splitlines

ut: Italici, Francigene, Anglici, Bohemi, Theotoni et ceteris, 

\Polish{Chociaż więc liczne narody mają różne kształty liter wyrażających ich
mowę, to jednak niektóre zgadzają się w używaniu znaków i kształtów
jednego i tego samego wzoru, np. Włosi, Francuzi, Anglicy, Czesi,
Niemcy i inni:}

qui sibi

\fulllines

caracteres et litteras Latini ideomatis usurpant, paucis circa eosdem

\footnotetext[16]{W rkp. wyraz ten jest pisany najczęściej przez y-: ydeoma.}
punctis ad differenciam appositis, ut ex hoc proprium \conf{ideoma}{}{\color{red}\footnotemark[16]} sufficienter

\splitlines

valeant scribere.

\Polish{ wszyscy oni przejęli sposoby (pisma) i litery z języka
łacińskiego, dodawszy do niektórych z nich dla odróżnienia kropki, aby
ich mogli należycie używać do pisania we własnym języku.}

\indentK Propter quam sufficienciam tales gentes, et

\fulllines

precipue nobis Polonis viciniores, videlicet Bohemi et Almani,

omnia sua in foro civili acta, privilegiata et cetera munimenta

in proprio idiomate per latinas litteras scribunt, certas differencias

\splitlines

apponendo, 

\Polish{Dzięki temu narody te, a zwłaszcza sąsiadujące z nami,
Polakami, mianowicie Czesi i Niemcy, piszą wszystkie akty państwowe,
przywileje i inne ważne rzeczy we własnym języku przy pomocy liter
łacińskich, dodawszy niektóre rozróżnienia.}

ut sic quicquid veritatis acte inter ipsos contingat, quod

in scripto ex necessitate reponendum esset, de verbo ad verbum in eodem

ideomate, in quo actum est, omnibus, quibus horum noticiam habere spectat,

\extra{habere} legatur et ne alias propter peregrinam differencium \hyphh{ideo}{matum}

\hypht{ideo}{matum} interpretacionem utpote de Latino in hoc vel in istud idcirco

principalis veritas rei acte suffocetur, quoniam iam hoc multociens compertum

est, sed ut de plano simpliciter, sicut est acta, sic eciam in eodem

\splitlines


ideomate scripta audientibus legatur.

\Polish{ W ten sposób wszystkie
akty prawne i umowy, które między sobą zawarli, a które trzeba
zachować na piśmie, każdy, kogo to dotyczy, może czytać w brzmieniu
autentycznym i 
%\marginpar{s. 87}
dosłownym w tym języku, w którym je zawierano, i prawdy dokonanej rzeczy nie
zniekształca przekład z różniących się od siebie języków, np. z łaciny
na taki lub inny język — a takie zniekształcenia nieraz się zdarzały —
lecz można ją prosto i zrozumiale odczytać w tym samym języku, w
którym akt został dokonany.}


\indentK Unde licet talis defectus

\fulllines

interpretacionis non contingat principaliter propter impericiam \hyphh{inter}{pretancium}

\hypht{inter}{pretancium}, advertendum tamen est, quod ipse quandoque accidit ex necessitate

diversarum proposicionum seu orationum diversorum ideomatum diversis vocibus

ac vocum ordinacionibus eundem conceptum mentis exprimencium, prout


\splitlines

unumquodque consuevit iuxta modum suum. 


\Polish{Trzeba jeszcze zwrócić uwagę, że wspomniane zniekształcenia w
przekładach mogą pochodzić nie tylko z nieudolności tłumaczy. Niekiedy
wynikają one z konieczności narzuconej przez rozmaitość zdań czy
zwrotów w różnych językach, wyrażających jedną i tę samą myśl za
pomocą różnych form i różnych ich układów, zgodnie ze zwykłymi
właściwościami każdego języka.}

Constat enim certi ideomatis

\fulllines

aliquam oracionem esse veram in voce secundum consuetudinem proprie lingue,

que si ad aliam interpretabitur, mox, ut verba exigunt, falsificabitur.

\Polish{ Wiadomo bowiem, że jakieś zdanie czy
wyrażenie w jednym języku może brzmieć prawdziwie, stosownie do
zwyczaju tego właśnie języka, a stanie się od razu fałszywe, jeśli się
je przełoży słowo po słowie.}

\indentK Nam licet Latinus vere dicat: \textit{Cervisia defecatur seu purgatur}, 

% niekonsekwentna pisownia w oryginale
et interpres in Polonico similiter dicet vere: \polski{pijwo ſzyⱥ vſtawa}


\splitlines

% niekonsekwentna pisownia w oryginale
\polski{albo pywo ſzye czyſczy}, in simili tamen nihil valet. 

\Polish{Bo choć Łacinnik powie poprawnie np.: \textit{Cervisia defecatur} albo
\textit{purgatur}, i podobnie tłumacz polski poprawnie powie:
\polski{Piwo się ustawa} albo: \polski{Piwo się czyści}, to jednak w
% sie (u Kucały)????
podobnym (innym) zwrocie taki przekład nie może mieć zastosowania.}

Dicit enim Latinus:

\splitlines

\textit{Panis comeditur}, sue mentis conceptum vere exprimendo. 

\newtip{15}{'Chleb jest jedzony'.}
\Polish{ Bo wprawdzie Łacinnik, wyrażając swoją myśl właściwie, mówi 
\textit{Panis comeditur}¹,}

Et interpres

\fulllines

% zakładam, że to zwykłe g
dicit in Polonico: \polski{Chleb ſzyⱥ gye}, et hoc est falsum, ut liquet de

\splitlines

se. 

\newtip{16}{Chodzi prawdopodobnie o to, że przekład \textit{panis
  comeditur} jako \textit{chleb się je} jest zły z punktu widzenia przede
  wszystkim logicznego: w zwrocie łacińskim \textit{panis} jest podmiotem, w
  zwrocie polskim \textit{chleb} — dopełnieniem (nie je sam siebie, jak piwo
  czyści się samo).}
\Polish{i tłumacz po polsku mówi: \textit{Chleb się je}, jest to
  jednak, jak widać, przekład fałszywy¹.}

Et sic de aliis.

\Polish{Tak samo jest i z innymi zwrotami.}

\indentK Similiter lingua Almanica dicit: \textit{Proicias per domum}

\fulllines

\footnotetext[17]{Tj. über.}
— \almanica{werf heber haus}, significans per hanc preposicionem \almanica{heber}{\color{red}\footnotemark[17]} ex

\splitlines

trinsecum seu conuexum domus. 

\newtip{17}{'Przerzuć przez dom'.}

\Polish{Podobnie w języku niemieckim (łacińskie) \textit{Proicias per
    domum}¹ można przełożyć jako: \almanica{Werf heber Haus}, przy czym
  przyimek \almanica{heber} oznacza 'na zewnątrz’, czyli 'ponad'
  domem. }

Sed cum intendit: per intraneitatem


\splitlines
domus, significare, tunc aliam preponit preposicionem, dicens: 

\almanica{verf durch}


\fulllines

\almanica{haus} 

\Polish{Lecz gdy (tłumacz łacińskie \textit{per}) pojmie jako znaczące
'przez wnętrze’ domu, wówczas użyje innego przyimka i powie:
\almanica{Wer durch Haus},}

\end{document}



%%% Local Variables: 
%%% mode: latex
%%% TeX-PDF-mode: t
%%% TeX-engine: luatex 
%%% TeX-master: "ParkoszLatinPolish"
%%% End: 
