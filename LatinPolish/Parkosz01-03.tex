\url{http://wbl.klf.uw.edu.pl/13/2/iParkosz.djvu?djvuopts=&page=43&zoom=width&showposition=0.5,0.18}

\ppageno=3

% TO DO?:
% http://tex.stackexchange.com/questions/30930/how-to-output-a-counter-with-leading-zeros

%\renewcommand{\theFancyVerbLine}{03-0\arabic{FancyVerbLine}\phantom{a}}

%\begin{VerbatimLatin}
\fulllines

Iesus Christus{\color{red}\footnotemark[1]}
% 1	Napisane u samej góry strony.


\Polish{Jezus Chrystus.}

%\begin{VerbatimLatin}[firstnumber=last]

% psuje numerację nawet bez \relsize???
% numerację psuje również {\colorbox{black!15}{\textbf{P}}}
%\dc{P}ugna pro patria, quia ipsam
\colorbox{black!15}{\textbf{P}}ugna pro patria, quia ipsam

defendere laus est meritoria.{\color{red}\footnotemark[2]}

% TODO:? Transkribus średnik
% 2	Motto przedmowy zostało prawdopodobnie zapożyczone z jakiegoś

% poematu średniowiecznego, gdyż — jak wskazał R. Ganszyniec — wido¬

% czny w nim jest układ lieksametryczny. Wyrażenie „pugna pro patria”

% wywodzi się z przedmowy do szkolnego podręcznika pt. Disticha Catonis.
%\end{VerbatimLatin}


\Polish{Walcz za ojczyznę, ponieważ jej obrona przynosi zasłużoną sławę.}

  
Patria hic notat comunitatem, comunitas diuturnitatem, \hyphh{di}{uturnitas}

\newtip{1}{W nawiasy ujęto niektóre wyrazy i zwroty nie mające odpowiedników w tekście łacińskim.}
\Polish{Ojczyzna oznacza tutaj społeczność, społeczność (zaś)¹ długotrwałość,(a) długotrwałość}


\splitlines



\hypht{di}{uturnitas} eligibilitatem, ut patet tercio \textit{Topicorum}{\color{red}\footnotemark[3]}.


\newtip{2}{Arystotelesa.}
\Polish{to, co (bardziej) pożądane, jak to pokazano w
trzeciej księdze \textit{Topików}¹.
}

 Hinc est, quod
% 3	Aristotelis Topica 3, 1, 116a.

\Polish{Stąd wynika pogląd}

\footnotetext[1]{Napisane u samej góry strony.}

\footnotetext[2]{Motto przedmowy zostało prawdopodobnie zapożyczone
z jakiegoś poematu średniowiecznego, gdyż — jak wskazał R. Ganszyniec
— widoczny w nim jest układ heksametryczny. Wyrażenie „pugna pro
patria” wywodzi się z przedmowy do szkolnego podręcznika pt. Disticha
Catonis.}

\footnotetext[3]{Aristotelis \textit{Topica} 3, 1, 116a.}

\newpage

\splitlines

inconcussa patrum ac philosophorum sanxit autoritas: 

\Polish{uświęcony
niewzruszoną powagą ojców (Kościoła) i filozofów, że}

Sacius reipublice

\splitlines

fore insudandum, quam alias cuilibet privato comodo adherendum,

\Polish{lepiej w pocie
czoła pracować dla rzeczypospolitej niż służyć czyjemukolwiek
pożytkowi prywatnemu.}

\hyphh{sal}{vata}

\splitlines

\hypht{sal}{vata} enim comunitate, salvatur eius pars, et non e contra.

\Polish{Bo jeśli się dobrze ma społeczność, dobrze się
ma (również) jej część, a nie odwrotnie.}

Nec obviat Philosophus in

\splitlines

\textit{Predicamentis} dicens:



\newtip{3}{Arystoteles.}\Polish{Nie przeczy temu Filozof¹ w
\newtip{4}{Tj. \textit{Kategoriach}.}\textit{Predikamentach}¹, mówiąc:}

Non existentibus primis substanciis, id est singularibus, impossibile est aliquid

\fulllines

horum, id est comune, remanere{\color{red}\footnotemark[4]}, quia hoc intelligitur naturaliter vel, ut libet,
\footnotetext[4]{Aristotelis \textit{Cathegoriae}.}

\splitlines

moraliter. 

\Polish{Gdyby nie istniały elementy pierwsze, tj. proste substancje,
  nie mogłoby też istnieć to, co się z nich składa, tj. ogół —
  ponieważ to trzeba rozumieć w znaczeniu naturalnym lub, jeśli ktoś
  woli, moralnym.}

\indentK Unde et Romanis diu consuetum erat triumphantem pro republica

\fulllines

variis honoribus ac multis muneribus preficere, ut patet eorum gesta inspicere

\splitlines

volentibus. 

\Polish{Dlatego też i Rzymianie długo zachowywali zwyczaj wyróżniania
zwycięzcy w walce za ojczyznę różnymi zaszczytami i licznymi darami, o
czym się może przekonać każdy, kto zechce czytać ich dzieje.}

Et hoc ipsum tangit Theodolus in suis \textit{Eglogis} dicens

\newtip{5}{Theodul żyjący w X w. napisał dzieło pt. \textit{Ecloga, qua comparatur miracula Novi Testamenti cum veterum poetarum commentis}.}
\Polish{To samo
ma na myśli Teodolus w swoich 
\textit{Eklogach}¹, mówiąc (ekloga IV):}

\splitlines

\margin{Egloga \conf{IV}{15}}{\color{red}\footnotemark[5]}: 
\footnotetext[5]{IV — tak czyta Ulanowski niezbyt jasny zapis.}

\indentKcyt Excedit laudes hominum, qui primus agones

\indentKcyt Instituit fieri sub vertice

\splitlines

\indentKcyt \phantom{Instituit fieri sub vertice }montis Olimpi.	

\indentKcyt Aurea victrices obnubit laurea crines,

\indentKcyt Ducit pompa domum,

\splitlines

\indentKcyt \phantom{Ducit pompa domum, }sequitur confusio victum.

\Polish{Ponad pochwały ludzkie wzniósł się ten, co pierwszy\\
U stóp góry Olimpu igrzyska stanowił.\\
%\marginpar{s. 84}
Odtąd w laur uwieńczony zwycięzca powracał\\ 
Tryumfalnie do domu, zwyciężonemu zaś wstyd towarzyszył.}

\indentK Hac eciam saluberrima ratione exigente et quasi

\fulllines

ex necesitate inducente nedum apud fideles verumeeiam et circa gentiles, ut

refert Aristotiles, quotienscumque quempiam pro comunitate certantem \hyphh{con}{tingit}

\hypht{con}{tingit} ponere vitam, ipsum honorifice sepultum in suis superstitibus, puta

filiis et filiabus, fovere indesinenter favoribus ratum atque gratum

\splitlines

comunitati extitit. 

\Polish{Ze względu też na tę najbardziej zdrową i niejako z konieczności
wynikającą zasadę był zwyczaj nie tylko u chrześcijan, ale również u
pogan — jak poświadcza Arystoteles — że ilekroć zginął ktoś z
walczących za społeczność, to go społeczność owa zawsze grzebała z
wielkimi zaszczytami, a jego potomków, to jest synów i córki,
obdarzała nieustannie dowodami życzliwości i wdzięczności.}

Claret ergo patriam \conf{defendenti}{} gloriam attingere \conf{militi}{}.{\color{red}\footnotemark[6]}
\footnotetext[6]{Łoś poprawia na: defendentis\ldots militis.}

\Polish{Jest bowiem jasne, że żołnierzowi broniącemu ojczyzny należy
  się sława.}

\fulllines

\indentP Nos itaque monitu talium inducti, non quod intendamus laudi, quoniam hoc ipsum

solius Dei est, sed inspicientes ad nutum, ex quo reipublice agitur comodum,

censuimus ex causis infra dicendis paternum idioma, quod notabiliter traximus

ex Polonorum lingua, fore caracteribus latinis cum paucis differenciis appositis

scribendum, ut in hoc, quantum ad presens attinet, contra cacetem{\color{red}\footnotemark[7]} id est malum
\footnotetext[7]{Wyraz oznaczający błędy językowe, por. u Juvenalisa
  (Sat. VII 51): scribendi cacoetbes.}

et insufficientem usum scripture polonice laborantes videamur patriam comodose

\splitlines

defendere et ad sufficientem modum scribendi inducere.

\newtip{6}{Bardziej odpowiednie byłoby używanie w przekładzie tego
  akapitu pierwszej osoby: ja\ldots uznałem itd. — wprowadzenie tej formy
  sugerował p. prof. J. Domański (recenzent opracowania). Byłoby to
  jednak niemal uznaniem za autora tych słów — samego Parkosza (czego
  zresztą wykluczyć nie można).}
\Polish{Tak więc i my¹, kierując się wspomnianymi napomnieniami — nie dla
% 6 Bardziej odpowiednie byłoby używanie w przekładzie tego akapitu
% pierwszej osoby: ja... uznałem itd. — wprowadzenie tej formy sugerował
% p. prof. J. Domański (recenzent opracowania). Byłoby to jednak niemal
% uznaniem za autora tych słów — samego Parkosza (czego zresztą
% wykluczyć nie można).
zyskania chwały, gdyż ta się należy samemu Bogu, ale przez wzgląd na
zależne od jego woli dobro społeczne — uznaliśmy za słuszne z
przyczyn, o których będzie mowa niżej, narzecze ojczyste, które rzecz
jasna jest językiem Polaków, wyrażać w piśmie literami łacińskimi z
dodaniem niewielu znaków odmiennych.}

\Polish{Chcielibyśmy bowiem, pracując nad
udoskonaleniem teraźniejszego złego i nieodpowiedniego sposobu
pisania, zostać w tym zakresie uznanymi za broniących dobra
powszechnego i wprowadzających odpowiadający potrzebom sposób
pisania.}

\indentK Sed ante

\fulllines

exordium materie intente phas exordium ponere, ut materie presentis necessitatem

audientes eam libencius amplecterentur et odio eidem obviantes audacter

\splitlines

repellerentur. 

\Polish{Lecz zanim się zacznie przedstawiać rzecz samą, należy dodać
wstępne wyjaśnienia, ażeby ci, którzy poznają jej potrzebę, tym
chętniej ją przyjęli i mogli się śmiało przeciwstawić takim, którzy są
jej niechętni.}

Et racio ponetur, cur eadem frui utiliter debeatur.

\Polish{Znajdzie się tu również uzasadnienie, dlaczego z tej
rzeczy winno się owocnie korzystać.}

\indentK Sit ergo propositi

\fulllines

nostri pro themate hoc verbum, quod scribit Plato, divinissimus philosophus, in \textit{Thymeo},

\splitlines

dicens: 

\newtip{7}{Tj. w \textit{Tymeuszu}.}
\Polish{Niech hasłem naszego założenia będą słowa Platona, boskiego filozofa,
(napisane) w \textit{Timajosie}¹:}

Ad hoc datus est nobis sermo, ut presto iudicia mentis nostre fierent{\color{red}\footnotemark[8]}.
\footnotetext[8]{R. Glanszyniec zwrócił uwagę, że ten niedokładny
  cytat z Platona zgadza się z cytowaniem go przez XIII-wiecznego
  gramatyka Goswina de Marbis.}

\Polish{Na to jest nam dana mowa, by sądy naszego umysłu mogły się
  uzewnętrzniać.}

\fulllines

Ubi est advertendum, quia in hoc verbo: iudicia, tangitur humana civilitas,

\splitlines

de qua Aristotiles primo Politicorum dicit: 

\Polish{A trzeba zauważyć, że w słowie „sądy” chodzi o społeczność
  ludzką, o której mówi Arystoteles w pierwszej księdze
  \textit{Polityki}:}

Homo est animal politicum, id est civile, masuetum

\fulllines

et domesticum, suorum conceptuum alteri comunicativum et hoc per signa vocis,

\Polish{Człowiek jest
stworzeniem politycznym, to jest społecznym, oswojonym i domowym,
zdolnym do przekazywania swych pojęć innym, a to przez znaki
głosowe.}

\splitlines

\footnotetext[9]{Peri hermenias.}
ut idem notat primo \conf{Pery}{} \conf{ermenias}{}{\color{red}\footnotemark[9]}, dicens: 

\newtip{8}{Traktatu \textit{O zdaniu}.}
\Polish{To wyraża również w rozdziale pierwszym
\textit{Hermeneutyki}¹, mówiąc:}

Sunt ergo ea, que sunt in voce, earum, que sunt

\splitlines

in anima, passionum note. 

\Polish{ Istnieją więc znaki głosowe wrażeń powstających w duszy.}

Et inferius eciam ibidem innuit, quod voces sunt signa

\splitlines

conceptuum et scripta — vocum, ex quarum vocum composicione sermo fit hominum. 

\Polish{I dalej w tymże rozdziale potwierdza, że dźwięki
mowy są znakami pojęć, a pismo jest znakami dźwięków; z tych dźwięków
składa się mowa ludzka.}

Unde

\fulllines

sermo nihil aliud est, quam instrumentum vocale, quo mentis nostre conceptum

\splitlines

\footnotetext[10]{Cytaty z Arystotelesa w różnych wersjach powtarzają
  się w dziełach średniowiecznych z zakresu gramatyki, np. przytoczony
  cytat w traktacie Generalis doctrina de modis significandi
  grammaticalibus: Ideo dicit Aristoteles primo Peri hermenias: Ea,
  que sunt in voce, sunt note, id est signa, earum passionum, que sunt
  in anima, id est conceptuum seu actuum intellegendi (\so{K. Ganszyniec},
  \textit{Metrificale Marica z Opatowca i traktaty gramatyczne XIV i XV wieku},
  Studia staropolskie pod red. K. Budzyka, t. VI, Wrocław 1960,
  s. 150.}
exprimimus alteri cognoscendum{\color{red}\footnotemark[10]}.

\Polish{Stąd mowa nie jest niczym innym, jak tylko narzędziem
  głosowym, przy pomocy którego pojęcia naszego umysłu dajemy poznać
  innym.}

\indentK  Et talis sermo est multifarius, multiplicitas

\fulllines

\footnotetext[11]{W rkp. błędnie: diversitatis.}
autem sermonis et \conf{diversitas}{}{\color{red}\footnotemark[11]} originaliter processit ex divisione lingue

\splitlines

filiorum Noe, turrim edificancium secundum illud Scripture: 

\Polish{Mowa ludzka jest zróżnicowana, a wielość i zróżnicowanie
  języków wywodzi się z rozdzielenia języka synów Noego budujących
  wieżę, jak to czytamy w Piśmie św.:}

Descendamus

\fulllines

et confundamus linguam filiorum Ne, et non intelligat unusquisque vocem proximi

\splitlines

sui. 

\Polish{Zejdźmy i pomieszajmy język synów Noego, aby żaden nie
  rozumiał głosu swego najbliższego.}

Atque ita divisit eos ex illo loco in universam terram. 

\Polish{I tak rozproszył ich (Bóg) z owego miejsca po całej ziemi.}

Et cessaverunt

\splitlines

edificare civitatem, 

\Polish{I przestali
budować miasto.}

et idcirco vocatum est nomen loci illius Babel, 

\Polish{A owemu miejscu nadano nazwę Babel,}

quia

\splitlines

\footnotetext[12]{\textit{Genesis} XI 7-9. W cytacie są niewielkie różnice w stosunku do
Wulgaty.}
ibi divisum est labium universe terre{\color{red}\footnotemark[12]}. 

\newtip{9}{\textit{Księga Rodzaju} XI, 7-9.}
\Polish{dlatego że tam
został pomieszany język całej ziemi¹.}

Cuius quidem divisi labii, id est sermonis,

\fulllines

diverse naciones diversas habent figuras ac litteras seu caracteres,

\newtip{10}{Tj. Aleksander de Villa Dei, ur. ok. 1150 r. w Villedieu w
  Normandii, w napisanym heksametrem podręczniku gramatyki
  pt. \textit{Doctrinale puerorum in unum digestum}.} 
\Polish{Z powodu
  tego pomieszania języka, tj. mowy, różne narody mają różne postacie
  i litery, czyli znaki 
  % przedstawiające w piśmie ich języki zależnie
  % od ich wynalazców,}
\end{document}









%%% Local Variables: 
%%% mode: latex
%%% TeX-PDF-mode: t
%%% TeX-engine: luatex 
%%% TeX-master: "ParkoszLatinPolish"
%%% End: 
