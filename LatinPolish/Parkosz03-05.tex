\url{http://wbl.klf.uw.edu.pl/13/2/iParkosz.djvu?djvuopts=&page=45&zoom=width&showposition=0.5,0.18}

\ppageno=3

\fulllines

\footnotetext[18]{Tj. ostia.}
ut dato, quod alicuius domus duo hostia sibi opposita sint aperta per intra, que hostia{\color{red}\footnotemark[18]}

\splitlines

cum intendit, ut proiciatur, sic exprimit. 

\Polish{ rozumiejąc w ten sposób, że jakiś dom ma
dwoje
%\marginpar{s. 88}
drzwi naprzeciw siebie otwartych na przestrzał, i myśląc o nich chce
powiedzieć, że trzeba coś przerzucić.}

Hos autem differentes duos conceptus ideomatis

\splitlines

istius uno modo exprimit sermo latinus, dicendo: \textit{Proicias per domum}, 

\Polish{ W języku łacińskim natomiast te
dwa różne zwroty (niemieckie) wyraża się tylko jednym sposobem,
mówiąc: \textit{Proicias per domum},}

non faciens

\splitlines
inter primum et secundum modos exprimendi differentiam.

\newtip{18}{Tj. między jednym a drugim znaczeniem.}
\Polish{ i nie czyniąc różnicy między
jednym a drugim sposobem wyrażania¹}

\indentK Et quamvis bonitas intelligentis 

\fulllines

potest unam et eandem oracionem ad hunc vel ad istum sensum interpretari et per hoc ex

\splitlines

equivocacionis errore ad intencionem veram et principalem eam reducere, 

\Polish{A choć dzięki sprawności rozumu możemy jeden i ten sam zwrot tłumaczyć
sobie w takim lub innym znaczeniu i w ten sposób od wprowadzającej w
błąd ekwiwokacji dochodzić do jego autentycznego i zasadniczego sensu
}

quia sic

\fulllines

scribit beatus Hieronimus in Epistola ad \conf{Pachomnium}{}{\color{red}\footnotemark[19]} de optimo genere interpretandi,
\footnotetext[19]{Błędnie zam.: ad Pammachium. W zbiorze
  \textit{Corpus scriptorum ecclesiasticorum Latinorum}, vol. LIV:
  Sancti Eusebii Hieronymi epistulae, pars I: epistulae I-LXX,
  recensuit I. Hilberg, Vindobonae — Lipsiae 1910, na s. 503-26
  (epistola) \textit{Ad Pammachium de optimo genere interpretandi}.}

\newtip{19}{Ojciec i doktor Kościoła, ok. 347-420, znany głównie
  jako tłumacz Biblii na j. łaciński (tzw. Wulgata).}
\secondtip{20}{W tekście błędnie: ad Pachomium — do Pachomiusza. Pachomiusz,
organizator życia klasztornego w Kościele, zmarł rok przed urodzeniem
się Hieronima. Hieronim tłumaczył jego regułę zakonną, stąd
prawdopodobnie wynikła pomyłka autora czy raczej kopisty tekstu.}
\Polish{— pisze bowiem św. Hieronim¹
w liście do Pammachiusza² o najlepszym
sposobie tłumaczenia,}

diversa diversorum interpretum et autorum inducit testimonia et exempla,

\splitlines

ut Tullii, Terencii, Hilarii, 

\newtip{21}{Tj. Cycerona. Marcus Tullius Cicero, mąż stanu, mówca i filozof rzymski z I w. p.n.e.}
\secondtip{22}{Publius Terentius Afer, komediopisarz rzymski z II w. p.n.e.}
\thirdtip{23}{Św. Hilary z Poitiers, IV w.}
\Polish{ powołując się na różne świadectwa i przykłady
różnych tłumaczy i autorów, jak Tulliusza¹,
Terencjusza²,
Hilarego³,}

que dicunt, quod quilibet gnarus interpres debet

\fulllines

diligenter intendere, ut non ex verbo verbum, sed ex sensu debitum et

\splitlines

aptum transferat sensum{\color{red}\footnotemark[20]}, et Oratii{\color{red}\footnotemark[21]} verbum est: 
\footnotetext[20]{Hieronim, Ad Pammachium (jw., s. 508): Ego enim non solum
fateor, sed libera voce profiteor me in interpretatione Graecorum\ldots
non verbum e verbo, sed sensum exprimere de sensu.}
\footnotetext[21]{Tj. Horatii.}  

\Polish{które pokazują, że każdy znający się na rzeczy tłumacz powinien się
pilnie przykładać do tego, by tłumaczyć nie słowo w słowo, lecz żeby
w tłumaczeniu znaczenie właściwie oddawać; a i Horacy mówi:}

Non ex verbo verbum, sed sensum

\fulllines

ex ensu curabis reddere fidus interpres{\color{red}\footnotemark[22]}, non tamen hoc passim ac comuniter
\footnotetext[22]{U Horacego (\textit{De arte poetica} 133-4): Nec
  verbum verbo curabis reddere fidus interpres, i tak cytuje Horacego
  Hieronim (op. cit., s. 510).}

\splitlines

omnibus facere constat facile, 

\Polish{Chcąc być wiernym tłumaczem, nie będziesz się starał tłumaczyć
  słowo w słowo, lecz sens oddawać — to jednak wiadomo, że nie dla
  wszystkich jest to równie łatwe.}

quia sicut idem beatus Hieronimus ibidem ait:

\Polish{Albowiem, jak tenże św. Hieronim mówi w tym samym miejscu,}


\fulllines
Difficile est alienas lineas (id est litteras{\color{red}\footnotemark[23]}) insequentem non alicubi excedere
\footnotetext[23]{Glosa, u Hieronima brak.}

et arduum, ut, que in aliena lingua bene dicta sunt, eundem decorem

\splitlines

in translacione conservent. 

\Polish{trudna to rzecz i mozolna, idąc cudzym szlakiem, tj. śladem
cudzego pisania, nie wypaść gdzieś z koleiny, aby to, co dobrze
wyrażono w obcym języku, zachowało tę samą ozdobność w
przekładzie.}

Nam quodlibet ideoma suarum vocum diversas habet
\end{document}






\footnotetext[24]{W rkp. raczej: Zoralbicze.}

\renewcommand{\theFancyVerbLine}{\textcolor{green}{05-17\alph{FancyVerbLine}}}
\begin{VerbatimLatin}[firstnumber=1]
proprietates, igitur etc.

\indentK Cautissime itaque nonnulle gentes sua facta, acta,
\end{VerbatimLatin}
\renewcommand{\theFancyVerbLine}{05-\arabic{FancyVerbLine}\phantom{a}}

\begin{VerbatimLatin}[firstnumber=18]
gesta proprio linguagio scriptitant, ne erroris aliquid propter diversa

ideomata incidat. Solum ab hac securitate, solum famosissima
\end{VerbatimLatin}
\renewcommand{\theFancyVerbLine}{\textcolor{green}{05-20\alph{FancyVerbLine}}}
\begin{VerbatimLatin}[firstnumber=1]
Polonorum lingua est aliena.

\indentK Quod videns Spectabilissimus Iacobus
\end{VerbatimLatin}

\renewcommand{\theFancyVerbLine}{05-\arabic{FancyVerbLine}\phantom{a}}

\begin{VerbatimLatin}[firstnumber=21]
Parcosii de \conf{Zorawicze}{}{\color{red}\footnotemark[24]}, decretorum doctor, canonicus Cracoviensis
% 24	w raczej: Zoralbicze.

et rector eclesie parochialis in Skalka, et animadvertens, ne huc

usque per \conf{amplius}{}{\color{red}\footnotemark[25]} nacio Polonica in hoc posterior ac defectuosior aliis
% 25	Łoś poprawia na: per amplior.

(cum multo sui periculo) maneat, edidit unum sufficientissimum modum,

quem in presenti tractatulo tradidit, per quem Polonicum ideoma scribendum

fore sufficienter reliquit, ut per eum nacio consolata sua valeat
\end{VerbatimLatin}

\footnotetext[25]{Łoś poprawia na: per amplior.}

\renewcommand{\theFancyVerbLine}{\textcolor{green}{05-27\alph{FancyVerbLine}}}
\begin{VerbatimLatin}[firstnumber=1]
in proprio ideomate scriptitare gesta.

\indentP Quamvis multis male
\end{VerbatimLatin}
\renewcommand{\theFancyVerbLine}{05-\arabic{FancyVerbLine}\phantom{a}}

\begin{VerbatimLatin}[firstnumber=28]
opinantibus, imo, ut verius dicatur, fantastice somniantibus

ac chimerice, usus materie presentis tractatuli, quo, ut alie gentes,

domini Poloni in suo ideomate proprias scriberent intenciones, videretur \hyphh{occa}{sionem}

\hypht{occa}{sionem} quandam erroris afferre, quam ipsi, pro hac sua parte \hyphh{allegan}{tes}

\hypht{allegan}{tes}, in certis hominibus dicunt evenisse, dantes erroris causam non

aliud esse, quam que ex copia librorum teologicalium et aliorum in comuni

ideomate ipsorum conscripta existunt. Unde secundum sic autumantes

accidit, nedum mares, verum eciam et mulieres varias scripturas legentes
\end{VerbatimLatin}
\renewcommand{\theFancyVerbLine}{\textcolor{green}{05-36\alph{FancyVerbLine}}}
\begin{VerbatimLatin}[firstnumber=1]
exorbitare.

\indentK Absit autem hec paralogismi estimacio, ne modico \hyphh{fer}{menti}
\end{VerbatimLatin}
\renewcommand{\theFancyVerbLine}{05-\arabic{FancyVerbLine}\phantom{a}}

\begin{VerbatimLatin}[firstnumber=37]
\hypht{fer}{menti} presentis masse corrumpatur intencio. Non enim res vel \hyphh{instru}{mentum}

\hypht{instru}{mentum} in se proprie est laudis vel vituperii receptivum, sed tantum eius usus

vel abusus. Vini enim non labes, sed tua, si post vina labes{\color{red}\footnotemark[26]}.
% 26	Ganszyniec wskazał, że jest to parafraza wiersza znanego auto
% rowi przedmowy z glosy do Disticha Catonis: Si post vina labes, non
% vini, sed tua labes.

Ex simili itaque Sacra pagina deberet pretermitti racione, qua ipsi

heretici suas muniunt intenciones, ut patet in tractatu venerabilis

Benedicti, decretorum doctoris, abbatis Marsilie, ubi quemlibet articulum

per Sacram scripturam demonstrant, per hoc autem, quod demonstrant, non \hyphh{redar}{guuntur}

\hypht{redar}{guuntur} heretici, sed quod demonstratam indigeste capiunt, vituperium non

ammittunt. Uti quidem instrumento et bene, et male contingit iuxta utentis

facultatem, quia secundum Innocencium instrumentum non agit per se,

sed coniunctum suo motori, ideo non agunt libere (sc. Sacra pagina?), nisi inquantum coniuncta \hyphh{mo}{ventis}

\hypht{mo}{ventis} voluntati. Namut ait Petrus de Tharantasio, ipsa voluntas
\end{VerbatimLatin}

\footnotetext[26]{Ganszyniec wskazał, że jest to parafraza wiersza
  znanego autorowi przedmowy z glosy do \textit{Disticha Catonis}: Si
  post vina labes, non vini, sed tua labes.}

\renewcommand{\theFancyVerbLine}{\textcolor{green}{05-49\alph{FancyVerbLine}}}
\begin{VerbatimLatin}[firstnumber=1]
est, que claudit oculos et aperit, prout vult.

\indentK Eecedat ergo demens 
\end{VerbatimLatin}


%%% Local Variables: 
%%% mode: latex
%%% TeX-PDF-mode: t
%%% TeX-engine: luatex 
%%% TeX-master: "ParkoszLatinPolish"
%%% End: 
