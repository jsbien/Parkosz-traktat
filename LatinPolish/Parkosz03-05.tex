\url{http://wbl.klf.uw.edu.pl/13/2/iParkosz.djvu?djvuopts=&page=45&zoom=width&showposition=0.5,0.18}

\ppageno=3

\fulllines

\footnotetext[18]{Tj. ostia.}
ut dato, quod alicuius domus duo hostia sibi opposita sint aperta per intra, que hostia{\color{red}\footnotemark[18]}

\splitlines

cum intendit, ut proiciatur, sic exprimit. 

\Polish{ rozumiejąc w ten sposób, że jakiś dom ma
dwoje
%\marginpar{s. 88}
drzwi naprzeciw siebie otwartych na przestrzał, i myśląc o nich chce
powiedzieć, że trzeba coś przerzucić.}

Hos autem differentes duos conceptus ideomatis

\splitlines

istius uno modo exprimit sermo latinus, dicendo: \textit{Proicias per domum}, 

\Polish{ W języku łacińskim natomiast te
dwa różne zwroty (niemieckie) wyraża się tylko jednym sposobem,
mówiąc: \textit{Proicias per domum},}

non faciens

\splitlines
inter primum et secundum modos exprimendi differentiam.

\newtip{18}{Tj. między jednym a drugim znaczeniem.}
\Polish{ i nie czyniąc różnicy między
jednym a drugim sposobem wyrażania¹}

\indentK Et quamvis bonitas intelligentis 

\fulllines

potest unam et eandem oracionem ad hunc vel ad istum sensum interpretari et per hoc ex

\splitlines

equivocacionis errore ad intencionem veram et principalem eam reducere, 

\Polish{A choć dzięki sprawności rozumu możemy jeden i ten sam zwrot tłumaczyć
sobie w takim lub innym znaczeniu i w ten sposób od wprowadzającej w
błąd ekwiwokacji dochodzić do jego autentycznego i zasadniczego sensu
}

quia sic

\fulllines

scribit beatus Hieronimus in Epistola ad \conf{Pachomnium}{}{\color{red}\footnotemark[19]} de optimo genere interpretandi,
\footnotetext[19]{Błędnie zam.: ad Pammachium. W zbiorze
  \textit{Corpus scriptorum ecclesiasticorum Latinorum}, vol. LIV:
  Sancti Eusebii Hieronymi epistulae, pars I: epistulae I-LXX,
  recensuit I. Hilberg, Vindobonae — Lipsiae 1910, na s. 503-26
  (epistola) \textit{Ad Pammachium de optimo genere interpretandi}.}

\newtip{19}{Ojciec i doktor Kościoła, ok. 347-420, znany głównie
  jako tłumacz Biblii na j. łaciński (tzw. Wulgata).}
\secondtip{20}{W tekście błędnie: ad Pachomium — do Pachomiusza. Pachomiusz,
organizator życia klasztornego w Kościele, zmarł rok przed urodzeniem
się Hieronima. Hieronim tłumaczył jego regułę zakonną, stąd
prawdopodobnie wynikła pomyłka autora czy raczej kopisty tekstu.}
\Polish{— pisze bowiem św. Hieronim¹
w liście do Pammachiusza² o najlepszym
sposobie tłumaczenia,}

diversa diversorum interpretum et autorum inducit testimonia et exempla,

\splitlines

ut Tullii, Terencii, Hilarii, 

\newtip{21}{Tj. Cycerona. Marcus Tullius Cicero, mąż stanu, mówca i filozof rzymski z I w. p.n.e.}
\secondtip{22}{Publius Terentius Afer, komediopisarz rzymski z II w. p.n.e.}
\thirdtip{23}{Św. Hilary z Poitiers, IV w.}
\Polish{ powołując się na różne świadectwa i przykłady
różnych tłumaczy i autorów, jak Tulliusza¹,
Terencjusza²,
Hilarego³,}

que dicunt, quod quilibet gnarus interpres debet

\fulllines

diligenter intendere, ut non ex verbo verbum, sed ex sensu debitum et

\splitlines

aptum transferat sensum{\color{red}\footnotemark[20]}, et Oratii{\color{red}\footnotemark[21]} verbum est: 
\footnotetext[20]{Hieronim, Ad Pammachium (jw., s. 508): Ego enim non solum
fateor, sed libera voce profiteor me in interpretatione Graecorum\ldots
non verbum e verbo, sed sensum exprimere de sensu.}
\footnotetext[21]{Tj. Horatii.}  

\Polish{które pokazują, że każdy znający się na rzeczy tłumacz powinien się
pilnie przykładać do tego, by tłumaczyć nie słowo w słowo, lecz żeby
w tłumaczeniu znaczenie właściwie oddawać; a i Horacy mówi:}

Non ex verbo verbum, sed sensum

\fulllines

ex ensu curabis reddere fidus interpres{\color{red}\footnotemark[22]}, non tamen hoc passim ac comuniter
\footnotetext[22]{U Horacego (\textit{De arte poetica} 133-4): Nec
  verbum verbo curabis reddere fidus interpres, i tak cytuje Horacego
  Hieronim (op. cit., s. 510).}

\splitlines

omnibus facere constat facile, 

\Polish{Chcąc być wiernym tłumaczem, nie będziesz się starał tłumaczyć
  słowo w słowo, lecz sens oddawać — to jednak wiadomo, że nie dla
  wszystkich jest to równie łatwe.}

quia sicut idem beatus Hieronimus ibidem ait:

\Polish{Albowiem, jak tenże św. Hieronim mówi w tym samym miejscu,}


\fulllines
Difficile est alienas lineas (id est litteras{\color{red}\footnotemark[23]}) insequentem non alicubi excedere
\footnotetext[23]{Glosa, u Hieronima brak.}

et arduum, ut, que in aliena lingua bene dicta sunt, eundem decorem

\splitlines

in translacione conservent. 

\Polish{trudna to rzecz i mozolna, idąc cudzym szlakiem, tj. śladem
cudzego pisania, nie wypaść gdzieś z koleiny, aby to, co dobrze
wyrażono w obcym języku, zachowało tę samą ozdobność w
przekładzie.}

Nam quodlibet ideoma suarum vocum diversas habet

\splitlines

proprietates, igitur etc.

\Polish{Przecież każdy język ma różne właściwości swoich wyrazów,
a zatem itd.}

\indentK Cautissime itaque nonnulle gentes sua facta, acta,

\fulllines

gesta proprio linguagio scriptitant, ne erroris aliquid propter diversa

\splitlines

ideomata incidat. 

\Polish{Bardzo więc przezornie niektóre narody spisują swoje
czyny, postępki i dokonania we własnym języku, ażeby się (tu) nie
% \marginpar{s. 89}
wkradł jakiś błąd z obcego języka.}

Solum ab hac securitate, solum famosissima

\splitlines

Polonorum lingua est aliena.

\Polish{Taka przezorność jest nieznana
jednemu tylko przesławnemu językowi polskiemu.}

\indentK Quod videns Spectabilissimus Iacobus

\fulllines

Parcosii de \conf{Zorawicze}{}{\color{red}\footnotemark[24]}, decretorum doctor, canonicus Cracoviensis
\footnotetext[24]{W rkp. raczej: Zoralbicze.}

\splitlines

et rector eclesie parochialis in Skalka, 

\newpage
\newtip{24}{Tj. prawa kanonicznego.}
\secondtip{25}{W Krakowie.}
\Polish{Widząc to znakomity Jakub syn Parkosza z Żórawic, doktor dekretów¹,
kanonik krakowski i rektor kościoła parafialnego na Skałce²,}

et animadvertens, ne huc

\fulllines

usque per \conf{amplius}{}{\color{red}\footnotemark[25]} nacio Polonica in hoc posterior ac defectuosior aliis
\footnotetext[25]{Łoś poprawia na: per amplior.}

(cum multo sui periculo) maneat, edidit unum sufficientissimum modum,

\splitlines

quem in presenti tractatulo tradidit, 

\Polish{ w tej
myśli, by naród polski w tych rzeczach nie pozostawał nadal — z
wielkim dla siebie niebezpieczeństwem — w tyle za innymi, wynalazł i
przedstawił w niniejszej rozprawce jeden całkowicie wystarczający
sposób,}

per quem Polonicum ideoma scribendum

\fulllines

fore sufficienter reliquit, ut per eum nacio consolata sua valeat

\splitlines

in proprio ideomate scriptitare gesta.

\Polish{dzięki któremu można będzie zadowalająco pisać po polsku,
ażeby naród pocieszony mógł spisywać swoje dzieje we własnym
języku.}

\indentP Quamvis multis male

\fulllines

opinantibus, imo, ut verius dicatur, fantastice somniantibus

ac chimerice, usus materie presentis tractatuli, quo, ut alie gentes,

domini Poloni in suo ideomate proprias scriberent intenciones, videretur \hyphh{occa}{sionem}

\splitlines
\hypht{occa}{sionem} quandam erroris afferre, 


\Polish{Wprawdzie wielu błędnie myślącym, a właściwiej trzeba
powiedzieć: fantastycznie i chimerycznie bredzącym, wydaje się, że
używanie środka zaleconego w tej rozprawce, zachęcającej, by panowie
Polacy, tak jak inne narody, myśli swoje wyrażali we własnym języku,
może się stać okazją do błędów.}

quam ipsi, pro hac sua parte \hyphh{allegan}{tes}


\fulllines
\hypht{allegan}{tes}, in certis hominibus dicunt evenisse, dantes erroris causam non

aliud esse, quam que ex copia librorum teologicalium et aliorum in comuni

\splitlines

ideomate ipsorum conscripta existunt. 

\Polish{Twierdzą oni mianowicie, że okazja
taka przydarzyła się pewnym ludziom, a przyczynę błędów upatrują nie w
czym innym, lecz w tym właśnie, że pośród wielkiej ilości ksiąg
teologicznych i innych istnieją też księgi napisane w ich własnym
języku.}

Unde secundum sic autumantes

\fulllines

accidit, nedum mares, verum eciam et mulieres varias scripturas legentes

\splitlines

exorbitare.

\Polish{Stąd to — według tak twierdzących — zdarza się, że nie tylko
mężczyźni, ale i kobiety, czytając takie księgi, wpadają w
błędy.}

\indentK Absit autem hec paralogismi estimacio, ne modico \hyphh{fer}{menti}

\splitlines

\hypht{fer}{menti} presentis masse corrumpatur intencio. 

\Polish{Trzymajmy się jednak z daleka od takiego paralogistycznego mniemania,
aby odrobina kwasu nie zepsuła (zdrowej) intencji, z jaką sporządzono
niniejsze ciasto.}

Non enim res vel \hyphh{instru}{mentum}

\fulllines

\hypht{instru}{mentum} in se proprie est laudis vel vituperii receptivum, sed tantum eius usus

\splitlines

vel abusus. 

\Polish{Bo nie rzecz sama ani nie narzędzie zdolne jest
otrzymać pochwałę lub naganę, ale tylko jego właściwe używanie lub
nadużywanie.}

Vini enim non labes, sed tua, si post vina labes{\color{red}\footnotemark[26]}.
\footnotetext[26]{Ganszyniec wskazał, że jest to parafraza wiersza
  znanego autorowi przedmowy z glosy do \textit{Disticha Catonis}: Si
  post vina labes, non vini, sed tua labes.}

\Polish{Jeśliś upadł po nadużyciu wina, nie jest to upadek wina,
ale twój.}

\fulllines

Ex simili itaque Sacra pagina deberet pretermitti racione, qua ipsi

\splitlines

heretici suas muniunt intenciones, 

\Polish{Gdyby tak nie było, toby się trzeba wyrzec i Pisma św.,
ponieważ jego powagą także i heretycy popierają swoje mniemania,
dowodząc każdego artykułu za pomocą Pisma św.,}

ut patet in tractatu venerabilis

\fulllines

Benedicti, decretorum doctoris, abbatis Marsilie, ubi quemlibet articulum

per Sacram scripturam demonstrant, per hoc autem, quod demonstrant, non \hyphh{redar}{guuntur}

\hypht{redar}{guuntur} heretici, sed quod demonstratam indigeste capiunt, vituperium non

\splitlines

ammittunt. 

\newtip{26}{Autora m. in. rozprawy \textit{Tractatus contra diversos
    errores hereticorum} (znajdującego się m.in. w rękopisie
  Biblioteki Jagiellońskiej nr 423, k. 88-111).}
\Polish{jak się okazuje z
rozprawy czcigodnego Benedykta, doktora dekretów, opata z
Marsylii¹. Ale nie dlatego zasługują na naganę, że się powołują na
Pismo św., lecz dlatego, że je opacznie rozumieją.}

Uti quidem instrumento et bene, et male contingit iuxta utentis

\fulllines

facultatem, quia secundum Innocencium instrumentum non agit per se,

\splitlines

sed coniunctum suo motori, 

\newtip{27}{Prawdopodobnie Inocentego III albo Inocentego V (por. następną notkę).}
\Polish{Dobre lub złe 
%\marginpar{s. 90}
używanie narzędzia zależy od zdolności używającego, gdyż — według
Inocentego¹ — narzędzie nie działa samo przez się, ale w połączeniu
z tym, kto je wprawia w ruch.}

ideo non agunt libere (sc. Sacra pagina?), nisi inquantum coniuncta \hyphh{mo}{ventis}

\splitlines

\hypht{mo}{ventis} voluntati. 

\Polish{Znaczy to, że nie działa ono swobodnie,
lecz tylko z woli osoby narzędzie w ruch wprowadzającej.}

Namut ait Petrus de Tharantasio, ipsa voluntas

\splitlines

est, que claudit oculos et aperit, prout vult.

\newtip{28}{Późniejszy papież Inocenty V, 1225-1276. Główne dzieła:
  \textit{Commentaria in 4 libros sententiarum, Commentarii in epistolas
  s. Pauli} (m.in. w kilku rękopisach w Bibliotece Jagiellońskiej).}
\Polish{Bo — jak mówi
Piotr z Tarantasio¹ — to wola właśnie zamyka oczy i otwiera je, jak
chce.}

\indentK Recedat ergo demens 
\end{document}


%%% Local Variables: 
%%% mode: latex
%%% TeX-PDF-mode: t
%%% TeX-engine: luatex 
%%% TeX-master: "ParkoszLatinPolish"
%%% End: 
