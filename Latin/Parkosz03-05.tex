\url{http://wbl.klf.uw.edu.pl/13/2/iParkosz.djvu?djvuopts=&page=45&zoom=width&showposition=0.5,0.18}

\begin{VerbatimLatin}[numbers=none,formatcom=\color{blue}]
Sed cum intendit: per intraneitatem
domus, significare, tunc aliam preponit preposicionem, dicens: \almanica{verf durch}
\almanica{haus} 
\end{VerbatimLatin}
%
\renewcommand{\theFancyVerbLine}{05-0\arabic{FancyVerbLine}\phantom{a}}
%
\begin{VerbatimLatin}
ut dato, quod alicuius domus duo hostia sibi opposita sint aperta per intra, que hostia{\color{red}\footnotemark[18]}
% 18	Tj. ostia.

cum intendit, ut proiciatur, sic exprimit. Hos autem differentes duos conceptus ideomatis

istius uno modo exprimit sermo latinus, dicendo: \textit{Proicias per domum}, non faciens
\end{VerbatimLatin}
%
\footnotetext[18]{Tj. ostia.}
\renewcommand{\theFancyVerbLine}{\textcolor{green}{05-04\alph{FancyVerbLine}}}
\begin{VerbatimLatin}[firstnumber=1]
inter primum et secundum modos exprimendi differentiam.

\indentK Et quamvis bonitas intelligentis 
\end{VerbatimLatin}
\renewcommand{\theFancyVerbLine}{05-0\arabic{FancyVerbLine}\phantom{a}}

\begin{VerbatimLatin}[firstnumber=5]
potest unam et eandem oracionem ad hunc vel ad istum sensum interpretari et per hoc ex

equivocacionis errore ad intencionem veram et principalem eam reducere, quia sic

scribit beatus Hieronimus in Epistola ad \conf{Pachomnium}{}{\color{red}\footnotemark[19]} de optimo genere interpretandi,
% 19	Błędnie zam.: ad Pammachium. W zbiorze Corpus scriptorum
% ecclesiasticorum Latinorum, yol. LIV: Sancti Eusebii Hieronymi epis-
% tulae, pars I: epistulae I-LXX, recensuit I. Hilberg, Vindobonae —
% Lipsiae 1910, na s. 503-26 (epistola) Ad Pammachium de optimo genere
% interpretandi.

diversa diversorum interpretum et autorum inducit testimonia et exempla,

ut Tullii, Terencii, Hilarii, que dicunt, quod quilibet gnarus interpres debet
\end{VerbatimLatin}

\footnotetext[19]{Błędnie zam.: ad Pammachium. W zbiorze
  \textit{Corpus scriptorum ecclesiasticorum Latinorum}, vol. LIV:
  Sancti Eusebii Hieronymi epistulae, pars I: epistulae I-LXX,
  recensuit I. Hilberg, Vindobonae — Lipsiae 1910, na s. 503-26
  (epistola) \textit{Ad Pammachium de optimo genere interpretandi}.}

\renewcommand{\theFancyVerbLine}{05-\arabic{FancyVerbLine}\phantom{a}}

\begin{VerbatimLatin}[firstnumber=10]
diligenter intendere, ut non ex verbo verbum, sed ex sensu debitum et
\end{VerbatimLatin}

% przypis!:
\newpage
\footnotetext[20]{Hieronim, Ad Pammachium (jw., s. 508): Ego enim non solum
fateor, sed libera voce profiteor me in interpretatione Graecorum\ldots
non verbum e verbo, sed sensum exprimere de sensu.}

\begin{VerbatimLatin}[firstnumber=last]

aptum transferat sensum{\color{red}\footnotemark[20]}, et Oratii{\color{red}\footnotemark[21]} verbum est: Non ex verbo verbum, sed sensum
% 20	Hieronim, Ad Pammacliium (jw., s. 508): Ego enim non solum
% fateor, sed libera voce profiteor me in interpretatione Graecorum...
% non verbum e verbo, sed sensum exprimere de sensu.
%
%21	Tj. Horatii.


ex ensu curabis reddere fidus interpres{\color{red}\footnotemark[22]}, non tamen hoc passim ac comuniter
% 22	U Horacego (De arte poetica 133-4): Nec verbum verbo curabis
% reddere fidus interpres, i tak cytuje Horacego Hieronim (op. cit., s. 510).

omnibus facere constat facile, quia sicut idem beatus Hieronimus ibidem ait:

Difficile est alienas lineas (id est litteras{\color{red}\footnotemark[23]}) insequentem non alicubi excedere
% 23	Glosa, u Hieronima brak.

et arduum, ut, que in aliena lingua bene dicta sunt, eundem decorem

in translacione conservent. Nam quodlibet ideoma suarum vocum diversas habet
\end{VerbatimLatin}

\footnotetext[21]{Tj. Horatii.}  

\footnotetext[22]{U Horacego (\textit{De arte poetica} 133-4): Nec
  verbum verbo curabis reddere fidus interpres, i tak cytuje Horacego
  Hieronim (op. cit., s. 510).}

\footnotetext[23]{Glosa, u Hieronima brak.}

\footnotetext[24]{W rkp. raczej: Zoralbicze.}

\renewcommand{\theFancyVerbLine}{\textcolor{green}{05-17\alph{FancyVerbLine}}}
\begin{VerbatimLatin}[firstnumber=1]
proprietates, igitur etc.

\indentK Cautissime itaque nonnulle gentes sua facta, acta,
\end{VerbatimLatin}
\renewcommand{\theFancyVerbLine}{05-\arabic{FancyVerbLine}\phantom{a}}

\begin{VerbatimLatin}[firstnumber=18]
gesta proprio linguagio scriptitant, ne erroris aliquid propter diversa

ideomata incidat. Solum ab hac securitate, solum famosissima
\end{VerbatimLatin}
\renewcommand{\theFancyVerbLine}{\textcolor{green}{05-20\alph{FancyVerbLine}}}
\begin{VerbatimLatin}[firstnumber=1]
Polonorum lingua est aliena.

\indentK Quod videns Spectabilissimus Iacobus
\end{VerbatimLatin}

\renewcommand{\theFancyVerbLine}{05-\arabic{FancyVerbLine}\phantom{a}}

\begin{VerbatimLatin}[firstnumber=21]
Parcosii de \conf{Zorawicze}{}{\color{red}\footnotemark[24]}, decretorum doctor, canonicus Cracoviensis
% 24	w raczej: Zoralbicze.

et rector eclesie parochialis in Skalka, et animadvertens, ne huc

usque per \conf{amplius}{}{\color{red}\footnotemark[25]} nacio Polonica in hoc posterior ac defectuosior aliis
% 25	Łoś poprawia na: per amplior.

(cum multo sui periculo) maneat, edidit unum sufficientissimum modum,

quem in presenti tractatulo tradidit, per quem Polonicum ideoma scribendum

fore sufficienter reliquit, ut per eum nacio consolata sua valeat
\end{VerbatimLatin}

\footnotetext[25]{Łoś poprawia na: per amplior.}

\renewcommand{\theFancyVerbLine}{\textcolor{green}{05-27\alph{FancyVerbLine}}}
\begin{VerbatimLatin}[firstnumber=1]
in proprio ideomate scriptitare gesta.

\indentP Quamvis multis male
\end{VerbatimLatin}
\renewcommand{\theFancyVerbLine}{05-\arabic{FancyVerbLine}\phantom{a}}

\begin{VerbatimLatin}[firstnumber=28]
opinantibus, imo, ut verius dicatur, fantastice somniantibus

ac chimerice, usus materie presentis tractatuli, quo, ut alie gentes,

domini Poloni in suo ideomate proprias scriberent intenciones, videretur \hyphh{occa}{sionem}

\hypht{occa}{sionem} quandam erroris afferre, quam ipsi, pro hac sua parte \hyphh{allegan}{tes}

\hypht{allegan}{tes}, in certis hominibus dicunt evenisse, dantes erroris causam non

aliud esse, quam que ex copia librorum teologicalium et aliorum in comuni

ideomate ipsorum conscripta existunt. Unde secundum sic autumantes

accidit, nedum mares, verum eciam et mulieres varias scripturas legentes
\end{VerbatimLatin}
\renewcommand{\theFancyVerbLine}{\textcolor{green}{05-36\alph{FancyVerbLine}}}
\begin{VerbatimLatin}[firstnumber=1]
exorbitare.

\indentK Absit autem hec paralogismi estimacio, ne modico \hyphh{fer}{menti}
\end{VerbatimLatin}
\renewcommand{\theFancyVerbLine}{05-\arabic{FancyVerbLine}\phantom{a}}

\begin{VerbatimLatin}[firstnumber=37]
\hypht{fer}{menti} presentis masse corrumpatur intencio. Non enim res vel \hyphh{instru}{mentum}

\hypht{instru}{mentum} in se proprie est laudis vel vituperii receptivum, sed tantum eius usus

vel abusus. Vini enim non labes, sed tua, si post vina labes{\color{red}\footnotemark[26]}.
% 26	Ganszyniec wskazał, że jest to parafraza wiersza znanego auto
% rowi przedmowy z glosy do Disticha Catonis: Si post vina labes, non
% vini, sed tua labes.

Ex simili itaque Sacra pagina deberet pretermitti racione, qua ipsi

heretici suas muniunt intenciones, ut patet in tractatu venerabilis

Benedicti, decretorum doctoris, abbatis Marsilie, ubi quemlibet articulum

per Sacram scripturam demonstrant, per hoc autem, quod demonstrant, non \hyphh{redar}{guuntur}

\hypht{redar}{guuntur} heretici, sed quod demonstratam indigeste capiunt, vituperium non

ammittunt. Uti quidem instrumento et bene, et male contingit iuxta utentis

facultatem, quia secundum Innocencium instrumentum non agit per se,

sed coniunctum suo motori, ideo non agunt libere (sc. Sacra pagina?), nisi inquantum coniuncta \hyphh{mo}{ventis}

\hypht{mo}{ventis} voluntati. Namut ait Petrus de Tharantasio, ipsa voluntas
\end{VerbatimLatin}

\footnotetext[26]{Ganszyniec wskazał, że jest to parafraza wiersza
  znanego autorowi przedmowy z glosy do \textit{Disticha Catonis}: Si
  post vina labes, non vini, sed tua labes.}

\renewcommand{\theFancyVerbLine}{\textcolor{green}{05-49\alph{FancyVerbLine}}}
\begin{VerbatimLatin}[firstnumber=1]
est, que claudit oculos et aperit, prout vult.

\indentK Eecedat ergo demens 
\end{VerbatimLatin}


%%% Local Variables: 
%%% mode: latex
%%% TeX-master: t
%%% End: 
