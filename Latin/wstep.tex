  książka prof. Kucały \parencite*{TraktatPWN1985} znajduje się obecnie w
  kilku bibliotekach cyfrowych: Google Books, Hathi Trust Digital
  Library i --- \textit{last but not least} --- e-BUW
  (\url{http://ebuw.uw.edu.pl/publication/220504}). Znajdujące się w
  tych bibliotekach dygitalizacje książki różnią się różnymi aspektami
  technicznymi, których nie będziemy tutaj omawiać.

  
Propozycji transliteracji nie będę prezentował tutaj ze względów
technicznych, zwłaszcza że może ona ulec jeszcze zmianom. Głównym
problemem jest reprezentowanie dwóch rodzajów liter \textit{b} i
\textit{p}. Dla \textit{b z łuczkiem} dość oczywistym wyborem jest
LATIN SMALL/CAPITAL LETTER B WITH HOOK, ale już np. dla \textit{b
  kwadratowego} nie ma dobrego odpowiednika, kwadratowość można
mnemotechnicznie oddać poziomą prostą kreską, stosując LATIN SMALL/CAPITAL
LETTER B WITH STROKE.

% Dla pełności obrazu dodam, że znam pracę \parencite{al.55:_zasad}, ale
% nie traktuję jej bezkrytycznie. 

Aby znak wyświetlić lub wydrukować, potrzebny jest odpowiedni
font. Zaletą transliteracji traktatu Parkosza na znaki już występujące
w standardzie jest możliwość ich wyświetlenia --- przynajmniej
większości z nich --- typowym fontem zainstalowanym standardowo na
komputerze.

Oczywiście, aby możliwie wiernie oddać kształt znaków specyficznych
dla traktatu, potrzebny jest specjalny font. Podjąłem --- mimo braku
specjalistycznej wiedzy --- próbę stworzenia takiego prowizorycznego
fontu, a wyniki udostępniłem w internetowym repozytorium
\url{https://bitbucket.org/jsbien/parkosz-font}. Być może moje próby
zachęcą kompetentnych twórców fontów do poświecenia uwagi potrzebom
edycji dawnych tekstów polskich.

Warunkiem koniecznym stworzenia fontu jest przydzielenie występującym
w nim znakom konkretnych współrzędnych kodowych --- w konsekwencji
repozytorium to zawiera pośrednio aktualną wersję mojej propozycji
transliteracji.

Wspomniany wcześniej obszar użytku prywatnego został wykorzystany
przez środowisko nieformalnego projektu \textit{Medieval Unicode Font
  Initiative} (\url{http://folk.uib.no/hnooh/mufi/} do przypisania
współrzędnych kodowych znakom specyficznym dla tekstów
średniowiecznych. Co więcej, powstało również kilka fontów
uwzględniających te znaki. Dla traktatu Parkosza praktyczne znaczenie
ma tylko jeden znak zdefiniowany przez MUFI, mianowicie COMBINING
ABBREVIATION MARK SUPERSCRIPT UR ROUND R FORM o współrzędnej kodowej
F153.

Równolegle z projektowaniem transliteracji wykonałem korektę
łacińskiego odczytania i polskiego tłumaczenia traktatu. Korekta ta
nie ograniczała się do błędnie rozpoznanych znaków, ale wymagała
również odtworzenia struktury strony --- np. przypisy podzielone na
dwie strony wymagały scalenia i powiązania z odpowiednim fragmentem
tekstu. Uwzględniłem też
sugestie sformułowane w
artykule \parencite{rafał08:_jakub_parkos_polis_mnemon_verse}.


%%% Local Variables:
%%% mode: latex
%%% TeX-master: t
%%% End:
