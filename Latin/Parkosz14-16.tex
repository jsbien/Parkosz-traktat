\ppageno=16
\ppreviouspageno=15
\plineno=39
\psublineno=1

\newParkoszpage

{\relsize{-1}
\url{http://wbl.klf.uw.edu.pl/13/2/iParkosz.djvu?djvuopts=&page=56&zoom=width&showposition=0.5,0.18}

\url{http://wbl.klf.uw.edu.pl/13/2/iParkosz.djvu?djvuopts=&page=80&zoom=width&showposition=0.5,0.81}
}

\bigskip

\obeylines
\mono

\fulllines



% \fullpreviouslines


% {
% \color{blue}

% Tamen 

% eciam in suis locis possemus singulas predictas 
% }

\plineno=0

\fulllines

Adaam biƚ bÿƚ caƚ kaal czas çalo chood 

daaƚ dzaaƚ eſz ffitaa fiɠi i gee ÿe /ɠhaaɲ 

kroɬ ƚis ɬis ɱikaa mika ɲiſſki niſki 

othooſz ƥige piſchno qʋras roſſa rzøøſſa 

roſuuɱ. ſaaɱ ſchaad ſſzaadƚ ſzak \hyphh{zza}{raa}

\newtip{88}{\textit{uuɲ} niejasne, w oryginale mogło być \textit{uuy}, tj. \textit{uj} `wuj'}
\hypht{zza}{raa} Zaɱɲø to umee uuɲ¹ viƚa 
% 88	uuq niejasne> w oryginale mogło być uuy, tj. uj 'wuj\

ʋiɬaaƚ wſta xøøodz ÿąnczøc ÿoczi

ÿøøkaa

\newtip{89}{Lekcja (Ulanowskiego): contur, jest niepewna i niejasna.}
Amen et 3-ius est contur¹
%19	Lekcja (Ulanowskiego): contur, jest niepewna i niejasna.

Quicumque ergo vult sufficienter ac differencialiter

Polonicum idioma scribere, presentes \hyphh{cara}{cteres}

\hyphh{cara}{cteres} abecedarii hoc modo debet annotare, \hyphh{orthogra}{phiam}

\hypht{orthogra}{phiam} quoque hanc consuetudini frequenti \hyphh{incorpo}{rare}

\hypht{incorpo}{rare} ita, ut \textit{a} breve sic, ut prescriptum est, ponat 

et consequenter alias litteras cum earum differenciis ibi 

positis.	

Ad laudem Dei omnipotentis sueque 

genitricis gloriose virginis Marie. Illo iuvante,

qui creavit cuncta ex nichilo, cui et universa 

obediunt, cui manet imperium, qui regnat 

in secula seculorum. Amen.


pysal ſluga geft

\newtip{90}{Odczytanie Ulanowskiego.}
Kaƚſdey ɠodzÿnÿ varzylcowſjkÿ¹. |
% ,0 Odczytanie Ulanowskiego.






\endinput



% \fullfines

% \ppreviouspageno=16
% \plineno=0


% \fullpreviouslines


% {
% \color{blue}

% ???


% }




% \endinput


%%%%%%%%%%%%%%%%%%%%%%%%%%%%%%%%%%%%%%%%%%%%%%%%%%%%%%%%%%%%%%%%%%%%%%%%%%%%%%%%%%%%%%%%%%%



\catcode `\^^M=5

  \newtip{48}{Łoś niesłusznie uważa, że \textit{bika} w obu wypadkach
    napisano błędnie zamiast \textit{ƀyka}. Przykłady są bowiem podane
    w~pisowni dotychczasowej dla pokazania jej niewystarczalności do
    zróżnicowania wyrazów \textit{bika} i \textit{byka}.} 

\obeylines






\newcommand{\margin}[1]{\annotatetextBlue{\{#1\}}{zapisy na marginesie}}


% \renewcommand{\over}[1]{\colorbox{blue!10}{\{#1\}}}

\renewcommand{\over}[1]{\annotatetextBlue{\{#1\}}{zapisy nad rządkami}}

% litery i wyrazy dodane, (których w tekście brak)
%\newcommand{\add}[1]{\colorbox{olive!10}{<#1>}}
\newcommand{\add}[1]{\annotatetextOlive{<#1>}{litery i wyrazy dodane, (których w tekście brak)}}

% litery i wyrazy zbędne
% \newcommand{\extra}[1]{\colorbox{magenta!10}{[#1]}}
\newcommand{\extra}[1]{\colorbox{magenta!10}{[#1]}}

% przekreślenia
% MATHEMATICAL LEFT WHITE SQUARE BRACKET' (U+27E6)
% 'MATHEMATICAL RIGHT WHITE SQUARE BRACKET' (U+27E7)
\newcommand{\overstr}[1]{\annotatetextMagenta{⟦#1⟧}{przekreślenia}}

Co to jest:

http://cyfroteka.pl/catalog/ebooki/14307/36045/ff/101/OEBPS/Text/JAKOBA_SYNA_PARKOSZOWEGO_split_005.xhtml

V. WYDANIE BANDTKIEGO.

 

Traktat Parkosza po raz pierwszy został wydany w druku przez S. Bandtkiego w Poznaniu w r. 1839 p. t. "Jacobi Parcossii de Żorawice antiquissimus de orthographia polonica libellus, rogatu et sumptibus Eduardi com. Raczyński, opera et studio Georgii Samuelis Bandtkie editus". Wydanie to jednak obfituje w liczne błędy, pochodzące z fałszywego odczytania rękopisu. Tak np. czytamy na str. 17: "Satius vero pro factione insudandum" zam. "Sacius reipublice fore insudandum"; na str. 18: "non existentibus primis subjectis et singularibus nuspiam locus est" zam. "non esistentibus primis substanciis. id est singularibus, impossibile est"; na str. 19: "Aurea victores obnubit laurea cives" zam. "Aurea victrices obnubit laurea crines"; na str. 23: "divisae nationes divisas habent figuras" zam. "diverse naciones diversas habent figuras"; na str. 27: "errorum diversorum idiotarum" zam. "oracionum diversorum ideomatum"; na str. 28: "hoc est perinde, ut liquet de se et de aliis" zam. "hoc est falsum, ut liquet, et sic de aliis"; na str. 40: "a Latinis imitamus" zam. "a Latinis mutuavimus"; na str. 41: "vocalis producta enunciabitur" zam. "vocalis producta geminabitur"; na str. tejże: "ubi rectius breviationis et productionis" zam. "ubi ex ejus breviacione et produccione"; na str. 42: "V autem antiquitus vocalis" zam. "V autem in quantum vocalis"; na str. 44: "periculosum" zam. "peecatum"; na str. 45: "ubi ij longatur, debet juste molliri" zam. "ubi y longatur merito b deberet grossari et ubi i breviatur deberet juste molliri"; na str. 60: "lanea tunica" zam. wyrazów "hynca, kunca". Od takich błędów roi się w wydaniu Bandtkiego na każdej prawie stronicy. Na usprawiedliwienie pierwszego wydawcy należy przytoczyć okoliczność, że oryginał jest bardzo trudny do odczytania, ponieważ użyto w nim mnóstwa skróceń, z któremi tylko bardzo wprawny i doświadczony paleograf poradzić sobie zdoła. Uznając konieczność drugiego, poprawnego wydania książeczki Parkosza, a nie mając wprawy w czytaniu takich właśnie rękopisów średniowiecznych, musiałbym wyrzec się tego zamiaru bez łaskawej pomocy prof. dra Bolesława Ulanowskiego, który mi poprostu cały tekst rękopisu podyktował. Dzięki temu traktat ortograficzny Parkosza wychodzi obecnie z druku w postaci, wiernie oddającej właściwości rękopisu. Jeżeli zaś mimo to znajdują się w nim jeszcze miejsca ciemne lub błędne, niestety dość liczne, przyczyną tego jest okoliczność, że mamy do czynienia z tekstem, napisanym nie przez samego reformatora naszej pisowni, ale przez dość niedołężnego kopistę, który przytem również prawdopodobnie nie z oryginału przepisywał. Wskazuje na to wstęp, gdzie jest mowa o Jakóbie synu Parkosza, z pewnością nie przez niego samego napisany, a również niewątpliwie nie skomponowany i przez ostatniego kopistę. Ten zapewne przepisywał już kopię traktatu Parkoszowego w ów wstęp zaopatrzoną; powtórzył też błędy poprzednich kopistów i dodał do nich swoje. Są one tak wielkie, że niezawsze poprawić się dają, a tak liczne, że niepodobna ich poprawek wprowadzać do tekstu, umieszczam, więc je osobno pod numerowanemi odnośnikami razem z innemi uwagami.

Łoś???

V. WYDANIE BANDTKIEGO.

 

Traktat Parkosza po raz pierwszy został wydany w druku przez S. Bandtkiego w Poznaniu w r. 1839 p. t. "Jacobi Parcossii de Żorawice antiquissimus de orthographia polonica libellus, rogatu et sumptibus Eduardi com. Raczyński, opera et studio Georgii Samuelis Bandtkie editus". Wydanie to jednak obfituje w liczne błędy, pochodzące z fałszywego odczytania rękopisu. Tak np. czytamy na str. 17: "Satius vero pro factione insudandum" zam. "Sacius reipublice fore insudandum"; na str. 18: "non existentibus primis subjectis et singularibus nuspiam locus est" zam. "non esistentibus primis substanciis. id est singularibus, impossibile est"; na str. 19: "Aurea victores obnubit laurea cives" zam. "Aurea victrices obnubit laurea crines"; na str. 23: "divisae nationes divisas habent figuras" zam. "diverse naciones diversas habent figuras"; na str. 27: "errorum diversorum idiotarum" zam. "oracionum diversorum ideomatum"; na str. 28: "hoc est perinde, ut liquet de se et de aliis" zam. "hoc est falsum, ut liquet, et sic de aliis"; na str. 40: "a Latinis imitamus" zam. "a Latinis mutuavimus"; na str. 41: "vocalis producta enunciabitur" zam. "vocalis producta geminabitur"; na str. tejże: "ubi rectius breviationis et productionis" zam. "ubi ex ejus breviacione et produccione"; na str. 42: "V autem antiquitus vocalis" zam. "V autem in quantum vocalis"; na str. 44: "periculosum" zam. "peecatum"; na str. 45: "ubi ij longatur, debet juste molliri" zam. "ubi y longatur merito b deberet grossari et ubi i breviatur deberet juste molliri"; na str. 60: "lanea tunica" zam. wyrazów "hynca, kunca". Od takich błędów roi się w wydaniu Bandtkiego na każdej prawie stronicy. Na usprawiedliwienie pierwszego wydawcy należy przytoczyć okoliczność, że oryginał jest bardzo trudny do odczytania, ponieważ użyto w nim mnóstwa skróceń, z któremi tylko bardzo wprawny i doświadczony paleograf poradzić sobie zdoła. Uznając konieczność drugiego, poprawnego wydania książeczki Parkosza, a nie mając wprawy w czytaniu takich właśnie rękopisów średniowiecznych, musiałbym wyrzec się tego zamiaru bez łaskawej pomocy prof. dra Bolesława Ulanowskiego, który mi poprostu cały tekst rękopisu podyktował. Dzięki temu traktat ortograficzny Parkosza wychodzi obecnie z druku w postaci, wiernie oddającej właściwości rękopisu. Jeżeli zaś mimo to znajdują się w nim jeszcze miejsca ciemne lub błędne, niestety dość liczne, przyczyną tego jest okoliczność, że mamy do czynienia z tekstem, napisanym nie przez samego reformatora naszej pisowni, ale przez dość niedołężnego kopistę, który przytem również prawdopodobnie nie z oryginału przepisywał. Wskazuje na to wstęp, gdzie jest mowa o Jakóbie synu Parkosza, z pewnością nie przez niego samego napisany, a również niewątpliwie nie skomponowany i przez ostatniego kopistę. Ten zapewne przepisywał już kopię traktatu Parkoszowego w ów wstęp zaopatrzoną; powtórzył też błędy poprzednich kopistów i dodał do nich swoje. Są one tak wielkie, że niezawsze poprawić się dają, a tak liczne, że niepodobna ich poprawek wprowadzać do tekstu, umieszczam, więc je osobno pod numerowanemi odnośnikami razem z innemi uwagami.

http://cyfroteka.pl/ebooki/Jakoba_Syna_Parkoszowego-ebook/p14307i36045

<a href="http://cyfroteka.pl/ebooki/Jakoba_Syna_Parkoszowego-ebookRO/p14307i36045" target="_blank" title="Jakóba Syna Parkoszowego [Jan Łoś]  - KLIKAJ I CZYTAJ ONLINE" > <img src="http://cyfroteka.pl/images/BRD.png" style="border:none;background:none transparent;box-shadow:none;-webkit-box-shadow:none;-webkit-border-radius:0;border-radius:0;" alt="Jakóba Syna Parkoszowego [Jan Łoś]  - KLIKAJ I CZYTAJ ONLINE"/></a>

http://www.nexto.pl/ebooki/jakoba_syna_parkoszowego_p14307.xml

http://fbc.pionier.net.pl/id/oai:www.pbi.edu.pl:PBI_2011_05_24:ea41b5c0-8a06-44af-a29e-4fb6ddbbca85

%%% Local Variables: 
%%% mode: latex
%%% TeX-PDF-mode: t
%%% TeX-engine: luatex 
%%% TeX-master: "ParkoszLatin"
%%% default-input-method: "Parkosz-slash"
%%% End: 
