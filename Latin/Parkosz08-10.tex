\ppageno=10
\ppreviouspageno=9
\plineno=47
\psublineno=1

\newParkoszpage


{\relsize{-3}
%\url{http://ebuw.uw.edu.pl/Content/215606/directory.djvu?djvuopts=&page=50&zoom=width&showposition=0.5,0.18}
  \url{https://jsbien.github.io/Parkosz4IIIF/\#?cv=7&manifest=https://jsbien.github.io/Parkosz4IIIF/collection/ParkoszJBC/index.json}
  
\url{http://ebuw.uw.edu.pl/Content/215606/directory.djvu?djvuopts=&page=71&zoom=width&showposition=0.5,0.68}
}

\bigskip

\obeylines
\mono


\fullpreviouslines


{
\color{blue}
Non est facile 

assignare, tamen, si placet, fiet talis differencia — si placet,

}

\plineno=0

\fulllines
quia omnes voces et omnes caracteres ad placitum sunt inventoris

\splitlines
et sequencium.

\indentK In \parkosz{c} igitur grosse prolato seu asperato, ut in

\fulllines

exemplis primi modi, ut \parkosz{czas} \parkosz{czeġo} \parkosz{czyṃ} \parkosz{czoɬkaa} \parkosz{czubaacz}

\newtip{60}{W rkp.: quantocius.}
scribitur \parkosz{c} cum \parkosz{z}, ut solitum est scribi. Hoc enim \conf{tantocius}{}¹ \hyphh{ample}{ctemur}
% eo \y rkp.: quantocius.

\splitlines

\hypht{ample}{ctemur}, quanto quotidiano usu approbaverimus.

\indentK In grossius autem

\fulllines

prolatis, ut in exemplis secundi modi de \parkosz{a} et de \parkosz{o}, facilem

poterimus differenciam ponere, ut si scribamus \parkosz{cz} et \parkosz{y}, ut \parkosz{czyalo}

\parkosz{czijaſɲo}, \parkosz{czijjeṃø} \parkosz{czijjeḷø} \parkosz{czijoḷek} \parkosz{czijvḷaa} \parkosz{czijⱥġṇe},

\newtip{61}{W rkp.: esse.}
in his quidem diccionibus satis idonea \conf{esset}{}¹ expressio et differencia.
% 41	W rkp.: esse.

Sed in aliis nullo modo hec differencia sufficeret, scilicet quando occurerunt exempla

de \parkosz{I}: \parkosz{czijṃ}, \parkosz{czijɲ}, quod est prime asperacionis exemplum. Quomodo

autem scriberemus \parkosz{czyṇ} cum 2-a asperacione, quod significat sillabam de

illis diccionibus: \parkosz{bozaczÿṇ} \parkosz{prødoczÿṇ} \parkosz{Coczÿṇ} vel de similibus ?

Nescio enim, quomodo dicciones in scripto different \parkosz{czyɱ} prime \hyphh{aspe}{racionis}

\hypht{aspe}{racionis}, id est cum quo, et \parkosz{czijṇ} secunde asperacionis, que est sillaba de

predictis diccionibus sicut \parkosz{bozⱥczÿṇ} etc. Item \parkosz{czÿrpaal} id est hausit et

\parkosz{czÿrpal} id est passus est, saltem circa primam sillabam, de qua est sermo.

Different enim ista circa \parkosz{p} spissum et molle. Sic eciam \parkosz{ſczÿijrpaa}

\splitlines

\newtip{62}{Prawdopodobnie błąd zamiast \parkosz{fczyrpaa}.}
id est obstupet et \conf{\parkosz{ſczÿrkaa}}{}¹, et multa talia.
% 42	Prawdopodobnie błąd zamiast fczyrpaa.

\indentK Unde, si placeret,

\fulllines

quociescunque occurrerit illa grossior asperacio, \extra{quod} loco \parkosz{cz} et

\newtip{63}{W rkp.: duplex.}
\conf{duplicis}{}¹ \parkosz{ÿ} scribemus simplex \parkosz{c} cum uno tractu in parte
% <3 'w rkp_. duplex.

inferiore, quod antiqui loco \parkosz{z} ponebant, ut apparet

in antiquis libris, ut sic: \colorbox{olive}{ç {\Quivira ꝣ}}, et hac littera uteremur

semper, quocienscumque occurrerent nobis \parkosz{c} \parkosz{z} et \parkosz{I} suffocatum, id est \hyphh{grossi}{oris}

\hypht{grossi}{oris} asperacionis, ut in predictis exemplis: \parkosz{çalo} \parkosz{çaſɲo}

\overstr{c} \parkosz{çeṃø} \parkosz{çoḷek} \parkosz{çuḷa} \parkosz{çøġṇe}. Et hoc modo inter \parkosz{czyṃ}

et \parkosz{czyɲ} prime asperacionis et \parkosz{chçeeij} id est velis, \parkosz{çijrpaal}

\parkosz{ſçijrpaaḷ} et \parkosz{çypaaḷ} notabilis esset differencia. In molliter

autem asperatis scribatur simplex \parkosz{c} sine omni asperacione, ut

\parkosz{cap} \parkosz{ceɓuɬa} \parkosz{ciſz} id est caix, de \parkosz{o}: \parkosz{co} id est quid, \parkosz{cuudṇi}

\splitlines

id est pulcer.

\indentK Alie autem omnes dicciones a \parkosz{c} incipientes vel

\fulllines

\newtip{64}{W rkp.: \parkosz{v}.}
ubicunque post \parkosz{c} subsequitur \parkosz{a} \parkosz{o} \parkosz{u}¹ et \parkosz{ø} scribantur per
% 44	W rkp.: v.

\parkosz{k}. Exemplum primi ut \parkosz{kath} \parkosz{kaṃeeṇ} \parkosz{koth} \parkosz{kooth} \parkosz{kuurcz}

\splitlines

\parkosz{kuur}. Exemplum secundi: \parkosz{kɲaaᵽ} \parkosz{kṃotr} \parkosz{kroɬ} \parkosz{ktho}.

\indentK Et

\fulllines

idem possemus cum \parkosz{q}, ut loco eius \parkosz{k} scriberemus, ut \parkosz{kvap}

\parkosz{kvath} \parkosz{kveli} \parkosz{kvikaa}. Nam eciam apud Latinos \parkosz{q} in

\extra{in} aliquibus superfluit, excepto \parkosz{quicungue} et \parkosz{cuicunque}, que 

non possunt debite figurari eodem modo, scilicet per \parkosz{k}, tamen non in \parkosz{q}

nec in \parkosz{c}, sed in v consonante et vocali est vis differencie in

illis duabus diccionibus: \parkosz{quicunque} et \parkosz{cuicunque}, vel \parkosz{qui}

\newtip{65}{W rkp.: \parkosz{v}.}
et \parkosz{cui}. Unde si vellent Latini nostra differencia inter \conf{\parkosz{u}}{}¹
% 65 rkp.: v.

vocalem et \parkosz{v} consonantem uti, non indigerent \parkosz{q}, quia

loco eius sufficeret \parkosz{c}, prout eo ipsi utuntur. Loco

\splitlines

cuius iam nos ponemus \parkosz{k}.

\indentK De \parkosz{ch} autem asperato

\fulllines

nulla est difficultas, ut \parkosz{chleƀ} \parkosz{chṃeeɬ} \parkosz{chaarth}
\parkosz{chroſt}, quia antiquam retinet figuracionem, scilicet per \parkosz{c} et \parkosz{h}.


\endinput


%%%%%%%%%%%%%%%%%%%%%%%%%%%%%%%%%%%%%%%%%%%%%%%%%%%%%%%%%%%%%%%%%%%%%%%%%%%%%%%%%%%%%%%%%%%
60:








\catcode `\^^M=5

  \newtip{48}{Łoś niesłusznie uważa, że \textit{bika} w obu wypadkach
    napisano błędnie zamiast \textit{ƀyka}. Przykłady są bowiem podane
    w~pisowni dotychczasowej dla pokazania jej niewystarczalności do
    zróżnicowania wyrazów \textit{bika} i \textit{byka}.} 

\obeylines






\newcommand{\margin}[1]{\annotatetextBlue{\{#1\}}{zapisy na marginesie}}


% \renewcommand{\over}[1]{\colorbox{blue!10}{\{#1\}}}

\renewcommand{\over}[1]{\annotatetextBlue{\{#1\}}{zapisy nad rządkami}}

% litery i wyrazy dodane, (których w tekście brak)
%\newcommand{\add}[1]{\colorbox{olive!10}{<#1>}}
\newcommand{\add}[1]{\annotatetextOlive{<#1>}{litery i wyrazy dodane, (których w tekście brak)}}

% litery i wyrazy zbędne
% \newcommand{\extra}[1]{\colorbox{magenta!10}{[#1]}}
\newcommand{\extra}[1]{\colorbox{magenta!10}{[#1]}}

% przekreślenia
% MATHEMATICAL LEFT WHITE SQUARE BRACKET' (U+27E6)
% 'MATHEMATICAL RIGHT WHITE SQUARE BRACKET' (U+27E7)
\newcommand{\overstr}[1]{\annotatetextMagenta{⟦#1⟧}{przekreślenia}}



%%% Local Variables: 
%%% mode: latex
%%% TeX-PDF-mode: t
%%% TeX-engine: luatex 
%%% TeX-master: "ParkoszLatin"
%%% default-input-method: "Parkosz-slash"
%%% End: 
