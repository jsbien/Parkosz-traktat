\url{http://wbl.klf.uw.edu.pl/13/2/iParkosz.djvu?djvuopts=&page=44&zoom=width&showposition=0.5,0.18}

\footnotetext[14]{W oryginale było prawdopodobnie Montalmo lub Montalma, w kopii
jednak 3 laski dla m nie są równe, słusznie więc Ulanowski czytał: Montalino.}

\begin{VerbatimLatin}[numbers=none,formatcom=\color{blue}]
Cuius quidem divisi labii, id est sermonis,
% 12	Genesis XI 7-9. W cytacie są niewielkie różnice w stosunku do
% Wulgaty.
diverse naciones diversas habent figuras ac litteras seu caracteres,
\end{VerbatimLatin}

\renewcommand{\theFancyVerbLine}{04-0\arabic{FancyVerbLine}\phantom{a}}

\begin{VerbatimLatin}
ipsos sermones in scripto representantes iuxta suos inventores secundum illud
\end{VerbatimLatin}
\renewcommand{\theFancyVerbLine}{\textcolor{green}{04-11\alph{FancyVerbLine}}}
\begin{VerbatimLatin}[firstnumber=1]
Metriste:

\indentKcyt Invenit hebraicas Habraham patriarcha figuras,

\indentKcyt Sed Catius
\end{VerbatimLatin}
\renewcommand{\theFancyVerbLine}{\textcolor{green}{04-12\alph{FancyVerbLine}}}
\begin{VerbatimLatin}[firstnumber=1]
\indentKcyt \phantom{Sed Catius}grecas, Carmentis datque latinas.

\indentK Ex quibus litteris composicionem diccionum
\end{VerbatimLatin}

\renewcommand{\theFancyVerbLine}{04-\arabic{FancyVerbLine}\phantom{a}}

\begin{VerbatimLatin}[firstnumber=13]
facientes per easdem conceptus ac suas intenciones presentibus manifestant

et posteris ad legendum in scripto relinquunt. Et quamvis diversarum nacionum

diversi sunt sermones \add{et diversi caracteres}, in scribendo illos representantes, ut patet in Iohanne de

\conf{Montalino}{}{\color{red}\footnotemark[14]} qui diversarum nacionum forinaliter abecedaria secundum suas
% 14	W oryginale było prawdopodobnie Montalmo lub Montalma, w kopii
% jednak 3 laski dla m nie są równe, słusznie więc Ulanowski czytał: Mon

% talino.

figuras conscripsit, ut Teucrorum, Caldeorum, et sic de aliis, omnia tamen

abecedaria, et tria horum comuniora, scilicet Hebraicum, Grecum et Latinum,

licet in suis vocibus differant, ut aliter scribantur et aliter proferantur

— quia quedam naciones proferunt et scribunt sua per dicciones, aliquando

per sillabas, ut Greci: \textit{alfa}, \textit{beta} etc., Ruteni: \textit{as} \textit{buky} \textit{ve} \textit{de} \textit{la} \textit{hol} etc.,

Latini autem simplicissimi: per voces simplas exprimunt litteras, ut: \textit{a} \textit{b} \textit{c} \textit{d} etc. —

omnes tamen gentes in \conf{principali}{} conveniunt \conf{expressionis}{} \conf{elemento}{}{\color{red}\footnotemark[15]}.
% 15	Łoś poprawia na: Omnes tamen gentes in principalis conveniunt
% expressione elementi.

\textit{Alpha} enim apud Grecos et \textit{as} apud Rutenos idem est, quod \textit{a} apud
\end{VerbatimLatin}

\footnotetext[15]{Łoś poprawia na: Omnes tamen gentes in principalis
  conveniunt expressione elementi.}

\renewcommand{\theFancyVerbLine}{\textcolor{green}{04-25\alph{FancyVerbLine}}}
\begin{VerbatimLatin}[firstnumber=1]
Latinos.

\indentK Quamquam igitur multe naciones diversas sic habent figuras,
\end{VerbatimLatin}

\renewcommand{\theFancyVerbLine}{04-\arabic{FancyVerbLine}\phantom{a}}

\begin{VerbatimLatin}[firstnumber=26]
suos sermones representantes, nonnulle tamen sunt, que in unius

et in eiusdem idiomatis concordant caracteribus et figuris,

ut: Italici, Francigene, Anglici, Bohemi, Theotoni et ceteris, qui sibi

caracteres et litteras Latini ideomatis usurpant, paucis circa eosdem

punctis ad differenciam appositis, ut ex hoc proprium \conf{ideoma}{}{\color{red}\footnotemark[16]} sufficienter
%16	W rkp. wyraz ten jest pisany najczęściej przez y-: ydeoma.
\end{VerbatimLatin}

\footnotetext[16]{W rkp. wyraz ten jest pisany najczęściej przez y-: ydeoma.}

\renewcommand{\theFancyVerbLine}{\textcolor{green}{04-31\alph{FancyVerbLine}}}
\begin{VerbatimLatin}[firstnumber=1]
valeant scribere.

\indentK Propter quam sufficienciam tales gentes, et
\end{VerbatimLatin}

\renewcommand{\theFancyVerbLine}{04-\arabic{FancyVerbLine}\phantom{a}}

\begin{VerbatimLatin}[firstnumber=32]
precipue nobis Polonis viciniores, videlicet Bohemi et Almani,

omnia sua in foro civili acta, privilegiata et cetera munimenta

in proprio idiomate per latinas litteras scribunt, certas differencias

apponendo, ut sic quicquid veritatis acte inter ipsos contingat, quod

in scripto ex necessitate reponendum esset, de verbo ad verbum in eodem

ideomate, in quo actum est, omnibus, quibus horum noticiam habere spectat,

\extra{habere} legatur et ne alias propter peregrinam differencium \hyphh{ideo}{matum}

\hypht{ideo}{matum} interpretacionem utpote de Latino in hoc vel in istud idcirco

principalis veritas rei acte suffocetur, quoniam iam hoc multociens compertum

est, sed ut de plano simpliciter, sicut est acta, sic eciam in eodem
\end{VerbatimLatin}
\renewcommand{\theFancyVerbLine}{\textcolor{green}{04-42\alph{FancyVerbLine}}}
\begin{VerbatimLatin}[firstnumber=1]
ideomate scripta audientibus legatur.

\indentK Unde licet talis defectus
\end{VerbatimLatin}
\renewcommand{\theFancyVerbLine}{04-\arabic{FancyVerbLine}\phantom{a}}

\begin{VerbatimLatin}[firstnumber=43]
interpretacionis non contingat principaliter propter impericiam \hyphh{inter}{pretancium}

\hypht{inter}{pretancium}, advertendum tamen est, quod ipse quandoque accidit ex necessitate

diversarum proposicionum seu orationum diversorum ideomatum diversis vocibus

ac vocum ordinacionibus eundem conceptum mentis exprimencium, prout

unumquodque consuevit iuxta modum suum. Constat enim certi ideomatis

aliquam oracionem esse veram in voce secundum consuetudinem proprie lingue,

que si ad aliam interpretabitur, mox, ut verba exigunt, falsificabitur.

\indentK Nam licet Latinus vere dicat: \textit{Cervisia defecatur seu purgatur}, 

et interpres in Polonico similiter dicet vere: \parkosz{pijwo} \parkosz{ſzyⱥ} \parkosz{vſtawa}

\parkosz{albo} \parkosz{pywo} \parkosz{ſzye} \parkosz{czyſczy}, in simili tamen nihil valet. Dicit enim Latinus:

\textit{Panis comeditur}, sue mentis conceptum vere exprimendo. Et interpres

dicit in Polonico: \parkosz{Chleb} \parkosz{ſzyⱥ} \parkosz{ġye}, et hoc est falsum, ut liquet de
\end{VerbatimLatin}

\renewcommand{\theFancyVerbLine}{\textcolor{green}{04-55\alph{FancyVerbLine}}}
\begin{VerbatimLatin}[firstnumber=1]
se. Et sic de aliis.

\indentK Similiter lingua Almanica dicit: \textit{Proicias per domum}
\end{VerbatimLatin}

\renewcommand{\theFancyVerbLine}{04-\arabic{FancyVerbLine}\phantom{a}}

\begin{VerbatimLatin}[firstnumber=56]
— \almanica{werf heber haus}, significans per hanc preposicionem \almanica{heber}{\color{red}\footnotemark[17]} ex
% 17	Tj. über.

trinsecum seu conuexum domus. Sed cum intendit: per intraneitatem

domus, significare, tunc aliam preponit preposicionem, dicens: \almanica{verf durch}

\almanica{haus} 
\end{VerbatimLatin}

\footnotetext[17]{Tj. über.}

%%% Local Variables: 
%%% mode: latex
%%% TeX-master: "ParkoszLatin"
%%% End: 
