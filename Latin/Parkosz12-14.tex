\ppageno=14
\ppreviouspageno=13
\plineno=41
\psublineno=1

\newParkoszpage

{\relsize{-1}
\url{http://ebuw.uw.edu.pl/Content/215606/directory.djvu?djvuopts=&page=54&zoom=width&showposition=0.5,0.18}

\url{http://ebuw.uw.edu.pl/Content/215606/directory.djvu?djvuopts=&page=77&zoom=width&showposition=0.5,0.66}
}

\bigskip

\obeylines
\mono



\fullpreviouslines


{
\color{blue}

\indentP \add{P}ostremo quia relique littere, videlicet
\textit{h} . \textit{k}. \textit{ⱥ}. \textit{r}. \textit{t}. \textit{x} . \textit{ij} et \textit{z}., nisi in quantum cum
\textit{ſ} asperatur, nullam difficultatem aut  differenciam

scribendi aut proferendi a litteris latinis habent,
}

\plineno=0

\fulllines

nisi, ut premissum est, nos in antea k loco c obtusi

utemur. Nam etsi prius scribebantur iste dicciones: \parkosz{kath} \parkosz{kɬoda}

\parkosz{kɲap} \parkosz{kɱotr} \parkosz{krooɬ} \parkosz{kopa} etc. per \textit{c}, nos deinceps

\newtip{79}{W rkp.: singulas.}
scribemus omnes istas dicciones et similes¹ per \textit{k} ut \parkosz{kath}
% ł? W rkp.: singulas.

\parkosz{kɬoda} \parkosz{kɲap}, \parkosz{kɱotr} \parkosz{kroɬ} \parkosz{kuṇrath} \parkosz{køøth}. Et

in veritate \textit{q} sicut apud Latinos, ita et apud Polonos

superfluit. Nam, ut premissum est, sicut loco \textit{c} obtusi

\textit{k} utemur, sic loco \textit{q} \textit{k} uti possemus. Exemplum sicut

scribimus: \parkosz{Quas} \parkosz{qvath} \parkosz{qvap} \parkosz{qveɬi} \parkosz{kvikaa}. Et

nullus nos iuste reprehendere poterit.

\add{S}unt denique adhuc apud Latinos aliqui caracteres,

non littere quidem, sed litterarum suppleciones seu titelli sive

apices, id est sumitates, sic dicte, quia frequencius


% Quote/Cytat - Joel Fredell <jfredell@selu.edu> (Mon 20 Apr 2015
% 03:49:52 PM CEST):

% > Neither is available in Unicode or MUFI in my experience, and I've looked
% > as well for our Kempe project.  The closest equivalent is rotunda r
% > (&#xA75B;), but that raises the familiar problem of appropriating a glyph
% > based on resemblance rather than identification.

% I think I will use COMBINING LATIN SMALL LETTER R ROTUNDA' (U+1DE3).


% From: jsbien@mimuw.edu.pl (Janusz S. Bień)
% Subject: Re: How to encode "A mark resembling the Arabic numeral 2"?
% To: mufi-fonts@googlegroups.com
% Date: Tue, 21 Apr 2015 15:51:50 +0200 (1 week, 2 days, 2 hours ago)
% Reply-to: mufi-fonts@googlegroups.com


% Thanks to all who answered my query.

% [...]

% Thanks for drawing my attention to the "combining zigzag above".

% On Mon, Apr 20 2015 at 22:25 CEST, as@signographie.de writes:

% [...]

% > As always when such discussions arise, I suggest first to give at
% > least one scanned image of a typical occurence. It is a rather bad
% > idea to keep the eye out of the game here. Without *see*ing what you
% > mean – this talk becomes a hot candidate for failure.

% My original question was a general one and concerned the character
% described in Wikipedia.

% My specific problem is to encode the enclosed fragment of the
% transcription of the manuscript which relevant fragment is also
% enclosed. The characters has been commented by the editor of the
% transcription.

% The first one is described as "edam", it looks to me as TIRONIAN SIGN ET
% and COMBINING SHORT STROKE OVERLAY.

% The second one is described as "contra", I would encode it as LATIN
% SMALL LETTER CON A76F and COMBINING TILDE.

% The third one is described as "scilicet" or "sed", I would encode it as
% the long s and LATIN SMALL LETTER ET A76B.

% The fourth one is described as "con", "com", "cum" or "cun" and there is
% rather no doubt that it is just LATIN SMALL LETTER CON A76F.

% Quote/Cytat - Susana T Pedro <susana.t.pedro7@sapo.pt> (Wed 22 Apr
% 2015 12:14:28 PM CEST):

% > There is 1DD1 "COMBINING UR ABOVE" in Unicode block Combining
% > Diacritical Marls Suplement.
% >
% > The glyph may not represent the character's actual shape in any
% > particular manuscritpt, but that is not relevant.

% Thank for the suggestion, I will use the character.

% From: "Andrea de Leeuw van Weenen" <andreal@xs4all.nl>
% Subject: 1DD1
% To: mufi-fonts@googlegroups.com
% Date: Wed, 22 Apr 2015 14:03:23 +0200 (1 week, 1 day, 4 hours ago)
% Reply-To: mufi-fonts@googlegroups.com

% It was suggested to use 1DD1 &ur; combining ur above. Would &urrot;
% combining abbreviation mark superscript ur round r form (F153) not be
% better?
% Andrea


\newtip{80}{Tj. eciam.}
\secondtip{81}{Tj. contra.}
\thirdtip{82}{Tj. scilicet lub sed.}
\fourthtip{83}{Tj. con, com, cum lub cun.}
\fifthtip{84}{Tj. tur (końcówka).}
in superiori is parte ponuntur, videlicet \Cardo{⁊̵}¹, \Cardo{ꝯ̃}²,\Cardo{ſꝫ}³, \Cardo{ꝯ}⁴, \Cardo{t  }⁵,  Hos
% 80	Tj. edam.
% The first one is described as "edam",[!!!] it looks to me as TIRONIAN SIGN ET
% and COMBINING SHORT STROKE OVERLAY.
% 81	Tj. contra.
% The third one is described as "scilicet" or "sed", I would encode it as
% the long s and LATIN SMALL LETTER ET A76B.
% 82	Tj. scilicet lub sed.
% The second one is described as "contra", I would encode it as LATIN
% SMALL LETTER CON A76F and COMBINING TILDE.
% 83	Tj. con, com, cum lub cun.
% LATIN SMALL LETTER CON A76F.
% 84	Tj. tur (końcówka).
% 1DD1 "COMBINING UR ABOVE"
% NO-BREAK SPACE (dwa razy!)
% M+F153 COMBINING ABBREVIATION MARK SUPERSCRIPT UR ROUND R FORM
nos curamus nostris litteris annotare, tum quia

Poloni raro his apicibus seu titellis sive supplecionibus

utuntur, tum eciam quia, si eis collibuerit, similibus

caracteribus cum Latinis eos scriptitabunt.

\indentP \add{H}oc eciam silencio pretermittendum non est, quod omnes

littere abecedarii, cum in capitibus versuum aut

articulorum, aut capitulorum ponuntur, frequenter in

caracteribus et figuris a suis usitatis caracteribus et

figuris discrepant. Que nos versualia seu

capitalia vulgariter vocamus. Quod presertim de

capitalibus habet veritatem. Quia vero diversi diverso modo

quasi pro voluntate sua scribendo dicta versualia

et capitalia figurant, ideo de hoc regulas non

curamus ponere, quia isto usu et consuetudine scribendum

distingu untur. In fine autem nostri abecedarii exempla horum

annectemus.

\indentP \add{S}equitur modo abecedarium cum distinc

cionibus et exemplis caracterum et figurarum

presignatum. Hoc tamen premitto,

quod in nostro abecedario non eundem ordinem littere habent,

sicut in Latino, ex causis certis. Nam k non ponetur inter

I, et l, sed ante d, ut sit differenda insignior inter c proprie vocis

et c obtusum, loco cuius ut premissum k ponemus.

Sic eciam I ponetur ante h propter connexitatem exemplorum,

et y sequetur statim post I, ut habeatur inter ea \hyphh{ali}{qualis}

\hypht{ali}{qualis} differenda. Et z statim post s propter similitudinem vocis. Tamen

eciam in suis locis possemus singulas predictas

\newpage




\ppreviouspageno=15
\plineno=0


\fullpreviouslines


{
\color{blue}

litteras seu earum exempla ponere, sicut et
ponemus.


}




\endinput


%%%%%%%%%%%%%%%%%%%%%%%%%%%%%%%%%%%%%%%%%%%%%%%%%%%%%%%%%%%%%%%%%%%%%%%%%%%%%%%%%%%%%%%%%%%






\catcode `\^^M=5

  \newtip{48}{Łoś niesłusznie uważa, że \textit{bika} w obu wypadkach
    napisano błędnie zamiast \textit{ƀyka}. Przykłady są bowiem podane
    w~pisowni dotychczasowej dla pokazania jej niewystarczalności do
    zróżnicowania wyrazów \textit{bika} i \textit{byka}.}

\obeylines






\newcommand{\margin}[1]{\annotatetextBlue{\{#1\}}{zapisy na marginesie}}


% \renewcommand{\over}[1]{\colorbox{blue!10}{\{#1\}}}

\renewcommand{\over}[1]{\annotatetextBlue{\{#1\}}{zapisy nad rządkami}}

% litery i wyrazy dodane, (których w tekście brak)
%\newcommand{\add}[1]{\colorbox{olive!10}{<#1>}}
\newcommand{\add}[1]{\annotatetextOlive{<#1>}{litery i wyrazy dodane, (których w tekście brak)}}

% litery i wyrazy zbędne
% \newcommand{\extra}[1]{\colorbox{magenta!10}{[#1]}}
\newcommand{\extra}[1]{\colorbox{magenta!10}{[#1]}}

% przekreślenia
% MATHEMATICAL LEFT WHITE SQUARE BRACKET' (U+27E6)
% 'MATHEMATICAL RIGHT WHITE SQUARE BRACKET' (U+27E7)
\newcommand{\overstr}[1]{\annotatetextMagenta{⟦#1⟧}{przekreślenia}}



%%% Local Variables:
%%% mode: latex
%%% TeX-PDF-mode: t
%%% TeX-engine: luatex
%%% TeX-master: "ParkoszLatin"
%%% default-input-method: "Parkosz-slash"
%%% End:
