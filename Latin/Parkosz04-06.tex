\url{http://wbl.klf.uw.edu.pl/13/2/iParkosz.djvu?djvuopts=&page=46&zoom=width&showposition=0.5,0.18}

\begin{VerbatimLatin}[numbers=none,formatcom=\color{blue}]
Eecedat ergo demens 
\end{VerbatimLatin}
%
\renewcommand{\theFancyVerbLine}{06-0\arabic{FancyVerbLine}\phantom{a}}
%
\begin{VerbatimLatin}
arguicio et accedat comodosa Polonorum lingue in scripto servicio. Etc.

\dc{V}iginti duas litteras apud Hebreos, Siros et Caldeos in prologo

super libros Regum beatus scribit Hieronimus. Latini autem una

magis habent litteras, caracteribus quidem et figuris sed non in omnibus

vocibus differentes et obtusum seu raucum \textit{k} et \textit{q}. Quamvis quandoque

eidem caracteri diversos sonos attribuant, alique tamen eorum apud eos

superfluunt. Nam Latinum ideoma \textit{k} non indiget, \textit{h} eciam aspiracionis nota
\end{VerbatimLatin}
\renewcommand{\theFancyVerbLine}{\textcolor{green}{06-08\alph{FancyVerbLine}}}
\begin{VerbatimLatin}[firstnumber=1]
est.

\indentK Nostrum autem Sclavonicum ideoma, et presertim Polonicum, multo
\end{VerbatimLatin}
\renewcommand{\theFancyVerbLine}{06-0\arabic{FancyVerbLine}\phantom{a}}
%
\begin{VerbatimLatin}[firstnumber=9]
pluribus indiget litteris. Primo namque cum Latini quinque vocalibus contenti sunt,
\end{VerbatimLatin}
\renewcommand{\theFancyVerbLine}{06-\arabic{FancyVerbLine}\phantom{a}}
%
\begin{VerbatimLatin}[firstnumber=10]
Poloni \extra{autem} sextam ø adiciunt, nec sine ea illud ideoma scribi potest.

Nam licet alicubi loco \parkosz{ø} \parkosz{an} scribi possit, ut \parkosz{ranka} \parkosz{manka}, \parkosz{røka}

\parkosz{møka}, alibi tamen nullo modo, ut \parkosz{møøka} \parkosz{drøġa}. Alioquin inter penam et farinam
\end{VerbatimLatin}
\renewcommand{\theFancyVerbLine}{\textcolor{green}{06-13\alph{FancyVerbLine}}}
\begin{VerbatimLatin}[firstnumber=1]
et multa alia non erit differencia.

\indentK Omnes eciam vocales apud Polonos
\end{VerbatimLatin}
\renewcommand{\theFancyVerbLine}{06-\arabic{FancyVerbLine}\phantom{a}}
%
\begin{VerbatimLatin}[firstnumber=10]
modo longantur, modo patulo breviantur. Ex quarum longacione vel \hyphh{bre}{viacione}

\hypht{bre}{viacione} diversus consurgit sensus diccionum. Exemplum de \parkosz{a} sicut \parkosz{\hyphh{verci}{maak}}
% - ???:
%-maak, ubi si a producitur, una diccio est, si corripitur, due sunt |
\parkosz{\hypht{verci}{maak}}, ubi si \parkosz{a} producitur, una diccio est, si corripitur, due sunt

dicciones. Exemplum de \parkosz{e} sicut \parkosz{beel} \parkosz{bel}. Exemplum de \parkosz{i}: \parkosz{byl} \parkosz{bil}. Exemplum de \parkosz{o}, 

ubi longatur, ubi corripitur: \parkosz{kooth} \parkosz{koth}. De \parkosz{ø}: \parkosz{drøġa} \parkosz{drøøġa}.
\end{VerbatimLatin}
\renewcommand{\theFancyVerbLine}{\textcolor{green}{06-15\alph{FancyVerbLine}}}
\begin{VerbatimLatin}[firstnumber=1]
Exemplum de u: druġa druuġ.

\indentK Et quanquam Latini in produccione et \hyphh{correp}{cione}
\end{VerbatimLatin}

\footnotetext[27]{Prawdopodobnie błędnie zam. Heberhardi, tj. Eberliardi.}

\renewcommand{\theFancyVerbLine}{06-\arabic{FancyVerbLine}\phantom{a}}
%
\begin{VerbatimLatin}[firstnumber=16]
\hypht{correp}{cione} vocalium in scribendo nullam aut paucam faciant differenciam,

% - - ???:
% pro eo, quia de quantitate — produccione et breviacione vocalium — — suffi|
pro eo, quia de quantitate — produccione et breviacione vocalium — \hyphh{suffi}{cientes}

\hypht{suffi}{cientes} habent regulas in libris grammaticorum, scilicet Prisciani,

\conf{Hebreardi}{}{\color{red}\footnotemark[27]}, Allexandri et aliorum positas, has igitur sciendas
%27	Prawdopodobnie błędnie zam. Heberhardi, tj. Eberliardi.

per Latinos presupponunt, et ideo uno et eodem caractere seu figura

vocales breves et productas scribunt, ut brevitati scripturarum

consulant. Et quia has regulas prosodie Polonis illiteratis difficile

foret tradere, opportuit, ut inscribendo quantitas vocalium \hyphh{expri}{matur}

\hypht{expri}{matur}. Quod alio modo facere facile non est, nisi ut vocalis longa

geminetur et brevis simpla ponatur. Ut \parkosz{adaam}, ubi primum \parkosz{a}

breve, secundum longum et geminatum. Si enim differencias longe et brevis

vocalis tam in voce quam in caractere facere vellemus, forsan non

multum difficile foret. Nam Greci hunc modum scribendi servant, ut

aliter \textit{o} longum, aliter \textit{o} breve scribant. Nos autem, qui omnia a

Latinis mutuavimus, hanc novitatem pretermittere voluimus, forsan
\end{VerbatimLatin}
\renewcommand{\theFancyVerbLine}{\textcolor{green}{06-31\alph{FancyVerbLine}}}
\begin{VerbatimLatin}[firstnumber=1]
enim aliquibus foret odiosa.

\indentK Necesse autem habemus quantitatem \hyphh{voca}{lium}
\end{VerbatimLatin}
\renewcommand{\theFancyVerbLine}{06-\arabic{FancyVerbLine}\phantom{a}}
%
\begin{VerbatimLatin}[firstnumber=32]
\hypht{voca}{lium} in scribendo geminando et simplando exprimere, quia, ut

premissum est, hoc pretermisso magna occurreret in distinguendis

significatis dicciorum difficultas. Ideo etsi non omnis vocalis producta

geminabitur, saltem hoc observabitur, ubi ex eius breviatione et

produccione surgit notabilis diversitas significati eiusdem \hyphh{dic}{cionis}

\hypht{dic}{cionis}. Quamvis eciam et ex parte consonantium hec differencia notari possit, ut
\end{VerbatimLatin}
\renewcommand{\theFancyVerbLine}{\textcolor{green}{06-38\alph{FancyVerbLine}}}

\footnotetext[28]{Litery nadpisane Ulanowski czytał \textit{b} i
  \textit{v}. Pierwszą należy raczej czytać \textit{s}. Obie mogą być
  jakimiś skrótami (choć skróceń nie zaznaczono), np. sine variacione
  — bez zmian (występują litery \textit{h}, \textit{k}, \textit{q},
  \textit{r}, \textit{t}). Ostatnie \textit{c} (Ulanowski je czyta
  jako \textit{r}, a \textit{r} poprzednie jako \textit{v}) znalazło
  się tu przez jakąś pomyłkę.}

\begin{VerbatimLatin}[firstnumber=1]
infra dicetur.

\indentK \dc{P}orro omnes fere littere, exceptis scilicet \over{s} h \over{v} k q r t \extra{c}{\color{red}\footnotemark[28]}, in sono \hyphh{va}{riantur}
% 28	Litery nadpisane Ulanowski czytał b i v. Pierwszą należy raczej
% czytać s. Obie mogą być jakimiś skrótami (choć skróceń nie zaznaczono),
% np. sine variacione — bez zmian (występują litery h, 1c, q, r, t). Ostatnie c
% (Ulanowski je czyta jako r, a r poprzednie jako v) znalazło się tu przez
% Jakąś pomyłkę.

\end{VerbatimLatin}
\renewcommand{\theFancyVerbLine}{06-\arabic{FancyVerbLine}\phantom{a}}
%
\begin{VerbatimLatin}[firstnumber=39]
\hyphh{va}{riantur}. Nam \textit{c} quinqies, \textit{s} sexies sonum variat. \textit{V} autem

in quantum vocalis ad modum aliarum vocalium et longum, et

breve est, in quantum autem consonaeciam aliquociens variat sonum.

Nam cum componitur ex consona et vocali, tunc proprie scribitur, sic

\parkosz{wſta} \parkosz{wmaar}. Ponitur eciam ut simplex consonans et tunc
\end{VerbatimLatin}

%%% Local Variables: 
%%% mode: latex
%%% TeX-master: "ParkoszLatin"
%%% End: 
