http://wbl.klf.uw.edu.pl/13/2/iParkosz.djvu?djvuopts=&page=54&zoom=width&showposition=0.5,0.33

 q

file:///mnt/MyBookT2/hg-cykat/Lublin2015/ParkoszFR11.djvu?djvuopts=&page=22&zoom=page&showposition=0.49,0.49

http://wbl.klf.uw.edu.pl/13/2/iParkosz.djvu?djvuopts=&page=109&zoom=page&showposition=0.49,0.49&highlight=83,2443,2626,315

bent, |! nisi, ut premissum est, nos in antea k loco c obtusi | s. 14

utemur. Nam etsi prius scribebantur iste dicciones: kath kloda |
Jcąap Tcrriotr krool kopa etc. per c, nos deinceps | scribemus

omnes istas dicciones et similes79 per k ut kath | kloda kąap, 5

kviotr krol kunrath kooth. Et | in veritate q sicut apud Lati¬

nos, ita et apud Polonos | superfluit. Nam, ut premissum est,

sicut loco c obtusi | k utemur, sic loco q k uti possemus. E-

xemplum sicut | scribimus: Quas qvath qvap qveti kvikaa. Et I

nullus nos iuste reprehendere poterit. |	10
<S}unt denique adhuc apud Latinos aliqui caracteres, [non

littere quidem, sed litterarum suppleciones seu titelli sive | api¬

ces, id est sumitates, sic dicte, quia frequencius | in superiori

97 Jesi; napisane zzaia, w oryginale "było z pewnością zzara, kopista

wziął r za i, por. J. Łoś, Ziarać, Język Polski I, 1913, s. 190.
. *8 W rkp.: v.


http://wbl.klf.uw.edu.pl/13/2/iParkosz.djvu?djvuopts=&page=54&zoom=width&showposition=0.5,0.33&highlight=969,3136,2034,138

is parte ponuntur, videlicet i!80, 981, /ś82, ę83, t*84, Hos | nos

curamus nostris litteris annotare, tum quia | Poloni raro his

apicibus seu titellis sive supplecionibus | utuntur, tum eciain

quia, si eis collibuerit, similibus | caracteribus cum Latinis eos

scriptitabunt. |
<H)oc eciam silencio pretermittendum non est, quod omnes j

20 littere abecedarii, cum in capitibus versuum aut | articulorum,

aut capitulorum ponuntur, frequenter in | caracteribus et fi¬

guris a suis usitatis caracteribus et | figuris discrepant. Que

nos versualia seu | capitalia vulgariter vocamus. Quod preser-

25 tim de | capitalibus habet veritatem. Quia vero diversi di¬

verso modo | quasi pro voluntate sua scribendo dicta versu¬

alia | et capitalia figurant, ideo de hoc regulas non | curamus

ponere, quia isto usu et consuetudine scribendum | distingu-

3o untur. In fine autem nostri abecedarii exempla horum [ annec¬

temus. |
<S)equitur modo abecedarium cum distinc|cionibus et exem¬

plis caracterum et figurarum | presignatum. Hoc tamen pre-

mitto, | quod in nostro abecedario non eundem ordinem lit-

35 tere habent, | sicut in Latino, ex causis certis. Nam k non

ponetur inter 11, et l, sed ante d, ut sit differenda insignior

inter c proprie vocis | et c obtusum, loco cuius ut premissum

k ponemus. | Sic eciam I ponetur ante Ti propter eonnexi-

tatem exemplorum, | et y sequetur statim post I, ut habea-

40 tur inter ea ali|qualis differenda. Et z statim post s propter

similitudinem vocis. Tamen | eciam in suis locis possemus sin-

s. 15 gulas predictas || litteras seu earum exempla ponere, sicut et

ponemus. Non 1 est enim vicium idem pluries repetere ex causa

notandum. |
80	Tj. edam.
81	Tj. contra.
82	Tj. scilicet lub sed.
83	Tj. con, com, cum lub cun.
84	Tj. tur (końcówka).




\ppageno=8
\ppreviouspageno=7
\plineno=48
\psublineno=1

{\relsize{-1}
\url{http://wbl.klf.uw.edu.pl/13/2/iParkosz.djvu?djvuopts=&page=48&zoom=width&showposition=0.5,0.18}

\url{http://wbl.klf.uw.edu.pl/13/2/iParkosz.djvu?djvuopts=&page=67&zoom=width&showposition=0.5,0.67}
}

\bigskip

\obeylines
\mono
\fulllines


{
\color{blue}
Nec, ut pretactum est, per vocales longas aut

}

\splitlines

additas differencia notari potest.

\indentP {b, b}. Si igitur libet, <sit> b grossum sine unco et \hyphh{qua}{dratum}


\endinput

Ponitur eciam ut simplex consonans et tunc

}

\plineno=0

\pnoteno=29
\def\tip{Lekcja nie jest pewna, Ulanowski czytał: vilne.}
\footnotetext[\the\pnoteno]{\tip}

aliquando grossatur, aliquando molliter profertur. Exemplum primi: \parkosz{ʋiklad} \parkosz{ʋige}

\parkosz{ʋya}. Exemplum secundi: \parkosz{ʋila} \conf{{\parkosz{vilue}}}{}\annotatehere[\the\pnoteno]{\tip}. Quando igitur \textit{u} vocalis simpla ponitur, 
% 29	Lekcja nie jest pewna, Ulanowski czytał: vilne.

eodem modo sicut u scribitur, in superiori parte apertum, in inferiori ligatum. 

Quando autem longatur, tunc geminatur. Exemplum primi in \parkosz{uÿe} \parkosz{uyass}.

\advance\pnoteno by 1
\def\tip{Winno być: ſzuuwam, dmuuchaa.}
\footnotetext[\the\pnoteno]{\tip}

Exemplum secundi: \parkosz{ſzvwam} \parkosz{dmuchaa}\annotatehere[\the\pnoteno]{\tip}. Quando est consona et grossatur, tunc
% 30 Winno być: fsuuwam, dmuuchaa.

de superius primi cornu tractus ducatur, ut \parkosz{ʋaal} \parkosz{ʋyl} \parkosz{ʋiklad}.

Quando vero mollitur, tunc planis et equis cornibus scribatur, ut sic: \parkosz{vila}

\splitlines

\parkosz{vidzaall} \parkosz{vino}.

\indentK Hac differencia habita inter \textit{v} consonantem mollem et

\fulllines

grossam, non erit necesse ponere duplex \textit{ÿ}, ut olim ponebatur,  ut \parkosz{vyatr}·

\parkosz{wyege}· \parkosz{wyofna}· \parkosz{vyonczek}, quia non esset differenda inter \parkosz{vial} et \parkosz{vaal}

id est flavit, sed sic scribatur molle \textit{v}: \parkosz{vaal}· \parkosz{vathr}· \parkosz{vege}· \parkosz{voſna}·

\parkosz{vøczek}· \parkosz{vvaal}. Peccatum est enim fieri per plura, quod eque bene potest fieri per 

\splitlines

pauciora.

\indentK Ille autem sex littere: \textit{b} \textit{f} \textit{l} \textit{m} \textit{n} \textit{p} sub eadem vocis retentione

\newsplitline

et quantitate nunc grosse, nunc molliter proferuntur.

\indentK \margin{De b}. Exemplum primi

\fulllines

% ¹ 'SUPERSCRIPT ONE' (U+00B9)
% ² U+00B2 	SUPERSCRIPT TWO
% U+00B3 	SUPERSCRIPT THREE	
\catcode`\¹=13
\catcode`\²=13

% \advance\pnoteno by 1
% \def\tip{Jeden z tych przykładów winien być napisany przez b!, drugi przez ɓ.}
% \footnotetext[\the\pnoteno]{\tip}
% \def¹{\annotatehere[\the\pnoteno]{\tip}}

\def\newtip#1#2{
% \pnoteno=\psecnoteno
% \advance\pnoteno by 1
% \footnotetext[\the\pnoteno]{#1}
% \def¹{\annotatehere[\the\pnoteno]{#1}}}
\footnotetext[#1]{#2}
\def¹{\annotatehere[#1]{#2}}}

\newtip{31}{Winno być: ƀyk.}

de \parkosz{ɓ}· \parkosz{ɓyk}· \parkosz{ɓith} id est percussus. Exemplum secundi: \overstr{\textit{b}} \parkosz{bil} id est fuit, \parkosz{ɓyk}¹ id est
% 31 31	Winno być: byk.

thaurus. Nec ex parte vocalis geminate et longate potest esse

differentia. Nam idem \parkosz{ɓ} molle et grosse prolatum, cum \parkosz{y} longo et \parkosz{i}

brevi reperitur. Licet enim, ubi \parkosz{ÿ} longatur, merito \parkosz{b} deberet grossari,

et ubi \parkosz{i} breviatur, deberet iuste molliri, ut in exemplis suprapositis:

\newtip{32}{Jeden z tych przykładów winien być napisany przez ƀ, drugi przez ɓ}

\parkosz{ɓyk} et \parkosz{ɓik}¹ est tamen reperire dictiones, in quibus \parkosz{i} breviatur, et tamen \parkosz{ɓ}
% 32	Jeden z tycli przykładów winien być napisany przez b, drugi przez ó.

nunc mollitur, nunc grossatur, ut \parkosz{bith} id est habitatio, \parkosz{ɓith}

id est percussus. In his dictionibus \parkosz{i} breviatur, et tamen \parkosz{ɓ} mollitur et

grossatur. Aliquando e converso, \parkosz{y} producitur, et tamen \parkosz{ɓ} grosse et

\splitlines

molle profertur, ut \parkosz{ɓyl} id est percussit, \overstr{ɓ} \parkosz{byl} id est fuit.

\indentP \margin{ff}. Sic \parkosz{ff} nunc

\fulllines

grosse, nunc molle profertur, ut \parkosz{ffaal} et est proprium nomen, \parkosz{faal}

id est movit. Nec per vocales additas differre possunt. Nam \parkosz{ffaal}

et faal utrobique \parkosz{aa} longum seu geminatum, et \parkosz{f} differt in voce.

Nam si diceretur, quod vocalis longata seu producta faceret consonantem


% \def\secondtip#1{
% \psecnoteno=\pnoteno
% \advance\psecnoteno by 1
% \footnotetext[\the\psecnoteno]{#1}
% \def²{\annotatehere[\the\psecnoteno]{#1}}}

\def\secondtip#1#2{\footnotetext[#1]{#2}\def²{\annotatehere[#1]{#2}}}


\newtip{33}{W rkp.: grossiori.}
\secondtip{34}{W rkp.: molliori.}

%\def²{?}

litteram \conf{grossari}{}¹, et breviatam \conf{molliri}{}², sicut de \parkosz{ɓ} dictum est,
% 33	W rkp.: grossiori.
% 34	w rkp.: molliori.

hoc apparet falsum in exemplis statim positis: \parkosz{ffaaɬ} \parkosz{faal}. Imo

alicubi vocalis producitur et \parkosz{f} mollitur, ut \parkosz{fyſth}, et e converso,

\newtip{35}{Winno być: ffitaa.}
vocalis corripitur, et \parkosz{ff} grossatur, ut \parkosz{ffytaa}¹, et iterum \parkosz{i} breviatur,
% 35	Winno być: ffitaa.

\newtip{36}{Winno być: fyjth.}

\splitlines

ut \conf{\parkosz{fiſth}}{}¹ \overstr{ffy} \parkosz{ffita} \parkosz{figi}.
% 38 Winno być: fyjth.

\newtip{37}{Winno być: ɬl lub lɬ.}

\indentK \margin{ll}¹. Eodem modo \textit{l} nunc grossatur, nunc \hyphh{mol}{litur}
% 57 Winno być: li lub 11.

\fulllines

\hypht{mol}{litur}, ut \parkosz{ɬift} id est littera vel folium, \parkosz{liſth} ut est pars pedis, \parkosz{ɬis} id est

vulpis, \overstr{ɬi} \parkosz{liſz} id est calvus: ecce ceteris paribus nunc grossatur,

nunc mollitur sic et in fine positum. Exemplum: \parkosz{Staaɬ} id est calibs,

\splitlines

\parkosz{ſtaal} id est stetit.

\indentK \margin{m}. Eodem modo \textit{m} grossatur et mollitur circa easdem

\fulllines

vocales et consonantes positum. Exemplum: \parkosz{mika} id est Nicolaus, \parkosz{ɱikaa} 

\newtip{38}{Winno być: gyɱ.}

id est trahit. Et in fine diccionis positum. Exemplum: \parkosz{dyɱ}· \parkosz{groɱ}· \conf{\parkosz{gym}}{}¹
% 38 Winno być: gyrą.

\splitlines

\newtip{39}{Winno być: gym.}

id est eis, \parkosz{gyɱ}¹ id est tene, \parkosz{przym} id est suscipe.

\indentK \margin{n}. Consimiliter \textit{n} grossatur

\fulllines

et mollitur. Exemplum de grosso: \parkosz{ŋiſſki} id est de civitate Nissa. Exemplum

\splitlines

\newtip{40}{Winno być :ſyŋ}

de molli: \parkosz{nifki} id est declivis. Sic eciam in fine: \conf{\parkosz{ſyn}}{}¹, \parkosz{koon}.
% 40	Winno być: fy^.


\indentK \margin{p}. Consimiliter


\fulllines

\textit{p} grosse et molliter profertur: \parkosz{pan}· \parkosz{paʋel}· \parkosz{piſchno} \parkosz{pige}· \parkosz{pivo}.

\newtip{41}{Ulanowski czytał: povec.}
\secondtip{42}{Nie zróżnicowano pisowni p.}

\parkosz{pothr}· \conf{\parkosz{povee}}{}¹· \conf{\parkosz{popu}}{}²·
% 41	Ulanowski czytał: povec.
% 4J Nie zróżnicowano pisowni p.

Quas igitur differencias his vocibus harum litterarum attribuemus? Si

aliquas differencias ponere voluerimus, non parum difficultatis 

\newtip{43}{Winno być: habetur lub: habemus.}

habet¹, si nullas, multum erroris sequi necesse est. Nam eadem 

figura litteram molliter et grosse prolatam scribere non parvum

errorem parit. Nec, ut pretactum est, per vocales longas aut

%%% Local Variables: 
%%% mode: latex
%%% TeX-PDF-mode: t
%%% TeX-engine: luatex 
%%% TeX-master: "ParkoszLatin"
%%% End: 
