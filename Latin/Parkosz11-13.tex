\ppageno=13
\ppreviouspageno=12
\plineno=41
\psublineno=1

\newParkoszpage

{\relsize{-3}
%\url{http://ebuw.uw.edu.pl/Content/215606/directory.djvu?djvuopts=&page=53&zoom=width&showposition=0.5,0.18}
  \url{https://jsbien.github.io/Parkosz4IIIF/\#?cv=10&manifest=https://jsbien.github.io/Parkosz4IIIF/collection/ParkoszJBC/index.json}

\url{http://ebuw.uw.edu.pl/Content/215606/directory.djvu?djvuopts=&page=76&zoom=width&showposition=0.5,0.29}
}

\bigskip

\obeylines
\mono



\fullpreviouslines


{
\color{blue}

% \splitlines

% \newtip{72}{Można też czytać \textit{Roſſa}.}
% \parkosz{miſſa} \parkosz{Koſſa}¹.
% % 72	Można też czytać Boffa.

\indentK Si igitur vellemus propriam differenciam

\fulllines

habere inter \parkosz{S} in propria voce prolatum et \parkosz{s} molle, ut propositum,
}

\plineno=0

\fulllines

inter duas vocales, scribamus primum convolutum, sicut solitum

est poni in fine diccionum. Exemplum: \parkosz{Saaṃ} \parkosz{Seeṇ} \parkosz{Syɲ}

\parkosz{Søø} id est iudicium, \parkosz{SSøød} id est vas. Sic eciam positum cum

conso nantibus: \parkosz{Stado} \parkosz{Sḷuꝿa} \parkosz{Sboſzɲi} \parkosz{Struſ} \parkosz{Sṃøød} etc.

Ubi autem mollitur, scribatur simplex longum, ut \parkosz{ṃaſaal}

\parkosz{coſa}. Et sic quociens a modo \parkosz{ſ} longum reperiemus, legemus

ipsum molliter, nisi sit geminatum, ut ṃiffa id est scultella,

\splitlines

ṃøſſo.

\indentK Interdum autem \parkosz{S} non mollitur, nec eciam in sua

\fulllines

propria voce ponitur, sed interdum aspiratur, ut cum ponitur cum
\overstr{ch} c. h sequente, ut \parkosz{ſchadi} \parkosz{ḟchipee} \parkosz{ſciɱuɲ}

\splitlines

\newtip{73}{W rkp. \parkosz{ſchaṃ} \parkosz{ɓurzaa}.}
\parkosz{ſchopa} \parkosz{ſchuṃ} \parkosz{ſchaṃburzaa}¹.
% 73	-rkp, fcham óurzaa.

\indentK Aliquando autem asperatur

\splitlines

et hoc multipliciter.

\indentK Aliquando grosse. Exemplum: \parkosz{ſzaak} \parkosz{ſzeġotha}

\fulllines

\catcode `\^^M=5
\newtip{74}{Ulanowski czytał ſzum, co Bruckner proponował
  interpretować jako \parkosz{ſzum<p>}, tj. żump.}
\obeylines
\parkosz{ſzywoth} \parkosz{ſzoraʋ} \parkosz{ſzuur}¹ \parkosz{ſzøødɬo}. Et tunc scribitur per \parkosz{ſ} \hyphh{lon}{gum}
% 74	TJlanowski czytał fzum, co Bruckner proponował interpretować

% jako fzum(p), tj. żump.

\splitlines

\hyphh{lon}{gum} et \parkosz{z} et hoc patet in exemplis.

Aliquando grossius, ut quando \parkosz{S}

\fulllines

propriam vocem retinet et quodammodo I intersumit. Et tunc

\newtip{75}{Zapewne zamiast: convolutum.}
debet scribi per \parkosz{ſſ} convulsum¹ aut longum geminatum et \parkosz{z}.
% 75	Zapewne zamiast: convolutum.	;


Exemplum: \parkosz{ffzaṇo} \parkosz{ffzeṃo} \parkosz{ffzadlo} \parkosz{ffzivi} \parkosz{ffzirotha} \parkosz{\hyphh{ffzo}{ſtra}}

\newtip{76}{Można też czytać \parkosz{ſſzathka}, jak Ulanowski.}
\parkosz{\hypht{ffzo}{ſtra}} \parkosz{ffzutka}¹. Nec in his et similibus exemplis sufficit
%76	Można też czytać (Jzathka, jak TJlanowski.


scribere simplex \parkosz{ſ} longum et \parkosz{z}, et duplex \parkosz{ÿ}, ut antiquitus

scribebatur. Sic enim prius scribebatur \parkosz{ſzijano} \parkosz{ſzijemija}

\parkosz{ſzijoſtra}. Quamvis enim in diccionibus, ubi post \parkosz{S} \parkosz{z}

% sprawdzić!

ponitur \parkosz{a} \parkosz{e} \parkosz{o} \parkosz{v} et \parkosz{ø}, forte sic scribi posset, tamen

ubi post hanc asperacionem sequitur \parkosz{I}, ut \parkosz{ſziwi} \parkosz{\hyphh{ſziro}{tha}}

\parkosz{\hypht{ſziro}{tha}}, non esset differencia inter \parkosz{ſzijwij} id est canus et \parkosz{ſziwi} id est vivus

et multa talia. Melior igitur et notabilior erit

differencia hec, ut in predicta asperacione grossiore ponatur

\parkosz{ſſ} convolutum seu duplex longum et z, ut in exemplis supra positis,

\splitlines

sive I sequatur, sive non.

\indentK Aliquando autem asperatur molliter,

\fulllines

ut quasi in vocem \parkosz{z} declinet, et tamen quodammodo \parkosz{z} ingrossatur

et hoc modo aptissime duplex \parkosz{zz} ponatur, ut in exemplo:

\catcode `\^^M=5
\newtip{79}{Jest napisane \parkosz{zzaia}, w oryginale było z pewnością
  \parkosz{zzara}, kopista wziął \parkosz{r} za \parkosz{i}, por. J. Łoś,
  \textit{Ziarać}, Język Polski I, 1913, s. 190.}
\obeylines

\parkosz{zzaia}¹ \parkosz{zzele} \parkosz{zziɱa} \parkosz{zzolo} \parkosz{zzøɓa}. Nec sufficit, ut
% 97 Jesi; napisane zzaia, w oryginale "było z pewnością zzara, kopista

% wziął r za i, por. J. Łoś, Ziarać, Język Polski I, 1913, s. 190.


scribatur simplex \parkosz{ſ} et \parkosz{z} cum duplici \parkosz{ÿ}, ut antea scribebatur,

sic: \parkosz{ſzijarṇo} \parkosz{ſzijeṃija}. Quia etsi in diccionibus, ubi

\newtip{78}{W rkp.: \parkosz{v}.}
post \parkosz{z} \parkosz{a} \parkosz{e} \parkosz{o} \parkosz{u}¹ et \parkosz{ø} ponitur, sic scribendo aliquando differenciam
% . *8 W rkp.: v.

inter simplex \parkosz{z} et istud poneremus, tamen ubi post \parkosz{z} sequitur

\add{i}, nulla erit inter simplex \parkosz{z} et illud ingrossatum

seu geminatum differencia. Exemplum: \parkosz{zzijɱa} \parkosz{zzijɱɲo} \parkosz{zziɱozzeɬoṇ}. Igitur differencia
% do sprawdzenia
supra dicta est melior, que in omnibus veritatem optinet.

\indentP \add{P}ostremo quia relique littere, videlicet

\parkosz{h} . \parkosz{k}. \parkosz{ⱥ}. \parkosz{r}. \parkosz{t}. \parkosz{x} . \parkosz{ÿ} et \parkosz{z}., nisi in quantum cum
% ą !!!


\parkosz{ſ} asperatur, nullam difficultatem aut  differenciam

scribendi aut proferendi a litteris latinis habent,

\ppreviouspageno=14
\plineno=0


\fullpreviouslines


{
\color{blue}

nisi, ut premissum est, nos in antea k loco c obtusi

utemur.

}




\endinput


%%%%%%%%%%%%%%%%%%%%%%%%%%%%%%%%%%%%%%%%%%%%%%%%%%%%%%%%%%%%%%%%%%%%%%%%%%%%%%%%%%%%%%%%%%%

. *8 W rkp.: v.

ł? W rkp.: singulas.


\catcode `\^^M=5

  \newtip{48}{Łoś niesłusznie uważa, że \textit{bika} w obu wypadkach
    napisano błędnie zamiast \textit{ƀyka}. Przykłady są bowiem podane
    w~pisowni dotychczasowej dla pokazania jej niewystarczalności do
    zróżnicowania wyrazów \textit{bika} i \textit{byka}.}

\obeylines






\newcommand{\margin}[1]{\annotatetextBlue{\{#1\}}{zapisy na marginesie}}


% \renewcommand{\over}[1]{\colorbox{blue!10}{\{#1\}}}

\renewcommand{\over}[1]{\annotatetextBlue{\{#1\}}{zapisy nad rządkami}}

% litery i wyrazy dodane, (których w tekście brak)
%\newcommand{\add}[1]{\colorbox{olive!10}{<#1>}}
\newcommand{\add}[1]{\annotatetextOlive{<#1>}{litery i wyrazy dodane, (których w tekście brak)}}

% litery i wyrazy zbędne
% \newcommand{\extra}[1]{\colorbox{magenta!10}{[#1]}}
\newcommand{\extra}[1]{\colorbox{magenta!10}{[#1]}}

% przekreślenia
% MATHEMATICAL LEFT WHITE SQUARE BRACKET' (U+27E6)
% 'MATHEMATICAL RIGHT WHITE SQUARE BRACKET' (U+27E7)
\newcommand{\overstr}[1]{\annotatetextMagenta{⟦#1⟧}{przekreślenia}}



%%% Local Variables:
%%% mode: latex
%%% TeX-PDF-mode: t
%%% TeX-engine: luatex
%%% TeX-master: "ParkoszLatin"
%%% default-input-method: "Parkosz-slash"
%%% End:
