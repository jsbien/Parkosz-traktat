\ppageno=11
\ppreviouspageno=10
\plineno=0
\psublineno=1

\newParkoszpage


{\relsize{-1}
\url{http://ebuw.uw.edu.pl/Content/215606/directory.djvu?djvuopts=&page=51&zoom=width&showposition=0.5,0.18}

\url{http://ebuw.uw.edu.pl/Content/215606/directory.djvu?djvuopts=&page=73&zoom=width&showposition=0.5,0.44}
}


\plineno=0

\fulllines

\indentK \textit{D} eciam duas habet voces, sed facile differentes, scilicet \textit{d} simplex

% znak nieczytelny, l' w dal' może pod wpływem Łosia?
seu spissum et \textit{dz} molle. Exemplum primi: \parkosz{daɬ} \parkosz{deeɲko} \parkosz{diɱ}

% znak nieczytelny, dlaczego oddany jako n'? Do wyjaśnienia.
\parkosz{dooɱ} \parkosz{duɱaa} \parkosz{doøb}. Exemplum secundi: \parkosz{dzaḷ} \parkosz{dzen'} \parkosz{dziv} \parkosz{\hyphh{dzol}{da}}

\parkosz{\hyphh{dzol}{da}} \parkosz{dzuḷa} \parkosz{dzøøġiɬ} \parkosz{dzøøſḷa}. Itaque prima exempla scribantur

per simplex \parkosz{d}, reliqua per \parkosz{d} et \parkosz{z}, quod satis usitatum est.

\indentK Tamen-adhuc \parkosz{dz}, ceteris eodem modo se habentibus, interdum asperius,

interdum mollius profertur. Exemplum primi: \parkosz{ꝿviſzdz} id est sibila,

\parkosz{ꝿviſd} id est nux perforata a verme. Exemplum secundi: \parkosz{Ġvijſzdz}

id est posterior pars selle. Ecce in his exemplis aliter et aliter \textit{dz}

sonat. Et quamvis hec asperacio et mollicio videatur

provenire ex parte \textit{d} et \textit{z}, tamen in veritate primum \textit{dz} in se asperius

quam secundum profertur. Inter hec autem aptam non possumus

dare differendam, sed lectoris prudencie discernendum relinquimus.

Si enim Latini alphabeti inventor, qui fuit peritissimus,

plurimas \add{differencias} litterarum lectoris ingenio reliquit, cur nobis

hoc non liceat, quod peritioribus licuit?

\indentK \textit{G} autem quamvis apud Latinos eciam duas format voces,

ut \textit{gaudium}, \textit{genus}, tamen ipsi hanc differenciam non curant,

ut eam scripto annotent, sed legencium discretioni relinquunt.

Ex quo autem cepimus facere differencias inter aliarum litterarum

\splitlines

voces, et hanc exprimemus.

\indentK Sunt in Polonico huiuscemodi

\fulllines

dicciones ab utraque eius voce incipientes, mediantes et

finientes. Exemplum prime vocis ut \parkosz{ꝿaad} \parkosz{ꝿedka} \parkosz{ꝿid}

\parkosz{ꝿode} \parkosz{ꝿuſz} \parkosz{ꝿøffz} \parkosz{ɱayda} \parkosz{ṃiꝿdal} \parkosz{rooꝿ} \parkosz{ſṃuꝿ}.

Exemplum secunde vocis ut \parkosz{ġee} id est comedit, \parkosz{ġeeṃi} id est comedimus

cum suis condeclineis. Scribantur ergo primi exempli dicciones

per \parkosz{ꝿ} cum unco retorto versus dexteram partem, sicut

scribunt ipsum Italici, ut sic: \parkosz{ꝿraad} \parkosz{ꝿruuda} \parkosz{ꝿroſch},

secundi autem modi seu exempli per \parkosz{ġ} simplex, ut est

usitatum, cum simplici unco versus sinistram partem,

ut retorto versus dexteram, ut \parkosz{ġee} \parkosz{ġeeɱ} \parkosz{ġyɱ}.

\indentK Nec hoc silencio pretereundum est, quod Latini quasi eadem voce

\textit{I} consonantem et \textit{g} litteram, prout secundum hanc consideracionem

\newtip{66}{Winno być: Ianua.}
sine unco retorto scribitur, exprimunt, ut \conf{\parkosz{Iaana}}{}¹,
% “ Winno być: Ianua.

% TODO - sprawdzić!
\parkosz{Iaroṇiṃus}, \parkosz{Iohaṇṇes}, \parkosz{Iuṇius}, \parkosz{ġeṇus}, \parkosz{ġyṃṇasiuṃ}. Et eodem modo Poloni

necesse est ut faciant. Si igitur placet hanc differenciam legentibus

exprimere et eam non aliquibus signis annotare, bene quidem, sed

facilius esset legere, differencia annotata. Possemus tamen hanc differenciam

hoc modo dis tinguere in diccionibus Polonicis, ut quociens \textit{g}

ponitur cum aliqua vocali, et non scribitur seu non

profertur per \parkosz{g} rotundum, tunc non debet scribi \parkosz{g} simplex,

\newtip{67}{Ulanowski czytał: Jaaṇus.}
sed \textit{I} consona, ut \conf{\parkosz{Iacuus}}{}¹ \parkosz{Iaaṇ} \parkosz{Iost}, quod est proprium nomen
% •7 Ulanowski czytał: Jaanus.

\splitlines

sancti Iodoci.

\indentK Item quociens iungitur cum his consonis: \textit{b}

\fulllines

\textit{n} \textit{w} \textit{d} \textit{l} \textit{r}, profertur et scribitur cum unco retorto,

\ppageno=12
\plineno=0

\fullpreviouslines

{
\color{blue}

ut \PARKOSZ{ṃaꝿda} \PARKOSZ{ꝿlaſz} \PARKOSZ{ꝿnew} \PARKOSZ{ꝿrod}.

}

\ppageno=0

\fulllines


ut \PARKOSZ{ṃaꝿda} \PARKOSZ{ꝿlaſz} \PARKOSZ{ꝿnew} \PARKOSZ{ꝿrod}. Sed si ponitur cum \textit{e} et \textit{I}


\endinput


%%%%%%%%%%%%%%%%%%%%%%%%%%%%%%%%%%%%%%%%%%%%%%%%%%%%%%%%%%%%%%%%%%%%%%%%%%%%%%%%%%%%%%%%%%%






«s -yy rfep.; v.





\catcode `\^^M=5

  \newtip{48}{Łoś niesłusznie uważa, że \textit{bika} w obu wypadkach
    napisano błędnie zamiast \textit{ƀyka}. Przykłady są bowiem podane
    w~pisowni dotychczasowej dla pokazania jej niewystarczalności do
    zróżnicowania wyrazów \textit{bika} i \textit{byka}.}

\obeylines






\newcommand{\margin}[1]{\annotatetextBlue{\{#1\}}{zapisy na marginesie}}


% \renewcommand{\over}[1]{\colorbox{blue!10}{\{#1\}}}

\renewcommand{\over}[1]{\annotatetextBlue{\{#1\}}{zapisy nad rządkami}}

% litery i wyrazy dodane, (których w tekście brak)
%\newcommand{\add}[1]{\colorbox{olive!10}{<#1>}}
\newcommand{\add}[1]{\annotatetextOlive{<#1>}{litery i wyrazy dodane, (których w tekście brak)}}

% litery i wyrazy zbędne
% \newcommand{\extra}[1]{\colorbox{magenta!10}{[#1]}}
\newcommand{\extra}[1]{\colorbox{magenta!10}{[#1]}}

% przekreślenia
% MATHEMATICAL LEFT WHITE SQUARE BRACKET' (U+27E6)
% 'MATHEMATICAL RIGHT WHITE SQUARE BRACKET' (U+27E7)
\newcommand{\overstr}[1]{\annotatetextMagenta{⟦#1⟧}{przekreślenia}}



%%% Local Variables:
%%% mode: latex
%%% TeX-PDF-mode: t
%%% TeX-engine: luatex
%%% TeX-master: "ParkoszLatin"
%%% default-input-method: "Parkosz-slash"
%%% End:
