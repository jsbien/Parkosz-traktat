\ppageno=9
\ppreviouspageno=8
\plineno=49
\psublineno=1

{\relsize{-1}
\url{http://wbl.klf.uw.edu.pl/13/2/iParkosz.djvu?djvuopts=&page=49&zoom=width&showposition=0.5,0.18}

\url{http://wbl.klf.uw.edu.pl/13/2/iParkosz.djvu?djvuopts=&page=69&zoom=width&showposition=0.5,0.81}
}

\bigskip

\obeylines
\mono


\fullpreviouslines


{
\color{blue}
Et

}

\plineno=0

\fulllines
\newtip{50}{W rkp. zamiast \textit{m} jest napisane mylnie \textit{l}.}
non erit a modo necesse circa \conf{\parkosz{m}}{}¹ molle ponere duplex \textit{ÿ} et vocalem \hyphh{se}{quentem}
% so ■yy rkp. zamiast m jest napisane mylnie l.

\hypht{se}{quentem}, ut \parkosz{myedz} \parkosz{myood}, sed \textit{m} tenue sta\add{bi}t loco y duplicis

geminati, \textit{e} vel \textit{o}, vel alia vocali manente producta, ut \parkosz{maal} id est

\splitlines
adeps, \parkosz{meedz}· \parkosz{mood} id est mel, \parkosz{mąſſjo}· \parkosz{maaſga}.

\indentK Similiter de \textit{p} et \textit{n},

\fulllines
circa quas consonantes plus errabatur. Nam quociens \textit{N} et \textit{p} 

mollia occurrebant, semper per duplex \textit{y} et vocalem occurrentem 

scribebant, sive illa breviabatur, sive producebatur. Exemplum primi de \textit{A}: 

\parkosz{gnyaaſdo}, de \textit{E}: \parkosz{nyewyaſta}, de \textit{I}: \parkosz{nÿcz}. Hec differencia 

insufficiens erat, nam inter \parkosz{Nya}, quod fuit idolum, et \parkosz{naa}, 

sillabam in diccione \parkosz{gnyaſdo} positam, non erat differencia. Sic de 

\catcode `\^^M=5

\newtip{51}{Łoś uważa, że \textit{kooɲ} jest błędnie napisane zamiast
  \textit{koon}. Jest to jednak prawdopodobnie zapis w pisowni
  niezmodernizowanej, w której na końcu często pisano \textit{ɲ}.}

\obeylines

\parkosz{ɲ} in fine diccionis posito, ut \parkosz{ſgoɲ} id est finis, \parkosz{kooɲ}¹ id est equs. Et 
% 61 Łoś uważa, że Tcoo7i jest błędnie napisane zamiast koon. Jest to

% jednak prawdopodobnie zapis w pisowni niezmodernizowanej, w której

% na końcu często pisano ą.

in fine illius diccionis \parkosz{kooɲ} necesse erat addere duplex \textit{y} ad 

denotandum \textit{i}, quo facto non erit differencia inter \conf{\parkosz{koonÿ}}{}, nominativum singularem, 

\newtip{52}{Winno być: genitivum.}
\secondtip{53}{Winno być: fuga.}
et \parkosz{koonÿ}, accusativum¹ pluralem. Sic inter \parkosz{gonÿ} id est fugatur² in \hyphh{im}{perativo}
% 52 Winno być: genitivum.
% *s Winno być: fuga.

\splitlines
\hypht{im}{perativo}, et \parkosz{gonÿ} id est fugat.

\indentK Sic de \textit{P}. Quociens \textit{P} molle 

\fulllines
seu tenue debebat scribi, semper subiciebatur duplex \textit{y} 

cum vocali sequente, ut \parkosz{pyaſek} id est arena, \parkosz{pyechna}· \parkosz{pyotr}· 

\parkosz{pyąthno}. Sed hec differencia seu expressio insufficiens est. Nam 

iuxta hoc non erit differencia inter \parkosz{pya} et \parkosz{paą}, inter \parkosz{pye} et 

\parkosz{pee}. Sed ut de \textit{N} dicebatur, quod mollitur, quando sine cauda 

scribitur, sive eam sequatur aliqua vocalis, sive non, ut 

\overstr{gnyaſdo} \parkosz{gnaſdo}· \parkosz{newaaſta}· \parkosz{niſczotha}, similiter de \textit{p} 

\newtip{54}{W rkp.: teneri.}
molli, quando rotundatur, debet \conf{tenuari}{}¹, sequentibus quibuscumque 
% £« w rkp>: teneri.

vocalibus aut quando in fine ponitur. Exemplum primi de \textit{a}: 

\textit{paaſek}, de \textit{e}: \textit{pechina}, de \textit{I}: \textit{piſſarz}, \textit{piſſczek}, de o\textit{}: \textit{potr}, de \textit{ø}: \textit{pøthno}. 

\indentP Restat nunc videndum de litteris, que aliter variant voces,

\splitlines
quam spissitudine et tenuitate et sunt: \textit{c} \textit{d} \textit{g} \textit{ſſ}.

\indentK Et primum

\fulllines

circa \textit{c} occurrit non pauca difficultas. Nam \textit{c}, ut 

premisi, quinquies in voce variatur, secundum quod variis 

\newtip{55}{W rkp. błędnie: vocabulis.}
\conf{vocalibus}{}¹ iungitur. Nam iunctum cum \textit{a} aliquando concurrit 
% H W rkj). błędnie: vocabulis,

\newtip{56}{W rkp.: obmutescat.}
in voce cum \textit{k} et quasi \conf{obmutescit}{}¹, ut \parkosz{cath} id est tortor, \parkosz{captuur} 
% 5« "\\r r]q).: obmutescat.


id est capucium. Et ita utuntur Latini eo in ipsa eius voce. Et sic 

\newtip{57}{W rkp.: \textit{v}.}
apud eos iungitur cum \textit{a}, \textit{o}, et \textit{u}¹, ut \textit{capud}, \textit{collum}, \textit{cuculus}. 
%57 W rkp.: v.

Aliquando iungitur aliis et non concurrit in voce cum \textit{k}, sed quasi 

asperatur, id est proprium sonum retinet, quem in alphabeto habet. Sic eo 

utuntur Latini, quando iungitur \textit{e} vel \textit{i}, ut \textit{cepe}, \textit{cibus}. Cum a 

\splitlines
eciam sed valde raro, ut \textit{caix}.

\indentK Et hoc modo prolatum tripliciter in \hyphh{Polo}{norum}

\fulllines

\hyphh{Polo}{norum} ideomate asperatur: grosse, grossius et molliciter, 

quibuscunque vocalibus iungatur. Exemplum primi de \parkosz{a}: \parkosz{czas}· \parkosz{czapka}· 

\newtip{58}{Może błędny zapis zamiast \textit{czapla}.}
\parkosz{czaſcha}, \conf{\parkosz{czapka}}{}¹. Exemplum secundi: \parkosz{czaƚo} \parkosz{caaſk} \parkosz{cemø}. Exemplum tercii: 
% 48	Może błędny zapis zamiast czapla.

\overstr{caper} \parkosz{cap} id est hircus Vallacorum vel interieccio percudentis, 

\parkosz{hynca} \parkosz{kunca}. Sic eciam iunctum cum \parkosz{e} tripliciter asperatur. Exemplum 

primi: \parkosz{çego} \parkosz{çekaa}. Exemplum secundi: \parkosz{cemø}· \parkosz{celø}. Exemplum tercii: 

\parkosz{cepi} \parkosz{cebula}. Sic cum reliquis vocalibus, ut \parkosz{czyɱ} \parkosz{czyn} 

\newtip{59}{Łoś prawdopodobnie słusznie przypuszcza, że to błąd zamiast \textit{czøøbr}.}
\overstr{ut ar} id est arma, \parkosz{czyn} id est fac, \parkosz{czop}· \parkosz{czuge}· \conf{\parkosz{czeebu}}{}¹ \parkosz{czoſɲek}. 
% 59 Łoś prawdopodobnie słusznie przypuszcza, że to błąd zamiast

% czaebr.

Quam differenciam ponemus, ut debite et differenter ista in 

scriptis exprimamus, sicut in voce differunt. Non est facile 

assignare, tamen, si placet, fiet talis differencia — si placet,


\ppreviouspageno=9
\plineno=0
\psublineno=1


\fullpreviouslines

{
\color{blue}
quia omnes voces et omnes caracteres ad placitum sunt inventoris

et sequencium.

}


\endinput


%%%%%%%%%%%%%%%%%%%%%%%%%%%%%%%%%%%%%%%%%%%%%%%%%%%%%%%%%%%%%%%%%%%%%%%%%%%%%%%%%%%%%%%%%%%
\end{document}



Non est facile 

assignare, tamen, si placet, fiet talis differencia — si placet,

60:
eo \y rkp.: quantocius.


\catcode `\^^M=5

  \newtip{48}{Łoś niesłusznie uważa, że \textit{bika} w obu wypadkach
    napisano błędnie zamiast \textit{ƀyka}. Przykłady są bowiem podane
    w~pisowni dotychczasowej dla pokazania jej niewystarczalności do
    zróżnicowania wyrazów \textit{bika} i \textit{byka}.} 

\obeylines




\newcommand{\margin}[1]{\annotatetextBlue{\{#1\}}{zapisy na marginesie}}


% \renewcommand{\over}[1]{\colorbox{blue!10}{\{#1\}}}

\renewcommand{\over}[1]{\annotatetextBlue{\{#1\}}{zapisy nad rządkami}}

% litery i wyrazy dodane, (których w tekście brak)
%\newcommand{\add}[1]{\colorbox{olive!10}{<#1>}}
\newcommand{\add}[1]{\annotatetextOlive{<#1>}{litery i wyrazy dodane, (których w tekście brak)}}

% litery i wyrazy zbędne
% \newcommand{\extra}[1]{\colorbox{magenta!10}{[#1]}}
\newcommand{\extra}[1]{\colorbox{magenta!10}{[#1]}}

% przekreślenia
% MATHEMATICAL LEFT WHITE SQUARE BRACKET' (U+27E6)
% 'MATHEMATICAL RIGHT WHITE SQUARE BRACKET' (U+27E7)
\newcommand{\overstr}[1]{\annotatetextMagenta{⟦#1⟧}{przekreślenia}}



%%% Local Variables: 
%%% mode: latex
%%% TeX-PDF-mode: t
%%% TeX-engine: luatex 
%%% TeX-master: "ParkoszLatin"
%%% default-input-method: "Parkosz-slash"
%%% End: 
