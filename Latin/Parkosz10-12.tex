\ppageno=12
\ppreviouspageno=11
\plineno=42
\psublineno=1

\newParkoszpage

{\relsize{-2}
\url{http://ebuw.uw.edu.pl/Content/215606/directory.djvu?djvuopts=&page=52&zoom=width&showposition=0.5,0.18}

\url{http://ebuw.uw.edu.pl/Content/215606/directory.djvu?djvuopts=&page=74&zoom=width&showposition=0.5,0.7}
}

\bigskip

\obeylines
\mono



\fullpreviouslines


{
\color{blue}

\indentK Item quociens iungitur cum his consonis: \parkosz{b}

\parkosz{n} \parkosz{w} \parkosz{d} \parkosz{l} \parkosz{r}, profertur et scribitur cum unco retorto,

}

\plineno=0

\fulllines

ut \parkosz{ṃaꝿda} \parkosz{ꝿlaſz} \parkosz{ꝿnew} \parkosz{ꝿrod}.

vocalibus, quamvis non profertur cum \parkosz{ꝿ} retorto, non oportet in

Polonico ideomate, \parkosz{g} ut per \parkosz{I} consonam scribitur, sed imo per \parkosz{g},

ut \parkosz{ġeʋa} \parkosz{ġeeɱi} \parkosz{ġeɱu} \parkosz{ġijɱ}, quod et in Latino optinet

\splitlines

veritatem.

\indentK Insuper hoc considerandum, quod apud Latinos \parkosz{I} consona

\fulllines

iungitur omnibus vocalibus, ut in exemplis suprapositis. Et omnes \hyphh{dic}{ciones}

\newtip{68}{W rkp.: \parkosz{v}.}
\hypht{dic}{ciones}, ubi \parkosz{g} coniungitur cum \parkosz{a} \parkosz{o} et \parkosz{u}¹, in quibus \parkosz{I} consona
% «s -yy rfep.; v.

concordat cum \parkosz{g} in voce, semper scribuntur tales dicciones per \parkosz{I}, et nunquam

per \parkosz{g}. Exemplum: \parkosz{Iacobus}, \parkosz{Iaṇua}; de \parkosz{o}: \parkosz{Iocus}, \parkosz{ioculator}; de \parkosz{u}

: \parkosz{ius}, \parkosz{iudicium}. Ubi autem \parkosz{I} consona iungitur \parkosz{e} vocali, ibi

% TO DO: do wyjaśnienia
aliquando scribunt \parkosz{I} consonam, aliquando \parkosz{g} litteram. Exemplum ut \parkosz{Ieroṇiṃus},

\parkosz{Iereṃias} et hoc fere semper in propriis nominibus. In appellativis

autem ut frequenter scribunt per \parkosz{G}, ut \parkosz{genus}, \parkosz{genu}. Dico notanter:

ut frequenter, quia aliquando eciam nomina appellativa scribuntur per

\parkosz{I}, ut \parkosz{ieiunium}, \parkosz{iecur}. Ubicunque autem \parkosz{g} coniungitur cum \parkosz{I},

necessario scribitur per \parkosz{g}, ut \parkosz{gymnaſium}, \parkosz{viginti} etc., quia non

% \splitlines

potest debite scribi per \parkosz{I} consonantem.


\indentK Iam superest \parkosz{S} litteram in suis differenciis expedire. Itaque

\fulllines

in primis nota, quod \parkosz{S} apud Latinos duos caracteres ex

more optinet. Primo scilicet per tractum in longum scribitur,

prout comuniter in principio et in medio diccionum ponitur, ut

\parkosz{ſoḷ}, \parkosz{ſaḷ}, \parkosz{ṃiſit}, \parkosz{ṃiffa}. Alio modo scribitur convolute, sicut

comuniter in fine ponitur diccionum, ut \parkosz{abbas}, \parkosz{mas}. Insuper

\newtip{69}{„Insuper nota” w rkp. powtórzone.}
nota¹, quod \parkosz{S} sine aliqua aspiracionis
% •* „Insuper nota” w rkp. powtórzone.

seu asperacionis speciali nota prolatum, aliquando secundum suam

propriam vocem ponitur sic, prout comuniter sonat in fine diccionum

positum, ut \parkosz{abbas}, \parkosz{mas}, \parkosz{lebes}. Sic eciam, quando ponitur in

principio diccionum, vocali aut consona subsequente. Exemplum primi: \parkosz{ſaaṃ}

\parkosz{seeṇ} \parkosz{Syɲ} \parkosz{Sova} \parkosz{Suuɱ} \parkosz{Søød}. Exemplum secundi: \parkosz{Struſ}

\splitlines

\catcode `\^^M=5
\newtip{70}{Ulanowski czytał
  \parkosz{sbik}, ale końcowa litera jest do \parkosz{k} zupełnie
  niepodobna. Prawdopodobniejsza jest lekcja \parkosz{sbit} lub \parkosz{sbil}.}
\obeylines

\parkosz{Sbil}¹ \parkosz{Sꝿaꝿa} \parkosz{Sɱøød} \parkosz{Spøød}.
% 70	Ulanowski czytał sbik, ale końcowa litera Jest do k zupełnie nie¬

% podobna. Prawdopodobniejsza jest lekcja sbit lub sbil.

\indentK Si autem mollitur aut

\fulllines
tenuatur, sicut premissum est, positum circa vocalem aut consonantem,

tunc perdit vim suam et transit in aliam litteram, quia

in \parkosz{z}. Exemplum primi: \parkosz{Zavada} vel \parkosz{zijġɱuṇt} \parkosz{zophia}

\parkosz{Zuzaṇṇa} \parkosz{zøøb}. Exemplum secundi: \parkosz{zɱuda} \parkosz{zɓyḷ} \parkosz{zdroʋ}.

Et propter hoc in nostro Polonico alphabeto post \parkosz{S} ponemus

\splitlines

Z propter vicinitatem vocis.

\indentK Aliquando tamen eciam S mollitur

\fulllines

et non perdit suum caracterem, ut positum inter duas vocales,

vim suam optinentes. Exemplum de \parkosz{a}: \parkosz{ɱaſal} \parkosz{caſa} (pro: \parkosz{coſa}) \parkosz{kaa}

\overstr{ſaal}\parkosz{ſaaḷ}. Unde quamvis ibi molliter proferatur, tamen proprio

caractere scribitur, ut in exemplis supra positis. Et si tunc vult

\newtip{71}{Może zamiast \parkosz{coſſa}.}
propriam vocem optinere, opportet ipsum geminare. Exemplum: \parkosz{caſſa}¹
% 71	Może zamiast co (Ja.

\newpage
\splitlines

\newtip{72}{Można też czytać \parkosz{Roſſa}.}
\parkosz{ṃiffa} \parkosz{Koſſa}¹.
% 72	Można też czytać Boffa.

\indentK Si igitur vellemus propriam differenciam

\fulllines

habere inter \parkosz{S} in propria voce prolatum et \parkosz{s} molle, ut propositum,




\ppreviouspageno=13
\plineno=0

\fullpreviouslines

{
\color{blue}

inter duas vocales, scribamus primum convolutum, sicut solitum

est poni in fine diccionum.

}


\endinput


%%%%%%%%%%%%%%%%%%%%%%%%%%%%%%%%%%%%%%%%%%%%%%%%%%%%%%%%%%%%%%%%%%%%%%%%%%%%%%%%%%%%%%%%%%%


\catcode `\^^M=5

  \newtip{48}{Łoś niesłusznie uważa, że \textit{bika} w obu wypadkach
    napisano błędnie zamiast \textit{ƀyka}. Przykłady są bowiem podane
    w~pisowni dotychczasowej dla pokazania jej niewystarczalności do
    zróżnicowania wyrazów \textit{bika} i \textit{byka}.}

\obeylines






\newcommand{\margin}[1]{\annotatetextBlue{\{#1\}}{zapisy na marginesie}}


% \renewcommand{\over}[1]{\colorbox{blue!10}{\{#1\}}}

\renewcommand{\over}[1]{\annotatetextBlue{\{#1\}}{zapisy nad rządkami}}

% litery i wyrazy dodane, (których w tekście brak)
%\newcommand{\add}[1]{\colorbox{olive!10}{<#1>}}
\newcommand{\add}[1]{\annotatetextOlive{<#1>}{litery i wyrazy dodane, (których w tekście brak)}}

% litery i wyrazy zbędne
% \newcommand{\extra}[1]{\colorbox{magenta!10}{[#1]}}
\newcommand{\extra}[1]{\colorbox{magenta!10}{[#1]}}

% przekreślenia
% MATHEMATICAL LEFT WHITE SQUARE BRACKET' (U+27E6)
% 'MATHEMATICAL RIGHT WHITE SQUARE BRACKET' (U+27E7)
\newcommand{\overstr}[1]{\annotatetextMagenta{⟦#1⟧}{przekreślenia}}



%%% Local Variables:
%%% mode: latex
%%% TeX-PDF-mode: t
%%% TeX-engine: luatex
%%% TeX-master: "ParkoszLatin"
%%% default-input-method: "Parkosz-slash"
%%% End:
