\ppageno=03
\ppreviouspageno=02
\plineno=0
\psublineno=1

\newParkoszpage

{\relsize{-1}
\url{http://wbl.klf.uw.edu.pl/13/2/iParkosz.djvu?djvuopts=&page=43&zoom=width&showposition=0.5,0.18}
}


% TO DO?:
% http://tex.stackexchange.com/questions/30930/how-to-output-a-counter-with-leading-zeros
\fulllines
  
  \newtip{1}{Napisane u samej góry strony.}
  Iesus Christus¹
% 1	Napisane u samej góry strony.

  \dc{P}ugna pro patria, quia ipsam
% Why it is needed????:
  \advance\plineno by 1

\newtip{2}{Motto przedmowy zostało prawdopodobnie zapożyczone
z jakiegoś poematu średniowiecznego, gdyż — jak wskazał R. Ganszyniec
— widoczny w nim jest układ heksametryczny. Wyrażenie „pugna pro
patria” wywodzi się z przedmowy do szkolnego podręcznika pt. Disticha
Catonis.}

defendere laus est meritoria.¹
% TODO:? Transkribus średnik
% 2	Motto przedmowy zostało prawdopodobnie zapożyczone z jakiegoś

% poematu średniowiecznego, gdyż — jak wskazał R. Ganszyniec — wido¬

% czny w nim jest układ lieksametryczny. Wyrażenie „pugna pro patria”

% wywodzi się z przedmowy do szkolnego podręcznika pt. Disticha Catonis.
%


\newtip{3}{Aristotelis \textit{Topica} 3, 1, 116a.}


  
Patria hic notat comunitatem, comunitas diuturnitatem, \hyphh{di}{uturnitas}

\hypht{di}{uturnitas} eligibilitatem, ut patet tercio \textit{Topicorum}¹. Hinc est, quod
% 3	Aristotelis Topica 3, 1, 116a.

inconcussa patrum ac philosophorum sanxit autoritas: Sacius reipublice

fore insudandum, quam alias cuilibet privato comodo adherendum, \hyphh{sal}{vata}

\hypht{sal}{vata} enim comunitate, salvatur eius pars, et non e contra. Nec obviat Philosophus in

\textit{Predicamentis} dicens: Non existentibus primis substanciis, id est singularibus, impossibile est aliquid

\newtip{4}{Aristotelis \textit{Cathegoriae}.}


horum, id est comune, remanere¹, quia hoc intelligitur naturaliter vel, ut libet,
% 4	Aristotelis Cathegoriae.

\splitlines

moraliter. 

\indentK Unde et Romanis diu consuetum erat triumphantem pro republica

\fulllines

variis honoribus ac multis muneribus preficere, ut patet eorum gesta inspicere

volentibus. Et hoc ipsum tangit Theodolus in suis \textit{Eglogis} dicens

\splitlines

\newtip{5}{IV — tak czyta Ulanowski niezbyt jasny zapis.}
\margin{Egloga \conf{IV}{15}}¹

\indentKcyt Excedit laudes hominum, qui primus agones
% 5	IV — tak czyta Ulanowski niezbyt jasny zapis.

\indentKcyt Instituit fieri sub vertice

\splitlines

\indentKcyt \phantom{Instituit fieri sub vertice }montis Olimpi.	

\indentKcyt Aurea victrices obnubit laurea crines,

\indentKcyt Ducit pompa domum,

\splitlines

\indentKcyt \phantom{Ducit pompa domum, }sequitur confusio victum.

\indentK Hac eciam saluberrima ratione exigente et quasi

\fulllines

ex necesitate inducente nedum apud fideles verumeciam et circa gentiles, ut

refert Aristotiles, quocienscumque quempiam pro comunitate certantem \hyphh{con}{tingit}

\hypht{con}{tingit} ponere vitam, ipsum honorifice sepultum in suis superstitibus, puta

filiis et filiabus, fovere indesinenter favoribus ratum atque gratum

\newtip{6}{Łoś poprawia na: defendentis\ldots militis.}
comunitati extitit. Claret ergo patriam \conf{defendenti}{} gloriam attingere \conf{militi}{}.¹
% 6	Łoś poprawia na: defendentis... militis.

\indentP Nos itaque monitu talium inducti, non quod intendamus laudi, quoniam hoc ipsum

solius Dei est, sed inspicientes ad nutum, ex quo reipublice agitur comodum,

censuimus ex causis infra dicendis paternum idioma, quod notabiliter traximus

ex Polonorum lingua, fore caracteribus latinis cum paucis differenciis appositis

\newtip{7}{Wyraz oznaczający błędy językowe, por. u Juvenalisa
  (Sat. VII 51): scribendi cacoethes.}
scribendum, ut in hoc, quantum ad presens attinet, contra cacetem¹ id est malum
% 7	Wyraz oznaczający błędy językowe, por. u Juvenalisa (Sat. VII 51):
% scribendi cacoetbes.

et insufficientem usum scripture polonice laborantes videamur patriam comodose

\splitlines{}
defendere et ad sufficientem modum scribendi inducere.

\indentK Sed ante

\fulllines{}

exordium materie intente phas exordium ponere, ut materie presentis necessitatem

audientes eam libencius amplecterentur et odio eidem obviantes audacter

\splitlines{}
repellerentur. Et racio ponetur, cur eadem frui utiliter debeatur.

\indentK Sit ergo propositi



\fulllines{}

nostri pro themate hoc verbum, quod scribit Plato, divinissimus philosophus, in \textit{Thymeo},

\newtip{8}{R. Glanszyniec zwrócił uwagę, że ten niedokładny
  cytat z Platona zgadza się z cytowaniem go przez XIII-wiecznego
  gramatyka Goswina de Marbis.}

dicens: Ad hoc datus est nobis sermo, ut presto iudicia mentis nostre fierent¹.
% 8	R. Glanszyniec zwrócił uwagę, że ten niedokładny cytat z Platona
% zgadza się z cytowaniem go przez XIII-wiecznego gramatyka Goswina
% de Marbis.

Ubi est advertendum, quia in hoc verbo: iudicia, tangitur humana civilitas,

de qua Aristotiles primo Politicorum dicit: Homo est animal politicum, id est civile, masuetum

et domesticum, suorum conceptuum alteri comunicativum et hoc per signa vocis,


\newtip{9}{Peri hermenias.}
ut idem notat primo \conf{Pery}{} \conf{ermenias}{}¹, dicens: Sunt ergo ea, que sunt in voce, earum, que sunt
% 9	Peri hermenias.

in anima, passionum note. Et inferius eciam ibidem innuit, quod voces sunt signa

conceptuum et scripta — vocum, ex quarum vocum composicione sermo fit hominum. Unde

sermo nihil aliud est, quam instrumentum vocale, quo mentis nostre conceptum

\splitlines{}

\newtip{10}{Cytaty z Arystotelesa w różnych wersjach powtarzają
  się w dziełach średniowiecznych z zakresu gramatyki, np. przytoczony
  cytat w traktacie Generalis doctrina de modis significandi
  grammaticalibus: Ideo dicit Aristoteles primo Peri hermenias: Ea,
  que sunt in voce, sunt note, id est signa, earum passionum, que sunt
  in anima, id est conceptuum seu actuum intellegendi (\so{R. Ganszyniec},
  \textit{Metrificale Marka z Opatowca i traktaty gramatyczne XIV i XV wieku},
  Studia staropolskie pod red. K. Budzyka, t. VI, Wrocław 1960,
  s. 150.}
exprimimus alteri cognoscendum¹.
% 10	Cytaty z Arystotelesa w różnych wersjach powtarzają się w dziełach średniowiecznych z zakresu gramatyki, np. przytoczony cytat
% w traktacie Generalis doctrina de modis significandi grammaticalibus:
% Ideo dicit Aristoteles primo Peri hermenias: Ea, que sunt in voce, sunt
% note, id est signa, earum passionum, que sunt in anima, id est conceptuum seu actuum intellegendi (K. Ganszyniec, Metrificale Marica
% z Opatowca i traktaty gramatyczne XIV i XV wieku, Studia staropolskie
% pod red. K. Budzyka, t. VI, Wrocław 1960, s. 150.

\indentK  Et talis sermo est multifarius, multiplicitas

\fulllines{}
\newtip{11}{W rkp. błędnie: diversitatis.}
autem sermonis et \conf{diversitas}{}¹ originaliter processit ex divisione lingue
% 11	W rkp. błędnie: diversitatis.

filiorum Noe, turrim edificancium secundum illud Scripture: Descendamus

et confundamus linguam filiorum Noe, et non intelligat unusquisque vocem proximi

sui. Atque ita divisit eos ex illo loco in universam terram. Et cessaverunt

edificare civitatem, et idcirco vocatum est nomen loci illius Babel, quia

\newtip{12}{\textit{Genesis} XI 7-9. W cytacie są niewielkie różnice w stosunku do
Wulgaty.}
ibi divisum est labium universe terre¹. Cuius quidem divisi labii, id est sermonis,
% 12	Genesis XI 7-9. W cytacie są niewielkie różnice w stosunku do
% Wulgaty.

diverse naciones diversas habent figuras ac litteras seu caracteres,






%%% Local Variables: 
%%% mode: latex
%%% TeX-master: "ParkoszLatin"
%%% TeX-engine: luatex 
%%% End: 
