\ppageno=8
\ppreviouspageno=7
\plineno=48
\psublineno=1

{\relsize{-1}
\url{http://wbl.klf.uw.edu.pl/13/2/iParkosz.djvu?djvuopts=&page=48&zoom=width&showposition=0.5,0.18}

\url{http://wbl.klf.uw.edu.pl/13/2/iParkosz.djvu?djvuopts=&page=67&zoom=width&showposition=0.5,0.67}
}

\bigskip

\obeylines
\mono


\fullpreviouslines


{
\color{blue}
Nec, ut pretactum est, per vocales longas aut

}

\plineno=0


\splitlines

additas differencia notari potest.

\indentP \margin{\parkosz{ƀ}, b}. Si igitur libet, \add{sit} \parkosz{ƀ} grossum sine unco et \hyphh{qua}{dratum}

\fulllines

\hypht{qua}{dratum}, ut sic \parkosz{ƀ}, quod eciam musici vocant \textit{b} durum, et \parkosz{ɓ} molle cum 

unco superiori et inferius rotundum, ut sic \parkosz{ɓ}, quod musici eciam \parkosz{ɓ} molle 

\newtip{44}{Winno być: ɓika ɓiġe lub ƀika ɓiġe.}

vocant. Exemplum: \parkosz{ƀaƀø} \parkosz{bik} \parkosz{ƀodze}. Exemplum secundi: \parkosz{Beṅeck} \conf{\parkosz{byka}}{} \conf{\parkosz{ƀiġe}}{}¹. 
% 44	Winno być: bika biġę lub bika 6iġe.

\margin{ff}. \textit{ff} grossum scribamus geminum vel geminatum, \textit{f} molle simplum. 

Exemplum primi: \parkosz{ffaaɬ} \parkosz{ffaſth} \parkosz{ffitaa}. Exemplum secundi: \parkosz{fiſth} \parkosz{fiġi}. 

\margin{l}. Item ɬ spissum seu grossum fiat sine unco. Exemplum: \over{ɬap}  \parkosz{ḷapka}

\parkosz{ḷekce} \parkosz{ḷiſzeġo} \parkosz{ḷoſze} \parkosz{ḷudzy} \parkosz{løthka}. Molle cum unco superiori. 

\newtip{45}{Te dwa wyrazy można czytać: \textit{ɬoſch} \textit{ɬeſch} lub \textit{ɬoſth} \textit{ɬeſth}, a także (tak Ulanowski) \textit{ɬoſth} \textit{ɬeſch}.}

\splitlines
Exemplum: \parkosz{ɬaaſz} \parkosz{ɬis} \conf{\parkosz{ɬoſch}}{} \conf{\parkosz{ɬeſch}}{}¹ \parkosz{ɬudze} \parkosz{ɬąnkawką}.
% 45	Te dwa wyrazy można czytać: lojch lefch lub lojth le/th, a także

% (tak Ulanowski) lojth lejch.

\margin{ɱ ṁ}. \parkosz{ɱ} spissum 

\fulllines
cum cauda in tercio pede, sicut in fine diccionum poni solet, ut sic:

\parkosz{ɱara} \parkosz{ɱeġo} \parkosz{ɱiġe} \parkosz{ɱoġe} \parkosz{ṁødro} \parkosz{ɱuſchɱø}. \textit{M} autem molle sine 

\splitlines
cauda, ut \parkosz{ṁaal} \parkosz{ṁecz} \parkosz{ṁikolai} \parkosz{ṁood}.

\margin{ɲ ṅ}. Sic eciam \textit{ɲ} spissum 

\fulllines

cum cauda, sicut in fine diccionum poni solet, ut \parkosz{ɲapaſcz} \parkosz{ġɲath}

\parkosz{ɲaløcz} \parkosz{ɲoc} \parkosz{ɲos} \parkosz{ɲądza}. Molle sine cauda, ut \parkosz{ṅevaſta}

\parkosz{ṅeve} \parkosz{ṅiczs}. Erit autem differencia inter \parkosz{ɲ} spissum seu grossum et

\textit{ij} duplex, quia cauda \parkosz{ɲ} trahitur in rectum, cauda autem \textit{ij}

flectitur ad sinistrum. Insuper supra \textit{ij} ponitur duplex punctus ad

\splitlines
differendam \textit{ɲ} spissi.

% Missing character: There is no ᵽ (U+1D7D) in font "DejaVu Sans Mono"!
\margin{ᵽ p}. \parkosz{Ᵽ} spissum fiat quadratum, ut \parkosz{ƀ} grossum. 

\fulllines

\parkosz{ƥ} molle fiat rotundum. Exemplum primi: \parkosz{Ᵽaṅ} \parkosz{Ᵽavel}. \parkosz{Ᵽooġe} \parkosz{\hyphh{Ᵽyſch}{ṅo}}

\newtip{46}{Zaczyna się tutaj pojawiać p miękkie pisane nieco inaczej niż zwykle \textit{Ƥƥ}.}

\hypht{Ᵽyſch}{ṅo}. Exemplum secundi: \parkosz{Ƥotr} \parkosz{Ƥivo} \parkosz{Ƥiġe} \parkosz{Ƥilṅo}. \parkosz{Ƥech} \parkosz{Ƥecze} \parkosz{Ƥeceṅą}¹.
% 46	Zaczyna się tutaj pojawiać p miękkie pisane nieco inaczej niż zwykle

% p, mianowicie Pp.
Nec autem modo erit necesse post illas litteras molliter prolatas 

\textit{ij} duplex scribere, sicut hactenus scripsimus in omnibus predictis

consonantibus. Nam sic scribimus \parkosz{byl} id est percussit per duplex \textit{y},

volentes exprimere molle \parkosz{ɓ}, sed apparet manifeste, quod eciam circa ƀ

\splitlines

grossum seu spissum ponitur duplex \textit{ij}, ut \parkosz{ƀyl} id est fuit.

Qualis igitur inter hec

\fulllines
erit differencia ? Et — ut appareat sic, secundum quod alias scribebatur, differendam inter dicciones 

cadere non posse — ecce si scribatur more predicto \parkosz{byaal} id est albus et \parkosz{ɓyaal}

id est percuciebat, in scripto non erit differencia, cum in voce et significato differencia sit. Nunc ergo

scribatur \parkosz{ɓaaɬ} pro albo et \parkosz{ɓyaal} pro percuciebat. Et sic plana erit differencia.

\newtip{47}{Winno być raczej: laterinorum, lub: laterariorum.}
Sic eciam \parkosz{bika} id est instrumentum latarnorum¹, ubi est \parkosz{ɓ}  molle et \textit{i} breviatur, 
% 47	Winno być raczej: laterinorum, lub: laterariorum.

% http://tex.stackexchange.com/questions/57318/is-there-a-counterpart-antidote-for-obeylines
% spacja!

\catcode `\^^M=5

  \newtip{48}{Łoś niesłusznie uważa, że \textit{bika} w obu wypadkach
    napisano błędnie zamiast \textit{ƀyka}. Przykłady są bowiem podane
    w~pisowni dotychczasowej dla pokazania jej niewystarczalności do
    zróżnicowania wyrazów \textit{bika} i \textit{byka}.} 

\obeylines

et \parkosz{bika} id est taurum vel \parkosz{bika}¹ id est mugit, ubi est ƀ grossum et \textit{i} longatur,
% 48 Łoś niesłusznie uważa, że biba w obu -wypadkach napisano błędnie

% zamiast byka. Przykłady są bowiem podane w pisowni dotychczasowej

% dla pokazania jej niewystarczalności do zróżnicowania wyrazów bika

% i byka.

non potest igitur differencia subsistere. Melior est itaque differenda ƀ quadrare pro spisso et \hyphh{rotun}{dare}

\splitlines
\hypht{rotun}{dare} pro molli.

\indentK Sic de \textit{ll}: si per duplex \textit{i} voluerimus \parkosz{ḷ} molle designare,

\fulllines
in aliquibus diccionibus bene conveniet, in aliquibus vero minime, ut \parkosz{ḷyſchka} id est vulpis, 

\textit{ḷyſſka} id est eruca, \parkosz{ḷyſth} id est folium, sed cum ex superius dictis vocalis geminata

deberet produci, in omnibus tamen premissis corripitur. Ecce prima ineptitudo. \hyphh{De}{nique}

\hypht{De}{nique}, ubi in fine diccionum \parkosz{ɬ} molle seu tenue ponitur, si sibi duplex

\textit{y} adiciatur, faciet significati confusionem. Exemplum: \parkosz{ſtaal} id est calibs. Si post \parkosz{ɬ} duplex \textit{y}

poneretur, stabit \parkosz{ſtaaly}, quod significat steterunt. Ubi erit nostra calibs ? Si autem sine 

y scribatur, erit \parkosz{ſtal} id est stetit.  Ecce alia ineptitudo. Melior ergo erit differencia,

ut ḷ grossum sive spissum scribamus sine unco, sicut baculum, ut \parkosz{ſtaal} 

\parkosz{lyſſy} id est stetit calvus, \parkosz{ɬ} molle seu tenue cum unco et sine addicione \textit{y}, 

\splitlines
ut \parkosz{ſtaɬ} id est calibs, \parkosz{ɬiſth} id est folium, \parkosz{ɬuud} id est populus.

\indentK Eodem modo de \textmd{m}. Olim 

\fulllines
\parkosz{Ṁikolai} \parkosz{ṁiſſa} \parkosz{ṁiḷaa} \parkosz{ṁyaaḷ} id est adeps vel verbum id est habuit

per duplex \textit{y} scribebant, volentes tenue \textit{m} denotari. Sed iam est ostensum,

quod \textit{m} posita circa vocales et consonantes non potest per \textit{y} duplex \textit{m} \hyphh{te}{nue}

\hypht{te}{nue} ab \parkosz{ɱ} spisso differre, ut \parkosz{ṁika} id est Nicolaus, \parkosz{ɱikaa} id est frequenter

trahit. Est igitur melior differencia, ut \parkosz{ɱ} molle seu tenue scribatur

\newtip{49}{To \textit{ɱykaa} jest tu niepotrzebne.}
sine cauda, ut sic: \parkosz{ṁika}· \parkosz{ṁikoḷay}· \parkosz{ ɱykaa}¹· \parkosz{ṁyaſto} \parkosz{ṁaaḷ}. Et
% ** To TPLykaa jest tu niepotrzebne.


\endinput


%%%%%%%%%%%%%%%%%%%%%%%%%%%%%%%%%%%%%%%%%%%%%%%%%%%%%%%%%%%%%%%%%%%%%%%%%%%%%%%%%%%%%%%%%%%

\newcommand{\margin}[1]{\annotatetextBlue{\{#1\}}{zapisy na marginesie}}


% \renewcommand{\over}[1]{\colorbox{blue!10}{\{#1\}}}

\renewcommand{\over}[1]{\annotatetextBlue{\{#1\}}{zapisy nad rządkami}}

% litery i wyrazy dodane, (których w tekście brak)
%\newcommand{\add}[1]{\colorbox{olive!10}{<#1>}}
\newcommand{\add}[1]{\annotatetextOlive{<#1>}{litery i wyrazy dodane, (których w tekście brak)}}

% litery i wyrazy zbędne
% \newcommand{\extra}[1]{\colorbox{magenta!10}{[#1]}}
\newcommand{\extra}[1]{\colorbox{magenta!10}{[#1]}}

% przekreślenia
% MATHEMATICAL LEFT WHITE SQUARE BRACKET' (U+27E6)
% 'MATHEMATICAL RIGHT WHITE SQUARE BRACKET' (U+27E7)
\newcommand{\overstr}[1]{\annotatetextMagenta{⟦#1⟧}{przekreślenia}}



%%% Local Variables: 
%%% mode: latex
%%% TeX-PDF-mode: t
%%% TeX-engine: luatex 
%%% TeX-master: "ParkoszLatin"
%%% default-input-method: "Parkosz-slash"
%%% End: 
