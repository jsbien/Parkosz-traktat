
\ppageno=7
\ppreviouspageno=6
\plineno=40
\psublineno=1

\newParkoszpage


{\relsize{-2}
\url{http://ebuw.uw.edu.pl/Content/215606/directory.djvu?djvuopts=&page=47&zoom=width&showposition=0.5,0.18}

\url{http://ebuw.uw.edu.pl/Content/215606/directory.djvu?djvuopts=&page=65&zoom=width&showposition=0.5,0.64}
}

\fullpreviouslines


{
\color{blue}
Ponitur eciam ut simplex consonans et tunc

}

\fulllines

\plineno=0

\newtip{29}{Lekcja nie jest pewna, Ulanowski czytał: \parkosz{vilne}.}

aliquando grossatur, aliquando molliter profertur. Exemplum primi: \parkosz{ʋiklad} \parkosz{ʋiġe}

\parkosz{ʋya}. Exemplum secundi: \parkosz{ʋila} \conf{{\parkosz{vilue}}}{}¹. Quando igitur \textit{u} vocalis simpla ponitur,
% 29	Lekcja nie jest pewna, Ulanowski czytał: vilne.

eodem modo sicut u scribitur, in superiori parte apertum, in inferiori ligatum.

Quando autem longatur, tunc geminatur. Exemplum primi in \parkosz{uije} \parkosz{uyass}.


\newtip{30}{Winno być: \parkosz{ſzuuwaṃ}, \parkosz{dṃuuchaa}.}

Exemplum secundi: \parkosz{ſzvwaṃ} \parkosz{dṃuchaa}¹. Quando est consona et grossatur, tunc
% 30 Winno być: fsuuwam, dmuuchaa.

de superius primi cornu tractus ducatur, ut \parkosz{ʋaal} \parkosz{ʋyl} \parkosz{ʋiklad}.

Quando vero mollitur, tunc planis et equis cornibus scribatur, ut sic: \parkosz{vila}

\splitlines

\parkosz{vidzaall} \parkosz{viṇo}.

\indentK Hac differencia habita inter \textit{v} consonantem mollem et

\fulllines

grossam, non erit necesse ponere duplex \textit{ij}, ut olim ponebatur,  ut \parkosz{vyatr}·

\parkosz{wyeġe}· \parkosz{wyoḟṇa}· \parkosz{vyoṇczek}, quia non esset differenda inter \parkosz{vial} et \parkosz{vaal}

id est flavit, sed sic scribatur molle \textit{v}: \parkosz{vaal}· \parkosz{vathr}· \parkosz{veġe}· \parkosz{voſṇa}·

\parkosz{vøczek}· \parkosz{vvaal}. Peccatum est enim fieri per plura, quod eque bene potest fieri per

\splitlines

pauciora.

\indentK Ille autem sex littere: \textit{b} \textit{f} \textit{l} \textit{m} \textit{n} \textit{p} sub eadem vocis \hyphh{reten}{tione}


\newsplitline

\hypht{reten}{tione} et quantitate nunc grosse, nunc molliter proferuntur.

\indentK \margin{De b}. Exemplum primi

\fulllines

\newtip{31}{Winno być: \parkosz{ƀyk}.}

de \parkosz{ɓ}· \parkosz{ɓyk}· \parkosz{ɓith} id est percussus. Exemplum secundi: \overstr{\textit{b}} \parkosz{bil} id est fuit, \parkosz{ɓyk}¹ id est
% 31 31	Winno być: byk.

thaurus. Nec ex parte vocalis geminate et longate potest esse

differentia. Nam idem \parkosz{ɓ} molle et grosse prolatum, cum \parkosz{y} longo et \parkosz{i}

brevi reperitur. Licet enim, ubi \parkosz{ij} longatur, merito \parkosz{b} deberet grossari,

et ubi \parkosz{i} breviatur, deberet iuste molliri, ut in exemplis suprapositis:

\newtip{32}{Jeden z tych przykładów winien być napisany przez \parkosz{ƀ}, drugi przez \parkosz{ɓ}}

\parkosz{ɓyk} et \parkosz{ɓik}¹ est tamen reperire dictiones, in quibus \parkosz{i} breviatur, et tamen \parkosz{ɓ}
% 32	Jeden z tycli przykładów winien być napisany przez b, drugi przez ó.

nunc mollitur, nunc grossatur, ut \parkosz{bith} id est habitatio, \parkosz{ɓith}

id est percussus. In his dictionibus \parkosz{i} breviatur, et tamen \parkosz{ɓ} mollitur et

grossatur. Aliquando e converso, \parkosz{y} producitur, et tamen \parkosz{ɓ} grosse et

\splitlines

molle profertur, ut \parkosz{ɓyl} id est percussit, \overstr{ɓ} \parkosz{byl} id est fuit.

\indentP \margin{ff}. Sic \parkosz{ff} nunc

\fulllines

grosse, nunc molle profertur, ut \parkosz{ffaal} et est proprium nomen, \parkosz{faal}

id est movit. Nec per vocales additas differre possunt. Nam \parkosz{ffaal}

et faal utrobique \parkosz{aa} longum seu geminatum, et \parkosz{f} differt in voce.

Nam si diceretur, quod vocalis longata seu producta faceret consonantem


\newtip{33}{W rkp.: grossiori.}
\secondtip{34}{W rkp.: molliori.}

litteram \conf{grossari}{}¹, et breviatam \conf{molliri}{}², sicut de \parkosz{ɓ} dictum est,
% 33	W rkp.: grossiori.
% 34	w rkp.: molliori.

hoc apparet falsum in exemplis statim positis: \parkosz{ffaaɬ} \parkosz{faal}. Imo

alicubi vocalis producitur et \parkosz{ḟ} mollitur, ut \parkosz{ḟyſth}, et e converso,

\newtip{35}{Winno być: \parkosz{ffitaa}.}
vocalis corripitur, et \parkosz{ff} grossatur, ut \parkosz{ffytaa}¹, et iterum \parkosz{i} breviatur,
% 35	Winno być: ffitaa.

\newtip{36}{Winno być: \parkosz{ḟyjth}.}

\splitlines

ut \conf{\parkosz{ḟiſth}}{}¹ \overstr{ffy} \parkosz{ffita} \parkosz{ḟiġi}.
% 38 Winno być: fyjth.

\newtip{37}{Winno być: \parkosz{ɬl} lub \parkosz{lɬ}.}

\indentK \margin{ll}¹. Eodem modo \textit{l} nunc grossatur, nunc \hyphh{mol}{litur}
% 57 Winno być: li lub 11.

\fulllines

\hypht{mol}{litur}, ut \parkosz{ɬiḟt} id est littera vel folium, \parkosz{liſth} ut est pars pedis, \parkosz{ɬis} id est

vulpis, \overstr{ɬi} \parkosz{liſz} id est calvus: ecce ceteris paribus nunc grossatur,

nunc mollitur sic et in fine positum. Exemplum: \parkosz{Staaɬ} id est calibs,

\splitlines

\parkosz{ſtaal} id est stetit.

\indentK \margin{m}. Eodem modo \textit{m} grossatur et mollitur circa easdem

\fulllines

vocales et consonantes positum. Exemplum: \parkosz{ṃika} id est Nicolaus, \parkosz{ɱikaa}

\newtip{38}{Winno być: \parkosz{ġyɱ}.}

id est trahit. Et in fine diccionis positum. Exemplum: \parkosz{dyɱ}· \parkosz{ġroɱ}· \conf{\parkosz{ġyṃ}}{}¹
% 38 Winno być: ġyrⱥ.

\splitlines

\newtip{39}{Winno być: \parkosz{ġym}.}

id est eis, \parkosz{ġyɱ}¹ id est tene, \parkosz{przyṃ} id est suscipe.

\indentK \margin{n}. Consimiliter \textit{n} grossatur

\fulllines

et mollitur. Exemplum de grosso: \parkosz{ŋiſſki} id est de civitate Nissa. Exemplum

\splitlines

\newtip{40}{Winno być :ſyŋ}

de molli: \parkosz{ṇiḟki} id est declivis. Sic eciam in fine: \conf{\parkosz{ſyṇ}}{}¹, \parkosz{kooṇ}.
% 40	Winno być: fy^.


\indentK \margin{p}. Consimiliter


\fulllines

\textit{p} grosse et molliter profertur: \parkosz{paṇ}· \parkosz{paʋel}· \parkosz{piſchṇo} \parkosz{piġe}· \parkosz{pivo}.

\newtip{41}{Ulanowski czytał: \parkosz{povec}.}
\secondtip{42}{Nie zróżnicowano pisowni \textit{p}.}

\parkosz{pothr}· \conf{\parkosz{povee}}{}¹· \conf{\parkosz{popu}}{}²·
% 41	Ulanowski czytał: povec.
% 4J Nie zróżnicowano pisowni p.

Quas igitur differencias his vocibus harum litterarum attribuemus? Si

aliquas differencias ponere voluerimus, non parum difficultatis

\newtip{43}{Winno być: habetur lub: habemus.}

\conf{habet}{}¹, si nullas, multum erroris sequi necesse est. Nam eadem

figura litteram molliter et grosse prolatam scribere non parvum

errorem parit. Nec, ut pretactum est, per vocales longas aut

\ppreviouspageno=8
\plineno=0

\fullpreviouslines


{
\color{blue}

additas differencia notari potest.

}

\newParkoszpage

%%% Local Variables:
%%% mode: latex
%%% TeX-PDF-mode: t
%%% TeX-engine: luatex
%%% TeX-master: "ParkoszLatin"
%%% default-input-method: "Parkosz-slash"
%%% End:
