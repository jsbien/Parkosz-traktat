% pdf2djvu --no-nfkc  ParkoszLatin4djvu.pdf -o ParkoszLatin4djvu.djvu
\documentclass[dvipsnames,12pt]{report}
\usepackage{polyglossia}
\setdefaultlanguage{polish}

\usepackage[a4paper,landscape]{geometry}
\usepackage[doublespacing]{setspace}
\usepackage{fontspec}

\usepackage{graphicx}
\usepackage{xcolor}

\usepackage{soul}


\usepackage{longtable}
\usepackage[pdfencoding=unicode]{hyperref}

\usepackage{fancyvrb}

\usepackage{relsize}

\usepackage{csquotes}

% XeTeX
\setmainfont[Mapping=tex-text]{DejaVu Serif}
\setsansfont[Mapping=tex-text]{DejaVu Sans}
\setmonofont{DejaVu Sans Mono}

%\font\Quivira="Quivira"
\font\Quivira="Symbola"


%\font\mono="DejaVu Sans Mono"
\font\mono="DejaVu Sans"
%\font\mono="DejaVu Sans"
%\font\mono="FreeSerif"

%\font\fontCardo="Cardo"
\font\fontCardo="Symbola"

\newcommand{\Cardo}[1]{{\fontCardo #1}}


% http://scholarsfonts.net/cardofnt.html

\newcommand{\symbolParkosz}[1]{{\largeParkosz\symbol{#1}}}

\newcommand{\sParkosz}[1]{{\fontspec[Scale=3.0]{Parkosz}\symbol{#1}}}

\newcommand{\parkosz}[1]{\index{#1}\colorbox{lime}{\textit{\Quivira #1}}}

% za ciemny:
% \newcommand{\almanica}[1]{\colorbox{lightgray}{\textit{#1}}}
\newcommand{\almanica}[1]{\colorbox{pink}{\textit{#1}}}



%hh hyphenation head
%ht hyphenation tail

% dc dropped capital

%\newcommand{\dc}[1]{{\relsize{3}\colorbox{black!15}{\textbf{#1}}}}
% size "migrates" into line number
\newcommand{\dc}[1]{{\colorbox{black!15}{\textbf{#1}}}}

% The color good for screen, bad for printing:
\newcommand{\hyphh}[2]{#1}
%\newcommand{\hyphh}[2]{#1\textcolor{RoyalPurple}{#2}}
\newcommand{\hypht}[2]{\fbox{#2}}


% marginalia
% zapisy na marginesie i nad rządkami
\newcommand{\margin}[1]{\colorbox{blue!10}{\{#1\}}}
%\newcommand{\margin}[1]{\annotatetextBlue{\{#1\}}{zapisy na marginesie}}

\renewcommand{\over}[1]{\colorbox{blue!10}{\{#1\}}}
%\renewcommand{\over}[1]{\annotatetextBlue{\{#1\}}{zapisy nad rządkami}}

% litery i wyrazy dodane, (których w tekście brak)
\newcommand{\add}[1]{\colorbox{olive!10}{<#1>}}

% litery i wyrazy zbędne
% \newcommand{\extra}[1]{\colorbox{magenta!10}{[#1]}}
\newcommand{\extra}[1]{\colorbox{magenta!10}{[#1]}}

% przekreślenia
% MATHEMATICAL LEFT WHITE SQUARE BRACKET' (U+27E6)
% 'MATHEMATICAL RIGHT WHITE SQUARE BRACKET' (U+27E7)
%\newcommand{\overstr}[1]{\annotatetextMagenta{{\Quivira⟦}#1{\Quivira⟧}}{przekreślenia}}
\newcommand{\overstr}[1]{\colorbox{magenta}{{\Quivira⟦}#1{\Quivira⟧}}}

% mało fontów:
% RIGHT WHITE SQUARE BRACKET (U+301B)


% wątpliwe odczytania
% confidence
% nie działa parametr specyfikacji koloru - może jakiś głupi błąd
\newcommand{\conf}[2]{\textcolor{orange}{#1}}


% wcięcia Kucały
% akapit:
\newcommand{\indentK}{\hskip2em}
% cytat
\newcommand{\indentKcyt}{\hskip5em}

% wcięcia Parkosza:
\newcommand{\indentP}{\colorbox{yellow}{\hskip2em}}

\parindent0pt

\DefineVerbatimEnvironment%
{VerbatimLatin}{Verbatim}
{commentchar=\%,numbers=left,numberblanklines=false,
commandchars=\\\{\}}


\newcount\ppageno
\newcount\ppreviouspageno


\newcount\plineno
\newcount\psublineno

\newcount\pnoteno
\newcount\psecnoteno

% odstęp na oko - lepiej zmierzyć i wstawić
\def\fulllines{\everypar{\advance\plineno by 1\llap{\the\ppageno-\ifnum\plineno<10 0\fi\the \plineno \hskip 1.5em}}}

\def\fullpreviouslines{\everypar{\advance\plineno by 1\llap{\the\ppreviouspageno-\ifnum\plineno<10 0\fi\the \plineno \hskip 1.5em}}}

% liczenie pustych wierszy:
% \let\engraf\par
% \def\par{\leavevmode\endgraf}

\def\splitlines{\advance\plineno by 1\psublineno=0\everypar{\advance\psublineno by 1\llap{\textcolor{green}{\the\ppageno-\ifnum\plineno<10 0\fi\the \plineno-\the\psublineno \ }}}}

\def\newsplitline{\advance\plineno by 1}

\def\newParkoszpage{\everypar{}}

\catcode`\¹=13
\catcode`\²=13
\catcode`\³=13
\catcode`\⁴=13
\catcode`\⁵=13

\def\newtip#1#2{\def¹{}}
\def\secondtip#1#2{\def²{}}
\def\thirdtip#1#2{\def³{}}
\def\fourthtip#1#2{\def⁴{}}
\def\fifthtip#1#2{\def⁵{}}

\author{\textbf{Marian Kucała}\\(Korekta OCR, skład i transliteracja alfabetu Parkosza: Janusz S. Bień)} 

\title{TEKST TRAKTATU\\[-.5ex]([Fotokopia i] odczytanie)\\[-.5ex]Traktat o ortografii polskiej [Jakuba Parkosza]\\DOI: 10.5281/zenodo.3883863}


\date{\textbf{Licencje}:\\ \relsize{-1}Creative Commons [Wspólnota twórcza]\\[-.4ex]
Uznanie autorstwa --- Na tych samych warunkach --- Polska\\
wersja 2.5 lub późniejsza\\\textit{także}\\
GNU Free Documentation License\\\textit{także}\\
domena publiczna (samo odczytanie)
\textbf{\today}}



\newfontfamily{\Junicode}{JunicodeTwoBeta-Regular.ttf}
\newcommand{\sgl}[1]{{\Junicode\addfontfeature{StylisticSet=10}#1}}

\catcode`\ƀ=\active
\def ƀ{\sgl{b\&\_\_p;\&\_\_s;}}

\catcode`\ɓ=\active
\def ɓ{\sgl{b\&\_\_p;\&\_\_l;}}

\catcode`\ᵽ=\active
\def ᵽ{\sgl{p\&\_\_p;\&\_\_s;}}

\catcode`\ƥ=\active
\def ƥ{\sgl{p\&\_\_p;\&\_\_h;}}

\catcode`\ɬ=\active
\def ɬ{\sgl{l\&\_\_p;\&\_\_l;}}

\catcode`\ʋ=\active
\def ʋ{\sgl{v\&\_\_p;\&\_\_h;}}

\catcode`\ꝿ=\active
\def ꝿ{\sgl{g\&\_\_p;\&\_\_h;}}

\catcode`\ɱ=\active
\def ɱ{\sgl{m\&\_\_p;\&\_\_d;}}

\catcode`\ɲ=\active
\def ɲ{\sgl{n\&\_\_p;\&\_\_d;}}

\catcode`\ç=\active
\def ç{\sgl{c\&\_\_p;\&\_\_h;}}

\catcode`\Ᵽ=\active
\def Ᵽ{\sgl{P\&\_\_p;\&\_\_s;}}

\catcode`\Ƥ=\active
\def Ƥ{\sgl{P\&\_\_p;\&\_\_r;}}

\usepackage{makeidx}\makeindex

\begin{document}
\message{Begin document}
\thispagestyle{empty}
\message{Begin title page}
% \maketitle

\begin{titlepage}
  {\textbf{Marian Kucała}\\Korekta OCR i skład: Janusz S. Bień\\
  Font Junicode: Peter S. Baker} 

\vfill {\relsize{1} {TEKST TRAKTATU\\[-.5ex]([Fotokopia i]
    odczytanie)\\[-.5ex]Traktat o ortografii polskiej [Jakuba
    Parkosza]}\\\relsize{-1} \\Zastepuje
  DOI: 10.5281/zenodo.3883863, 10.13140/RG.2.2.12938.31685/1, 10.13140/RG.2.2.12938.31685}
% DOI: 10.13140/RG.2.2.12938.31685

\vfill
{\textbf{Licencje}:\\ \relsize{-1}Creative Commons [Wspólnota twórcza]\\[-.4ex]
Uznanie autorstwa --- Na tych samych warunkach --- Polska\\
wersja 2.5 lub późniejsza\\\textit{także}\\
GNU Free Documentation License\\\textit{także}\\
domena publiczna (samo odczytanie)\\
\textbf{\today}}

\end{titlepage}

%\newpage
\pagenumbering{roman}


\phantomsection
% wrong encoding in the bookmark!:
%\addcontentsline{toc}{chapter}{Wstęp}
\addcontentsline{toc}{chapter}{Wprowadzenie}

\raggedright

\section*{Wstęp --- fragmenty (s.~5--6)}
\addcontentsline{toc}{section}{Wstęp --- fragmenty (s.~5--6)}
\label{sec:wstp}
Wydanie to i opracowanie jest oparte na badaniach i pracach
dotychczasowych — Jana Łosia, Bolesława Ulanowskiego, Witolda
Taszyckiego, Eyszarda Ganszyńca, Zdzisława Stiebera, także innych. Do
ustaleń tych badaczy wprowadzono tylko niewielkie korektury i
uzupełnienia. Nie chodzi tu bowiem o pokazanie rezultatów jakichś
nowych badań nad Parkoszem, a tym bardziej nowych odkryć, ale o
udostępnienie tego zabytku wszystkim, których interesuje historia
języka polskiego.

Do nazwisk wymienionych wyżej badaczy trzeba dodać nazwiska
recenzentów niniejszego wydania: prof. Juliusza Domańskiego i dra
Józefa Reczka, którzy tu wnieśli znaczny wkład pracy. Prot. Domański
wprowadził w wielu wypadkach korektury do odczytania tekstu
łacińskiego i niemało zmian do tłumaczenia (przede wszystkim wstępu do
traktatu). Dr Reczek zaproponował również szereg poprawek w
odczytaniu,
trochę w tłumaczeniu; jego zasługą jest odczytanie wyrazów:
\textit{siutka} zamiast \textit{siatka}, \textit{wiłuje}
zam. \textit{wilnie}, \textit{żur} zam. \textit{żum}<\textit{p}>.
Wydawca jest PP. Recenzentom bardzo wdzięczny.

Jacobus Parcossii de Zoravice jest nazywany Jakubem Parkoszowicem,
Jakubem synem Parkosza, Jakubem z Żórawic, a najczęściej krótko ---
Parkoszem (Jakubem Parkoszem nazywał go jeszcze za jego życia Zbigniew
Oleśnicki). Tej ostatniej nazwy używamy w niniejszym opracowaniu.

\section*{Zasady wydania (s. 34--35)}
\addcontentsline{toc}{section}{Zasady wydania (s. 34--35)}
\label{sec:zasady-wydania}%{Zasady wydania (s. 34--35)}


Wydanie niniejsze, jak to już powiedziano na wstępie, opiera się na
opracowaniach dotychczasowych, przede wszystkim na wydaniu Jana
Łosia. W odczytaniu tekstu bardzo pomocne było odczytanie Bolesława
Ulanowskiego. I choć różnic w stosunku do tamtego odczytania jest tu
dość sporo (część poprawek jest zasługą recenzentów), to jednak w
miejscach trudnych przyjmuje się najczęściej lekcje Ulanowskiego.


W tekście łacińskim skróty wyrazów rozwiązano bez zaznaczania
tego. Długie \textit{ſ} występujące z reguły na początku i w środku
wyrazów zastąpiono \textit{s} krótkim, zmieniono występowanie liter
\textit{u} i \textit{v}, majuskułową literę oznaczającą [\textit{i}] i
[\textit{j}] podobną do późniejszego \textit{J} oddaje się (zarówno w
wyrazach łacińskich jak polskich) przez \textit{i}, \textit{I};
wprowadzono poza tym parę drobnych zmian --- nie wprowadzono do
pisowni dyftongów. Zmodernizowano duże i małe litery, interpunkcję i
podział na akapity\footnote{W niniejszej wersji elektronicznej
przy podziale na wersy wykorzystano również sugestie Wójcika i Wydry ``{Jakub
    {Parkoszowic}'s {Polish} {Mnemonic} {Verse} about {Polish}
    {Orthography} from the 15th {Century}},'', \textit{Culture of
    Memory in East Central Europe in the Late Middle Ages and the
    Early Modern Period}, 2008, s. 119--127. [JSB]}.

Wyrazy i teksty polskie oddaje się w dokładnej transliteracji, pisząc
tylko znak przestankowy (kropkę) u dołu wiersza, a nie w środku jak w
rękopisie\footnote{Nie zawsze [JSB]}.

Oznaczono strony i wiersze rękopisu (a nie wiersze transkrybowanego
tekstu).  Przekład polski wstępu do traktatu jest oparty na
przekładzie Łosia, jednak oprócz wydawcy niemało zmian wprowadził tu
prof. J. Domański (por. \textit{Wstęp}). W przekładzie samego traktatu
znaczną trudność sprawiała terminologia: określenia głosek i ich
cech. Nie można terminów łacińskich tłumaczyć zbyt dosłownie, nie
można ich też zastępować dzisiejszymi językoznawczymi określeniami
cech fonemów (głosek) — trzeba było wybrać drogę pośrednią. Przekład
terminów pozostawia z pewnością wiele do życzenia, sprawa Parkoszowej
terminologii gramatycznej wymaga dokładniejszego przedstawienia
(wydawca ma zamiar poświęcić temu osobny artykuł).

W tłumaczeniu tekstu wyrazy polskie podaje się w transkrypcji, a tam,
gdzie mowa o pisowni — w transliteracji, dodając transkrypcję w
nawiasach. Długość samogłosek w transkrypcji zaznaczono tylko w
wypadkach oznaczenia jej w rękopisie (litery podwojone) i nie
budzących wątpliwości. Nosówkę w transkrypcji oznaczono przez
\textit{ą}\footnote{\parkosz{ⱥ} i \parkosz{ø} ??? --- JSB}, natomiast
w części wstępnej (przy omawianiu słownictwa) i w hasłach w indeksie
zastosowano \textit{ę} i \textit{ą}.

W komentarzu wyzyskano w znacznym stopniu uwagi Łosia.  Przy podawaniu
znaczeń większości wyrazów opierano się na \textit{Słowniku
  staropolskim}\footnote{\textit{Słownik staropolski},
  red. nacz. S. Urbańczyk, Wrocław\ldots, 1953
  nn.}\footnote{\url{http://rcin.org.pl/publication/39990} ---
  JSB}. Znaczenia wyrazów — ale tylko różne od współczesnych — podaje
się nie w notkach, ale w indeksie (przy wyrazach w staropolszczyźnie
wieloznacznych, a w traktacie nie mających kontekstu, znaczeń się nie
podaje). Indeks jest więc także rodzajem słownika.


Cały komentarz umieszczono w notkach pod tekstem; przy
czytaniu jest to wygodniejsze niż komentarz umieszczony po tekście. W
komentarzu tym poprawki pisowni przykładów podaje się tylko w tych
wypadkach, kiedy dotyczą omawianych głosek i liter.

Przykłady w tekście wyróżniono \parkosz{kursywą}\footnote{kolorem
  zielonym [JSB].}.

Nawiasy ujmują: \extra{} litery i wyrazy zbędne, \add{} litery i wyrazy
dodane (których w tekście brak), \margin{} zapisy na marginesie i nad
rządkami, \overstr{} przekreślenia. Pionowe kreski | oddzielają
wiersze (rządki) rękopisu, kreski podwójne || oddzielają
strony\footnote{nie dotyczy niniejszej wersji elektronicznej [JSB]}.

\newpage
\section*{Uwagi techniczne [JSB]}
\addcontentsline{toc}{section}{Uwagi techniczne [JSB]}
\label{sec:Uwagi techniczne}

{\relsize{-1}

\break Linki typu
\url{https://jsbien.github.io/Parkosz4IIIF/\#?cv=0&manifest=https://jsbien.github.io/Parkosz4IIIF/collection/ParkoszJBC/index.json}
prowadzą do skanu odpowiedniej strony w formacie IIIF i teoretycznie
powinny wyświetlić się w każdej przeglądarce WWW --- wykorzystana jest
przeglądarka IIIF o nazwie \textit{Universal Viewer}. Część adresu
za członem \textit{manifest=}, np.
\url{https://jsbien.github.io/Parkosz4IIIF/collection/ParkoszJBC/index.json}
można użyć do wyświetlenia skanu strony w innej przeglądarce IIIF.

Adresy postaci
\url{https://djvu.szukajwslownikach.uw.edu.pl/parkosz/parkosz.djvu?djvuopts&page=57&zoom=width&showposition=0.5,0.16}
nie są niestety już wspierane przez przeglądarki WWW. Aby z nich
korzystać, należy zainstalować program \textsf{djview4}
(\url{http://djvu.sourceforge.net/djview4.html}) i w programie wkleić
  ten adres do pola \textit{Open Location}.

  Modyfikowalna postać niniejszego tekstu jest dostępna w
  repozytorum
  % \url{https://bitbucket.org/jsbien/parkosz-traktat/} i
  \url{https://github.com/jsbien/Parkosz-traktat}. Zgodnie z zasadami
  licencji modyfikacji może dokonywac każdy zainteresowany bez pytania
  o zgodę.

  Więcej informacji na ten niniejszej edycji można znaleźć m.in. w
  artykułach: \textit{Traktat Parkosza. Eksperymentalna edycja
    elektroniczna}, \textit{Poznańskich Studiach
    Polonistycznych. Seria Językoznawcza}
  (\url{http://pressto.amu.edu.pl/index.php/pspsj/article/view/19852}),
  \textit{Parkosz’s Treatise from a Typographic Point of View}26(1),
  27-69. https://doi.org/10.14746/pspsj.2019.26.1.2, \textit{Scripta
    \& e-Scripta}} Issue No: 22. 2002, p. 37-57
(\url{https://www.ceeol.com/search/article-detail?id=1058326},
\url{http://e-scripta.ilit.bas.bg/archives/year-2022/issue-22/yanush-s-bien-traktatt-na-parkosh-ot-tipografska-gledna-tochka}.

\newpage
\pagenumbering{arabic}
% Package hyperref Warning: The anchor of a bookmark and its parent's must not
% (hyperref)                be the same. Added a new anchor on input line 270.
\phantomsection
\addcontentsline{toc}{chapter}{Odczytanie}
\phantomsection
\addcontentsline{toc}{section}{Strona 1 [3]}
 \input Parkosz01-03
 \newpage
 \phantomsection
\addcontentsline{toc}{section}{Strona 2 [4]}
 \input Parkosz02-04
 \newpage
 \phantomsection
\addcontentsline{toc}{section}{Strona 3 [5]}
 \input Parkosz03-05
 \newpage
 \phantomsection
\addcontentsline{toc}{section}{Strona 4 [6]}
 \input Parkosz04-06
\newpage
 \phantomsection
\addcontentsline{toc}{section}{Strona 5 [7]}
\input Parkosz05-07
\newpage
 \phantomsection
\addcontentsline{toc}{section}{Strona 6 [8]}
\input Parkosz06-08
\newpage
 \phantomsection
\addcontentsline{toc}{section}{Strona 7 [9]}
 \input Parkosz07-09
\newpage
 \phantomsection
\addcontentsline{toc}{section}{Strona 8 [10]}
 \input Parkosz08-10
\newpage
 \phantomsection
\addcontentsline{toc}{section}{Strona 9 [11]}
 \input Parkosz09-11
\newpage
 \phantomsection
\addcontentsline{toc}{section}{Strona 10 [12]}
 \input Parkosz10-12
\newpage
 \phantomsection
\addcontentsline{toc}{section}{Strona 11 [13]}
 \input Parkosz11-13
\newpage
 \phantomsection
\addcontentsline{toc}{section}{Strona 12 [14]}
 \input Parkosz12-14
\newpage
 \phantomsection
\addcontentsline{toc}{section}{Strona 13 [15]}
 \input Parkosz13-15
\newpage
 \phantomsection
\addcontentsline{toc}{section}{Strona 14 [16]}
 \input Parkosz14-16

 \everypar{}
% pomaga na automatyczne wyswietlanie adnotacji?:
 \newpage
 \printindex

\tableofcontents
%\printbibliography


\end{document}



%%% Local Variables: 
%%% mode: LaTeX
%%% TeX-PDF-mode: t
%%% TeX-engine: xetex 
%%% TeX-master: t
%%% End:
